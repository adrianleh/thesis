\chapter{Conclusion}\label{sec:conclusion}

The results, that are inline with the results from~\cite{aebi18bachelorarbeit}, and henceforth strongly suggest, that there is now imperical evidence, that idenpendend of the factor chosen loop unrolling does not yield a signifact performance benefit in the current state of~\libFIRM.
Even though through the added loop optimization, more loops were able to be unrolled, the increase in unrollability was signficantly less than predicted in the preliminary hypothesis, which certainly is a contributing factor to the underwhelming improvements.
Probably some restrictions, such as disallowing \texttt{break}-like structures, are too limited and could be dealt with through further optimizations.
Other restrictions, such as the conservative alias, or call manipulation, checks for the bound are unavoidable, if semantics are to be kept and forthright inherent to the task at hand.
Yet, inconsiderate of these reasons, the eminent challenge seems to be the lack of performance gain through unrolling loops.

Further, loop unrolling alone, will never (significantly) improve execution times without other optimizations improving the unrolled code.
Through the implicit added semantics as for having a specific modulus, respective to $f$, for each unrolled block, an optimization could be created that takes advantage of this and could simplify mathematical operations.
Before this potential is used, the results strongly suggest that it would be a more lucrative endeavor, to stick to less fancy optimizations that can take advantage of the unrolled loop structures, such as potentially automatically parallalizing non-conflicting operations across $f$ threads. % TODO: Common idea - soruce?

The efforts of increasing unrollable loops seems to currently exceed the benefits, due to the lack of optimizations using unrolled loops.
For if the desire for more unroallability, should pick up again, once the opitimzation yields benefits, it would seem a good point, to look at other loop structures, such as loops with breaks or a non-counting loop, unlike the ones examined in this thesis.
