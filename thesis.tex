\documentclass[parskip=full,12pt,a4paper,twoside,headings=openright]{scrreprt}
% switch to scrbook if you want roman page numbers for the front matter
% however scrbook has no 'abstract' environment!
% if your thesis is in english, use "parskip=no" instead

% binding correction (BCOR) von 1cm für Leimbindung
\KOMAoptions{BCOR=1cm}
\KOMAoptions{draft=yes}

\usepackage[utf8]{inputenc} % encoding of sources
\usepackage[T1]{fontenc}
\usepackage{studarbeit}
\usepackage[chapter]{algorithm}
\usepackage{algpseudocode}

\usepackage{indentfirst}
\setlength{\parindent}{5ex}


\title{Schleifenausrollen mit nicht konstanten Grenzen in FIRM}
\author{Adrian E. Lehmann}
\thesistype{Bachelorarbeit}
\zweitgutachter{Prof.~Dr.~rer.~net.~Bernhard~Beckert}
\betreuer{M.~Sc.~Andreas~Fried}
\coverimage{4665389330_d09f3d6b75_z.jpg}
\abgabedatum{\today{\year=2019 \month=0 \day=0}}

\newcommand{\libFIRM}{lib\textsc{Firm}}

\renewcommand\qedsymbol{$\blacksquare$}

\newcommand{\andinN}{\ensuremath{\medspace \cap \medspace \mathbb{N}_0}}
\newcommand{\upto}{\ensuremath{\medspace \boldsymbol{\rightarrow} \medspace}}
\newcommand{\zeroset}{\ensuremath{\{0\}}}
\newcommand{\Mloop}{\ensuremath{M_{\text{loop}}}}
\newcommand{\Mfixup}{\ensuremath{M_{\text{fixup}}}}
\newcommand{\ipl}{\ensuremath{i_\text{post loop}}}
\newcommand{\floor}[1]{\ensuremath{\lfloor{}#1\rfloor{}}}
\newcommand{\ceil}[1]{\ensuremath{\lceil{}#1\rceil{}}}
\newcommand{\mLightning}{\ensuremath{\text{\Lightning}}}
\newcommand{\cinterval}{\ensuremath{
    \begin{cases}
        \lbrack 0, c \cdot (f + 1) \lbrack &, \medspace c > 0\\
        \rbrack c \cdot (f + 1), 0 \rbrack &, \medspace c < 0
    \end{cases}
}}
\newcommand{\NInt}{\ensuremath{\mathbb{N} \cap {\lbrack t_{min}, t_{max} \rbrack}}}
\newcommand{\norm}[1]{\ensuremath{\left\lVert #1 \right\rVert}}


\begin{document}
% New definitions
\algnewcommand\algorithmicswitch{\textbf{switch}}
\algnewcommand\algorithmiccase{\textbf{case}}
% New "environments"
\algdef{SE}[SWITCH]{Switch}{EndSwitch}[1]{\algorithmicswitch\ #1\ \algorithmicdo}{\algorithmicend\ \algorithmicswitch}%
\algdef{SE}[CASE]{Case}{EndCase}[1]{\algorithmiccase\ #1}{\algorithmicend\ \algorithmiccase}%
\algtext*{EndSwitch}%
\algtext*{EndCase}%
\begin{otherlanguage}{ngerman} % Titelseite ist immer auf Deutsch
\mytitlepage
\end{otherlanguage}

\begin{abstract}
\begin{center}\Huge\textbf{\textsf{Zusammenfassung}}
\end{center}
\vfill

Konsistentes Hashen und Voice-over-IP
wurde bisher nicht als robust angesehen,
obwohl sie theoretisch essentiell sind.
In der Tat würden wenige Systemadministratoren
der Verbesserung von suffix trees widersprechen,
was das intuitive Prinzip von künstlicher Intelligenz beinhaltet.
Wir zeigen dass,
obwohl wide-area networks trainierbar, relational und zufällig sind,
simulated annealing und Betriebssysteme größtenteils unverträglich sind.
\vfill

Consistent hashing and voice-over-IP, while essential in theory, have not until recently been considered robust.
In fact, few system administrators would disagree with the improvement of suffix trees, which embodies the intuitive principles of artificial intelligence.
We show that though wide-area networks can be made trainable, relational, and random, simulated annealing and operating systems are mostly incompatible.
\vfill

Ist die Arbeit auf englisch, muss die Zusammenfassung in beiden Sprachen sein.
Ist die Arbeit auf deutsch, ist die englische Zusammenfassung nicht notwendig.
\todo{Das Titelbild ist von
\url{http://www.flickr.com/photos/x-ray_delta_one/4665389330/}
und muss durch ein zum Thema passendes Motiv ausgetauscht werden.}
\end{abstract}

\tableofcontents

\chapter{Introduction}\label{sec:intro}

%In diesem Kapitel wird das Problem vorgestellt, das diese Arbeit löst.
%Es sollte verständlich sein (anschauliches Beispiel?).
%Das Problem sollte wichtig sein,
%denn das motiviert weiterzulesen.

%Bestenfalls ist dieses Kapitel auch für Laien verständlich.

When developers craft code, there is a need to convert it from a human-readable high-level language into a machine-understandable language, called assembly.
In order to do this a programmer will run a \textit{compiler} that checks the code for multiple sources of errors, and, if the code is correct, convert it into an executable file.
Whilst converting the program into machine code the compiler will optimize the code.
This is only beneficial to the developer, as it ensures that his/her application will, in the end, run faster and/or require fewer system resources.
A simple example of an optimziation would be constant folding, where the compiler analyzes code and precalculating all constant values, instead of letting them waste valuable runtime to calculate each time the application runs.
An example can be seen below:
As humans we can immediately see that in~\cref{fig:intro:cf1} that $b$ will always be equal to $9$ and using constant folding the compiler will be able to also perform this computation.
The result of this can be seen in~\cref{fig:intro:cf2}
Henceforth, the runtime of the (admittedly small) program will be reduced, as there is one less calculation required.
Simplifying just one expression seems (and for a matter of fact is) quite useless, but in a real-world code, optimization like this is possible on expressions and hence noticeably improves the final product.

\begin{figure}[h]
    \begin{subfigure}[b]{0.5\textwidth}
        \begin{algorithmic}
            \State $a \gets 7$
            \State $b \gets a + 2$
        \end{algorithmic}
        \caption{Sample code snippet for constant folding}
        \label{fig:intro:cf1}
    \end{subfigure}
    \begin{subfigure}[b]{0.5\textwidth}
        \begin{algorithmic}
            \State $a \gets 7$
            \State $b \gets 9$
        \end{algorithmic}
        \caption{Code with constants folded}
        \label{fig:intro:cf2}
    \end{subfigure}
    \caption{Example of constant folding optimization}
    \label{fig:intro:cf}
\end{figure}



Of course there are many more possibilities to optimize code using a compiler.
As loops make up approximately 10\% of code of many real-world applications\footnote{Measured using gcc (spec2000): 8.6\% of FIRM nodes are in loops}, they were a natural point to focus optimization efforts upon.
Loops can be unrolled fairly straight forward if you know how often they are iterated, as we will discuss in~\cref{sec:basics}.

\begin{figure}
    \begin{subfigure}[b]{0.5\textwidth}
        \begin{algorithmic}
            \State $i \gets 0$
            \While{$i < 5$}
            \State \Call{Print}{$i$}
            \State $i \gets i + 1$
            \EndWhile
        \end{algorithmic}
        \subfigfill
        \caption{Loop with constant bounds}
        \label{fig:intro:unroll-simple-before}
    \end{subfigure}
    \begin{subfigure}[b]{0.5\textwidth}
        \begin{algorithmic}
            \State \Call{Print}{$0$}
            \State \Call{Print}{$1$}
            \State \Call{Print}{$2$}
            \State \Call{Print}{$3$}
            \State \Call{Print}{$4$}
        \end{algorithmic}
        \subfigfill
        \caption{Loop with constant bounds unrolled}
        \label{fig:intro:unroll-simple-after}
    \end{subfigure}
    \caption{Unrolling a loop with constant bounds}
    \label{fig:intro:unroll-simple}
\end{figure}

For example, we can see that~\cref{fig:intro:unroll-simple-before} can be easily converted to~\cref{fig:intro:unroll-simple-after}, whilst keeping all semantics in tact.
In~\cref{fig:intro:unroll-nostatic-bound} things get trickier, as we do not know what the exact value of $N$ is going to be, we will not be able to simply unroll a loop by copying its body a fixed number of times.
In~\cref{sec:basics} it will be exactly described, as to what we can do about these kinds of loops, and in~\cref{sec:impl} the algorithms and techniques to unroll these loops will be discussed.
Finally, in~\cref{sec:eval} the approach will be experimentally evaluated to see whether it yields a tangible benefit.
\begin{figure}[h]
    \begin{algorithmic}
        \State $i \gets 0$
        \State $N \gets$ \Call{FairDiceRoll}{}() \Comment{Random number in $[1,6]$}
        \While{$i < N$}
            \State \Call{Print}{$i$}
            \State $i \gets i + 1$
        \EndWhile
    \end{algorithmic}
    \caption{Loop without constant bound}
    \label{fig:intro:unroll-nostatic-bound}
\end{figure}


\chapter{Basics and related work}\label{sec:basics}

% Compiler
\section{Compiler}\label{sec:basics:compiler}

The primary function of a compiler is to automatically convert high-level code created by a developer into (optimized) machine code.
As a compiler is an inherently large software project, an architecture needs to be chosen that allows for extensions and modifications.
Modern compilers mostly follow a layered architecture style: They are each comprised of a front-, middle-, and back-end.
In this architecture, the front-end converts the high-level code into an abstract intermediary representation, which is then used by the middle-end for optimizations and transformations.
Lastly, the back-end is responsible for converting the optimized intermediary code into instructions for the target system architecture (e.g., RISC-V, x86, ARM, or similar).

% Basic blocks
\section{Basic blocks and control-flow}\label{sec:basics:bb-cf}

To better handle code and give it a logical structure, most compilers divide code up into so-called \textit{basic blocks}.
Basic blocks are sets of consecutive operations that do not contain jumps or targets thereof, but rather only jumps connecting them.
Therefore, a basic block is either executed entirely or not executed at all.

A usual way to represent this in a human-readable form is to output it as a control-flow-graph (CFG).
CFGs depict basic blocks as nodes and jumps between basic blocks as edges.
Furthermore, it is a convention in these graphs to have precisely one start-node and one end-node.

We note that CFGs, in general, are cyclical graphs.
They are non-cyclical graphs if the original code does not contain any jumps going backward in the control-flow.

Another important concept of CFGs is dominance.
To explain this concept, we define a starting node $S$, and assume there are two (not necessarily different) nodes, present in the CFG, $N_1$, and $N_2$.
With this information, we define dominance as follows:
$$N_1~\text{dominates}~N_2 \Longleftrightarrow \forall p \in \text{Paths}(S, N_2):~N_1 \in p$$
In lucid terms, this means $N_1$ dominates $N_2$, iff to get to $N_2$ from $S$ you have to visit $N_1$ on the way.
It is important to note that a block always dominates itself.

% Lifetime

\section{Loops}\label{sec:basics:loops}

We define a loop to be a set of nodes that are all in a cyclical control-flow structure.
Formally this can be expressed as:

$$L~\text{is a loop} \Longleftrightarrow L \neq \emptyset \wedge \forall n_1, n_2 \in L:~\exists Path(n_1, n_2) \subseteq L$$

Henceforth let $L$ be a loop.

If a loop is completely contained inside another loop, it is said to be nested.

$$L'~\text{is nested in}~L \Longleftrightarrow L' \subsetneq L \wedge L'~\text{is a loop}$$

If a loop has no nested loops inside of it, we will henceforth call it the \textit{innermost} loop.

$$L~\text{is innermost loop} \Longleftrightarrow \nexists L': L'~\text{is a nested loop in}~L$$

Loops can furthermore have a header, which is the sole entry point into a loop~\cite{aebi18bachelorarbeit} and defined as follows:

$$H~\text{is header of}~L \Longleftrightarrow H \in L \wedge \forall n \in L: H~\text{dominates}~n$$

N.B.: Not all loops have to have a header.

If a loop has a header, its body is the set containing all blocks in the loop, except for the header.
$$B~\text{is body of}~L \Longleftrightarrow B = L \backslash \{H\}, H~\text{is header of}~L$$

% SSA
\section{Single-Static-Assignment (SSA)}\label{sec:basics:ssa}

The \textit{single-static-assignment} (\textit{SSA}) form is a property of intermediary representations, that requires each value to only be assigned exactly once.
Moreover, every value has to be assigned before it is being used~\cite{cytron91}.
This property mainly implies that the block, which declares a given value $v$, has to dominate all blocks that use $v$.
This declaration point will be unambiguous across all possible usages.

The SSA form is used to simplify optimizations in the regard that one can be sure that a set point in the code currently defines a given value in use.

\Cref{fig:basics:SSA-simple} shows an example program in SSA form.
While in the base code $x$ is assigned twice, in the code transformed into SSA form, simply a new value was defined to make sure that each variable is only defined once.

\begin{figure}[h]
    \begin{subfigure}[b]{0.5\textwidth}
        \centering
        \begin{algorithmic}
            \State $x \gets 1$
            \State \Call{Print}{$x$}
            \State $x \gets 7$
            \State \Call{Print}{$x$}
        \end{algorithmic}
        \caption{An example piece of code not in SSA form}
    \end{subfigure}
    \begin{subfigure}[b]{0.5\textwidth}
        \centering
        \begin{algorithmic}
            \State $x_1 \gets 1$
            \State \Call{Print}{$x_1$}
            \State $x_2 \gets 7$
            \State \Call{Print}{$x_2$}
        \end{algorithmic}
        \caption{Thre same code in SSA form}
    \end{subfigure}
    \caption{An example program in SSA form}
    \label{fig:basics:SSA-simple}
\end{figure}

In a loop or a conditional statement, a scenario might arise where a given value could be assigned at multiple locations.
In cases like these we can use a \textit{$\Phi$-function}.
A $\Phi$-function is a theoretical construct that returns the correct value depending on the control-flow predecessor.

\Cref{fig:basics:SSA-phi} shows an example of the use of a $\Phi$-function, where depending on the control-flow, either $m_1$ or $m_2$  are selected.
\begin{figure}[h]
    \begin{subfigure}[b]{0.5\textwidth}
        \centering
        \begin{algorithmic}
            \Function{max}{$a: \mathbb{N},b: \mathbb{N}$}
            \State $m: \mathbb{N}$
            \If{a > b}
            \State $m \gets a$
            \Else
            \State $m \gets b$
            \EndIf
            \State \Return $m$
            \EndFunction
        \end{algorithmic}
        \subfigfill
        \caption{Non-SSA Code for a function that returns the maximum of its parameters}
    \end{subfigure}
    \begin{subfigure}[b]{0.5\textwidth}
        \centering
        \begin{algorithmic}
            \Function{max}{$a: \mathbb{N},b: \mathbb{N}$}
            \State $m: \mathbb{N}$
            \If{$a > b$}
            \State $m_1 \gets a$
            \Else
            \State $m_2 \gets b$
            \EndIf
            \State $m \gets\text{ }$ \Call{$\Phi$}{$m_1, m_2$}
            \State \Return m
            \EndFunction
        \end{algorithmic}
        \subfigfill
        \caption{Same Function converted into SSA}
    \end{subfigure}
    \caption{An example program transformed into SSA form}
    \label{fig:basics:SSA-phi}
\end{figure}

% LCSAA

\section{Loop-Closed-Single-Static-Assignment (LCSSA)}\label{sec:basics:LCSSA}

An extension to the aforementioned SSA form is the \textit{loop-closed-single-static-assignment} (\textit{LCSSA}) form.
A CFG in LCSSA form has all properties that a CFG in SSA form has, and additionally the property that each value assigned in a given loop and used outside of this loop has to be used by a $\Phi$-node in the first block after the loop.
This form is used to reduce special casing when transforming loops~\cite{aebi18bachelorarbeit}, and is therefore used through all of~\cref{sec:impl}.

To visualize this property,~\cref{fig:basics:LCSSA} depicts its effect.

\begin{figure}[h]
    \begin{subfigure}[b]{0.5\textwidth}
        \centering
        \begin{algorithmic}
            \Function{Foo}{}
            \State $x, y: \mathbb{N}$
            \Repeat
            \If{\Call{Condition}{}}
            \State $x_1 \gets 5$
            \Else
            \State $x_2 \gets 8$
            \EndIf
            \State $x \gets \Phi(x_1, x_2)$
            \Until{\Call{OtherCondition}{}}
            \State $y \gets x + 3$
            \EndFunction
        \end{algorithmic}
        \caption{Sample loop in SSA form}
    \end{subfigure}
    \begin{subfigure}[b]{0.5\textwidth}
        \centering
        \begin{algorithmic}
            \Function{Foo}{}
            \State $x, y: \mathbb{N}$
            \Repeat
            \If{\Call{Condition}{}}
            \State $x_1 \gets 5$
            \Else
            \State $x_2 \gets 8$
            \EndIf
            \State $x_3 \gets \Phi(x_1, x_2)$
            \Until{\Call{OtherCondition}{}}
            \State $x \gets \Phi(x_3)$
            \State $y \gets x + 3$
            \EndFunction
        \end{algorithmic}
        \caption{Same sample loop in LCSSA form}
    \end{subfigure}
    \caption{An example program transformed into LCSSA Form (adapted from~\cite{LLVM_LCSSA})}
    \label{fig:basics:LCSSA}
\end{figure}


% FIRM
\section{\libFIRM}\label{sec:basics:firm}

\libFIRM{} is a compiler middle- and back-end that takes a graph-based intermediate representation in SSA form, optimizes it, and produces assembly code~\cite{libfirm}.
Since 1996, Karlsruhe Institute of Technology (KIT) actively develops~\libFIRM.

A graph in \libFIRM{} contains information about basics blocks, the control-flow, and memory and data dependencies.
Basics blocks in \libFIRM{} contain further nodes that are responsible for the control-flow of the program.
These are pointed to by (other) basic blocks that are the target of these control-flow operations.
The resulting control-flow edges are represented by red edges in visualizations of \libFIRM{} graphs.
Any node operating on memory also connects to the previous node operating on memory, so that the node uses the prior state of memory and then provides a new state with its changes.
Memory is, like control flow, connected by edges, which are colored blue in graphical representations.
Lastly, \libFIRM{} has data dependency edges between nodes, which represent dependencies needed for calculations.

Another set of functionality that~\libFIRM{}  provides is loop information.
\libFIRM{} will not only (if applicable) be able to map blocks to their respective loops and vice-versa, but also has information on loop nesting structure.
Thus one can quickly determine, whether a loop is an innermost loop~\cite{libfirm}.
Further,~\libFIRM{} also allows for finding the basics block that is the header of a loop, given that it has a header~\cite{aebi18bachelorarbeit}.

\Cref{fig:basics:firm} portrays an example firm graph of the program initially shown in~\cref{fig:basics:SSA-phi}.
It is especially to be noted that the graph is in SSA form, since it contains a $\Phi$-node, and has both memory, data and control-flow edges.
\begin{figure}
    \centering
    % Scale factor 0.04065040650406504
\definecolor{color10}{RGB}{204,204,204}
\definecolor{color11}{RGB}{222,239,234}
\definecolor{color12}{RGB}{255,153,255}
\definecolor{color13}{RGB}{255,153,153}
\definecolor{color14}{RGB}{153,255,153}
\definecolor{color15}{RGB}{0,150,60}
\definecolor{color16}{RGB}{153,153,255}
\definecolor{color17}{RGB}{255,255,255}
\definecolor{color18}{RGB}{255,0,0}
\definecolor{color19}{RGB}{0,0,0}
\definecolor{color20}{RGB}{100,100,255}
% Bounding Box: 369.0, 862.0
\begin{tikzpicture}[scale=0.5, transform shape]
	\node[fill=color10, draw, minimum width=14.96951219512195cm, minimum height=35.040650406504064cm] (n26) at (7.484756097560975cm ,17.520325203252032cm) {};
	% 1 node layouts
	\node[scale=1.4781966001478197, transform shape] at (7.484756097560975cm ,34.57634654471545cm) {max};
	\node[fill=color11, draw, minimum width=13.75cm, minimum height=17.601626016260163cm] (n27) at (7.484756097560975cm ,24.695121951219512cm) {};
	% 1 node layouts
	\node[scale=1.4781966001478197, transform shape] at (7.484756097560975cm ,33.031631097560975cm) {Start Block  56};
	\node[fill=color11, draw, minimum width=5.447154471544716cm, minimum height=8.658536585365853cm] (n28) at (9.034552845528456cm ,9.939024390243903cm) {};
	% 1 node layouts
	\node[scale=1.4781966001478197, transform shape] at (9.034552845528456cm ,13.80398882113821cm) {Block  93};
	\node[fill=color11, draw, minimum width=3.6585365853658547cm, minimum height=3.3739837398373984cm] (n29) at (9.481707317073171cm ,2.296747967479675cm) {};
	% 1 node layouts
	\node[scale=1.4781966001478197, transform shape] at (9.481707317073171cm ,3.519435975609756cm) {End Block  54};
	\node[fill=color12, draw, minimum width=2.4390243902439024cm, minimum height=1.2195121951219512cm] (n30) at (9.48170731707317cm ,1.8292682926829267cm) {};
	% 1 node layouts
	\node[scale=1.4781966001478197, transform shape] at (9.48170731707317cm ,1.8292682926829267cm) {End  55};
	\node[fill=color13, draw, minimum width=3.089430894308943cm, minimum height=1.2195121951219512cm] (n31) at (9.48170731707317cm ,6.829268292682927cm) {};
	% 1 node layouts
	\node[scale=1.4781966001478197, transform shape] at (9.48170731707317cm ,6.829268292682927cm) {Return  96};
	\node[fill=color14, draw, minimum width=2.6016260162601625cm, minimum height=1.2195121951219512cm] (n32) at (8.221544715447154cm ,9.674796747967479cm) {};
	% 1 node layouts
	\node[scale=1.4781966001478197, transform shape] at (8.221544715447154cm ,9.674796747967479cm) {Phi Is 95};
	\node[fill=color15, draw, minimum width=0.8130081300813008cm, minimum height=0.8130081300813008cm] (n33) at (10.741869918699187cm ,9.674796747967479cm) {};
	\node[fill=color15, draw, minimum width=0.8130081300813008cm, minimum height=0.8130081300813008cm] (n34) at (9.034552845528454cm ,12.317073170731707cm) {};
	\node[fill=color15, draw, minimum width=0.8130081300813008cm, minimum height=0.8130081300813008cm] (n35) at (7.4085365853658525cm ,12.317073170731707cm) {};
	\node[fill=color16, draw, minimum width=3.4552845528455283cm, minimum height=1.2195121951219512cm] (n36) at (2.9471544715447155cm ,28.495934959349594cm) {};
	% 1 node layouts
	\node[scale=1.4781966001478197, transform shape] at (2.9471544715447155cm ,28.495934959349594cm) {Proj M M 62};
	\node[fill=color13, draw, minimum width=4.0650406504065035cm, minimum height=1.2195121951219512cm] (n37) at (6.6361788617886175cm ,17.11382113821138cm) {};
	% 1 node layouts
	\node[scale=1.4781966001478197, transform shape] at (6.6361788617886175cm ,17.11382113821138cm) {Proj X false 68};
	\node[fill=color13, draw, minimum width=3.902439024390244cm, minimum height=1.2195121951219512cm] (n38) at (11.432926829268293cm ,17.11382113821138cm) {};
	% 1 node layouts
	\node[scale=1.4781966001478197, transform shape] at (11.432926829268293cm ,17.11382113821138cm) {Proj X true 67};
	\node[fill=color13, draw, minimum width=2.7642276422764227cm, minimum height=1.2195121951219512cm] (n39) at (9.034552845528456cm ,19.959349593495933cm) {};
	% 1 node layouts
	\node[scale=1.4781966001478197, transform shape] at (9.034552845528456cm ,19.959349593495933cm) {Cond  66};
	\node[fill=color17, draw, minimum width=4.349593495934959cm, minimum height=1.2195121951219512cm] (n40) at (9.034552845528456cm ,22.804878048780488cm) {};
	% 1 node layouts
	\node[scale=1.4781966001478197, transform shape] at (9.034552845528456cm ,22.804878048780488cm) {Cmp b less 128};
	\node[fill=color17, draw, minimum width=4.308943089430894cm, minimum height=1.2195121951219512cm] (n41) at (6.473577235772358cm ,25.65040650406504cm) {};
	% 1 node layouts
	\node[scale=1.4781966001478197, transform shape] at (6.473577235772358cm ,25.65040650406504cm) {Proj Is Arg 1 64};
	\node[fill=color17, draw, minimum width=4.308943089430894cm, minimum height=1.2195121951219512cm] (n42) at (11.595528455284553cm ,25.65040650406504cm) {};
	% 1 node layouts
	\node[scale=1.4781966001478197, transform shape] at (11.595528455284553cm ,25.65040650406504cm) {Proj Is Arg 0 63};
	\node[fill=color17, draw, minimum width=4.5528455284552845cm, minimum height=1.2195121951219512cm] (n43) at (9.034552845528456cm ,28.495934959349594cm) {};
	% 1 node layouts
	\node[scale=1.4781966001478197, transform shape] at (9.034552845528456cm ,28.495934959349594cm) {Proj T T\_args 61};
	\node[fill=color12, draw, minimum width=2.6016260162601625cm, minimum height=1.2195121951219512cm] (n44) at (6.626016260162602cm ,31.341463414634145cm) {};
	% 1 node layouts
	\node[scale=1.4781966001478197, transform shape] at (6.626016260162602cm ,31.341463414634145cm) {Start  59};
	\node[fill=color15, draw, minimum width=0.8130081300813008cm, minimum height=0.8130081300813008cm] (n45) at (2.9471544715447155cm ,25.65040650406504cm) {};
	\node[fill=color15, draw, minimum width=0.8130081300813008cm, minimum height=0.8130081300813008cm] (n46) at (5.320121951219512cm ,22.804878048780488cm) {};
	\node[fill=color15, draw, minimum width=0.8130081300813008cm, minimum height=0.8130081300813008cm] (n47) at (12.748983739837398cm ,22.804878048780488cm) {};
	\draw[color=color18, -latex] (9.481707317073171cm ,3.9837398373983737cm) -- (9.48170731707317cm ,6.219512195121951cm);
	\node[] at (9.80691056910569cm ,4.763878302845528cm) {
		\scalebox{1.4781966001478197}{0}
	};
	\draw[color=color18, -latex] (7.6727642276422765cm ,14.268292682926829cm) -- (7.6727642276422765cm ,15.081300813008129cm) -- (6.6361788617886175cm ,15.081300813008129cm) -- (6.6361788617886175cm ,16.504065040650406cm);
	\node[] at (7.997967479674797cm ,14.860264227642276cm) {
		\scalebox{1.4781966001478197}{0}
	};
	\draw[color=color18, -latex] (10.396341463414634cm ,14.268292682926829cm) -- (10.396341463414634cm ,15.081300813008129cm) -- (11.432926829268293cm ,15.081300813008129cm) -- (11.432926829268293cm ,16.504065040650406cm);
	\node[] at (10.721544715447154cm ,14.860264227642276cm) {
		\scalebox{1.4781966001478197}{1}
	};
	\draw[color=color19, -latex] (8.709349593495933cm ,7.439024390243902cm) -- (8.709349593495933cm ,8.252032520325203cm) -- (8.221544715447154cm ,8.252032520325203cm) -- (8.221544715447154cm ,9.065040650406504cm);
	\node[] at (9.034552845528454cm ,8.03099593495935cm) {
		\scalebox{1.4781966001478197}{1}
	};
	\draw[color=color20, -latex] (10.254065040650405cm ,7.439024390243902cm) -- (10.254065040650405cm ,8.252032520325203cm) -- (10.741869918699187cm ,8.252032520325203cm) -- (10.741869918699187cm ,9.268292682926829cm);
	\node[] at (10.579268292682926cm ,8.03099593495935cm) {
		\scalebox{1.4781966001478197}{0}
	};
	\draw[color=color19, -latex] (8.871951219512194cm ,10.284552845528456cm) -- (8.871951219512194cm ,11.097560975609756cm) -- (9.034552845528454cm ,11.097560975609756cm) -- (9.034552845528454cm ,11.910569105691057cm);
	\node[] at (9.197154471544714cm ,10.876524390243903cm) {
		\scalebox{1.4781966001478197}{0}
	};
	\draw[color=color19, -latex] (7.571138211382112cm ,10.284552845528456cm) -- (7.571138211382112cm ,11.097560975609756cm) -- (7.4085365853658525cm ,11.097560975609756cm) -- (7.4085365853658525cm ,11.910569105691057cm);
	\node[] at (7.8963414634146325cm ,10.876524390243903cm) {
		\scalebox{1.4781966001478197}{1}
	};
	\draw[color=color20, -latex] (2.9471544715447155cm ,29.10569105691057cm) -- (2.9471544715447155cm ,29.91869918699187cm) -- (5.975609756097561cm ,29.91869918699187cm) -- (5.975609756097561cm ,30.73170731707317cm);
	\node[] at (3.272357723577236cm ,29.697662601626014cm) {
		\scalebox{1.4781966001478197}{0}
	};
	\draw[color=color18, -latex] (6.6361788617886175cm ,17.723577235772357cm) -- (6.6361788617886175cm ,18.536585365853657cm) -- (8.34349593495935cm ,18.536585365853657cm) -- (8.34349593495935cm ,19.349593495934958cm);
	\node[] at (6.961382113821138cm ,18.315548780487806cm) {
		\scalebox{1.4781966001478197}{0}
	};
	\draw[color=color18, -latex] (11.432926829268293cm ,17.723577235772357cm) -- (11.432926829268293cm ,18.536585365853657cm) -- (9.72560975609756cm ,18.536585365853657cm) -- (9.72560975609756cm ,19.349593495934958cm);
	\node[] at (11.758130081300813cm ,18.315548780487806cm) {
		\scalebox{1.4781966001478197}{0}
	};
	\draw[color=color19, -latex] (9.034552845528456cm ,20.56910569105691cm) -- (9.034552845528456cm ,22.195121951219512cm);
	\node[] at (9.359756097560975cm ,21.349244156504064cm) {
		\scalebox{1.4781966001478197}{0}
	};
	\draw[color=color19, -latex] (7.9471544715447155cm ,23.414634146341463cm) -- (7.9471544715447155cm ,24.227642276422763cm) -- (7.550813008130081cm ,24.227642276422763cm) -- (7.550813008130081cm ,25.040650406504064cm);
	\node[] at (8.272357723577235cm ,24.00660569105691cm) {
		\scalebox{1.4781966001478197}{0}
	};
	\draw[color=color19, -latex] (10.121951219512194cm ,23.414634146341463cm) -- (10.121951219512194cm ,24.227642276422763cm) -- (10.518292682926829cm ,24.227642276422763cm) -- (10.518292682926829cm ,25.040650406504064cm);
	\node[] at (10.447154471544716cm ,24.00660569105691cm) {
		\scalebox{1.4781966001478197}{1}
	};
	\draw[color=color19, -latex] (6.473577235772358cm ,26.260162601626014cm) -- (6.473577235772358cm ,27.073170731707318cm) -- (7.896341463414634cm ,27.073170731707318cm) -- (7.896341463414634cm ,27.88617886178862cm);
	\node[] at (6.798780487804878cm ,26.852134146341463cm) {
		\scalebox{1.4781966001478197}{0}
	};
	\draw[color=color19, -latex] (11.595528455284553cm ,26.260162601626014cm) -- (11.595528455284553cm ,27.073170731707318cm) -- (10.172764227642276cm ,27.073170731707318cm) -- (10.172764227642276cm ,27.88617886178862cm);
	\node[] at (11.920731707317072cm ,26.852134146341463cm) {
		\scalebox{1.4781966001478197}{0}
	};
	\draw[color=color19, -latex] (9.034552845528456cm ,29.10569105691057cm) -- (9.034552845528456cm ,29.91869918699187cm) -- (7.276422764227642cm ,29.91869918699187cm) -- (7.276422764227642cm ,30.73170731707317cm);
	\node[] at (9.359756097560975cm ,29.697662601626014cm) {
		\scalebox{1.4781966001478197}{0}
	};
	\draw[color=color20, -latex] (2.9471544715447155cm ,26.05691056910569cm) -- (2.9471544715447155cm ,27.88617886178862cm);
	\node[] at (3.272357723577236cm ,26.837049034552845cm) {
		\scalebox{1.4781966001478197}{0}
	};
	\draw[color=color19, -latex] (5.320121951219512cm ,23.211382113821138cm) -- (5.320121951219512cm ,24.227642276422763cm) -- (5.396341463414634cm ,24.227642276422763cm) -- (5.396341463414634cm ,25.040650406504064cm);
	\node[] at (5.645325203252033cm ,23.90497967479675cm) {
		\scalebox{1.4781966001478197}{0}
	};
	\draw[color=color19, -latex] (12.748983739837398cm ,23.211382113821138cm) -- (12.748983739837398cm ,24.227642276422763cm) -- (12.672764227642276cm ,24.227642276422763cm) -- (12.672764227642276cm ,25.040650406504064cm);
	\node[] at (13.074186991869919cm ,23.90497967479675cm) {
		\scalebox{1.4781966001478197}{1}
	};
\end{tikzpicture}

    \caption{A firm graph of a maximum function that returns the larger argument}
    \label{fig:basics:firm}
\end{figure}


% Unroll loops
\section{Loop unrolling}\label{sec:basics:unrolling}

Loop unrolling is a compiler optimization that attempts to duplicate the loop body, and to reduce the controlling instructions, such as the loop condition or repetitive arithmetic~\cite{aho_ullman_1979}.

\Cref{fig:basics:old-loop-unrolling} shows a pseudo-code example of unrolling a simple loop with a factor (the number of times the body is copied) of four.
It is to be noted that the loop condition has to be checked less often, on account of each loop iteration being four times as long as in the original program.

Further, it could also be used to vectorize the code, eliminate repeating conditions, and for many other following optimizations~\cite{fog_2018}.
A negative side effect of loop unrolling is that the binary size increases and that there could be more pressure on the code cache and registers, causing more spilled values~\cite{Sarkar2001}.

\libFIRM{} supports a restricted form of loop unrolling for loops that have static bounds and increments~\cite{helmer10studienarbeit}.
This optimization was recently improved, but now requires the intermediary representation to be in LCSSA form, which means \libFIRM{}'s intermediary representation has to be converted into LCSSA form prior to the optimization running~\cite{aebi18bachelorarbeit}.
Though this optimization had no preconditions and merely duplicated the loop in the hope a later optimization would remove the duplicated headers, so that it resembles an unrolled loop of our definition, in which only the body (and not the header) is duplicated.
In this form the amount of conditional jumps does not decrease in most cases, since the later optimization removing headers would not trigger.
Hence, the application of this method showed no improvements, which caused preconditions for the later removal to be added~\cite{libfirm-unroll-static}.
Currently, it only unrolls loops, like our loop, only with static bounds and for which a constant bit analysis can remove excess headers.
The benefits of the now changed optimization were also very slim, likely since the requirements for a loop to now be unrollable are very strict.
With these restrictions in place, only approximately 5\% of the innermost loops can be unrolled\footnote{Measured in \texttt{spec2006}}.


\begin{figure}[H]
    \begin{subfigure}[b]{0.5\textwidth}
        \centering
        \begin{algorithmic}
            \Function{Foo}{}
                \State $i \gets 0$
                \While{$i < 16$}
                    \State \Call{Print}{$i$}
                    \State $i \gets i + 1$
                \EndWhile
            \EndFunction
        \end{algorithmic}
        \caption{A function with a simple loop inside of it}
    \end{subfigure}
    \begin{subfigure}[b]{0.5\textwidth}
        \centering
        \begin{algorithmic}
            \Function{FooUnrolled}{}
                \State $i \gets 0$
                \While{$i < 16$}
                    \State \Call{Print}{$i$}
                    \State \Call{Print}{$i + 1$}
                    \State \Call{Print}{$i + 2$}
                    \State \Call{Print}{$i + 3$}
                    \State $i \gets i + 4$
                \EndWhile
            \EndFunction
        \end{algorithmic}
        \caption{A function with the same loop unrolled}
    \end{subfigure}
    \caption{A simple loop unrolled using firm}
    \label{fig:basics:old-loop-unrolling}
\end{figure}

% Duff's Device
\section{Duff's device}\label{sec:basics:duffs}

A common problem with the loop unrolling shown in~\cref{fig:basics:old-loop-unrolling} is that it requires the number of iterations to be constant and divisible by the unroll factor.
A way to tackle this issue is to use a construct known as Duff's device: It preemptively unrolls a loop with a given factor and uses~\textit{fixup} code to ensure that the remaining iterations are completed~\cite{duff_1983}.
Mathematically this means the construct executes the loop body $\floor{\frac{M}{f}} \cdot f + M \mod F = M$ times, where $M$ is the number of total times the loop body would be executed without the transformation and where $f$ is the unroll factor.
This is due to the fact that mod is defined as:
$$x \mod{} y = x - \floor{\frac{x}{y}} \cdot y$$
If we substitute $M$ for $x$ and $f$ for $y$ and rearrange for $M$, we get the aforementioned form.

\Cref{fig:basics:duff} shows an example of unrolling a loop with a non-divisible bound using a factor of eight\footnote{The original Duff's device used special C syntax to entangle the switch statement and loop~\cite{duff_1983}}.
Duff's device copies the loop body eight times and to ensure that the number of executions is correct, the first time around the code jumps to the corresponding instruction, depending on the need for fixup code.

Many compilers, such as GCC~\cite{gcc}, use Duff's device for unrolling loops and improving performance while keeping code size relatively small.
Further \libFIRM{} previously utilized Duff's device for unrolling loops with static bounds, but for which no unroll factor could be determined~\cite{helmer10studienarbeit}.

\begin{figure}
    \begin{subfigure}[b]{0.5\textwidth}
        \centering
        \begin{algorithmic}
            \Function{Foo}{$N: \mathbb{N}$}
                \State $i \gets 0$
                \While{$i < N$}
                    \State \Call{Print}{$i$}
                    \State $i \gets i + 1$
                \EndWhile
            \EndFunction
        \end{algorithmic}
        \caption{An example function with a loop}
    \end{subfigure}
    \begin{subfigure}[b]{0.5\textwidth}
        \centering
        \begin{algorithmic}
            \Function{FooDuffed}{$N: \mathbb{N}$}
                \State $i \gets 0$
                \Switch{$N \text{ mod } 4$}
                    \Case{$3$}
                        \State \Call{Print}{$i$}
                        \State $i \gets i + 1$ \Comment{Fall-through}
                    \EndCase
                    \Case{$2$}
                        \State \Call{Print}{$i$}
                        \State $i \gets i + 1$ \Comment{Fall-through}
                    \EndCase
                    \Case{$1$}
                        \State \Call{Print}{$i$}
                        \State $i \gets i + 1$
                    \EndCase
                \EndSwitch
                \While{$i < N$}
                    \State \Call{Print}{$i$}
                    \State \Call{Print}{$i + 1$}
                    \State \Call{Print}{$i + 2$}
                    \State \Call{Print}{$i + 3$}
                    \State $i \gets i + 4$
                \EndWhile
            \EndFunction
        \end{algorithmic}
        \caption{A function with the loop unrolled using Duff's device}
    \end{subfigure}
    \caption{A simple loop unrolled using Duff's device}
    \label{fig:basics:duff}
\end{figure}

\section{Overflow detection}\label{sec:basics:overflow}

When subtracting (or adding) two numbers that are in some form of integer-like representation the operation might over- or underflow, because the integer representation is a fixed bit two's complement representation.
Due to the therefore inherent limitation to the range of possible values, this problem is unavoidable, yet detectable.

\Cref{alg:basics:overflow:detect}~\cite{pmg_2009} shows a way to detect whether an overflow or underflow occurs for an operation $x - a, \text{ where } x, a \in \NInt$, by checking whether the result increased or decreased relative to the bounds and comparing it to the expectation.
\begin{algorithm}[H]
    \begin{algorithmic}
        \Function{SubtractionWillLeaveBounds}{$x, a: \ZInt$}
            \State $overflow \gets$\Call{HightestBitSet}{$x$}$ \wedge (a > t_{max} + x)$
            \State $underflow \gets (x > 0) \wedge (a < t_{min} + x)$
            \State \Return $overflow \vee underflow$
        \EndFunction
    \end{algorithmic}
    \caption{Algorithm that detects whether the operation $x - a$ will go out of the integer boundaries}
    \label{alg:basics:overflow:detect}
\end{algorithm}


\chapter{Design and implementation}\label{sec:impl}

In order to unroll a loop with non-static bounds, this thesis follows a specific approach:
First, we check whether we are able to unroll the loop.
\Cref{sec:impl:unrollability} describes the conditions necessary and how we check them.
If we determine a loop to be unrollable, we will unroll it with the unrolling process covered in \Cref{sec:impl:unroll}.
Once this process is complete, the loop condition of the unrolled loop will be adapted to make sure it runs less than or equal times compared to the original loop.
This is described in~\Cref{sec:impl:fixup:header-cond}.
After that, we will create the \textit{fixup code}\footnote{The term \textit{fixup code} describes that code that has to be added to account for cases where the number of times the loop is executed modulo the unrolling factor is not equal to zero.}
, as described in \Cref{sec:impl:fixup}.

It is to be noted that in terms of actually implementing this procedure, we will create the fixup code \textit{before} unrolling the loop.
While this order seems counter-intuitive, we chose it in order to simplify the implementation of loop duplication, as described in~\Cref{sec:impl:fixup:loop}.


Henceforth, we assume loops to be in the form of the loop shown in~\Cref{fig:impl:general-loop}.
In the reference loop \textit{cmp} refers to a comparison that can be one of the following: $<, >, \geq, \leq$.
Further, $I \in \mathbb{Z}$ refers to the starting value, $N \in \mathbb{Z}$ to the bound, and $c \in \mathbb{Z} \backslash \zeroset$ \label{sec:impl::def-c} to the increment\footnote{N.B.: $c$ may be negative and could hence also be a decrement} of such a loop.
We select this form in view of the fact that most loops follow the form of using a counter or iterating over a given container, which condenses down to this form.
Furthermore, this form allows for many arithmetic properties to be used, as seen in~\Cref{sec:impl:unroll}.

\begin{figure}[H]
    \centering
    \begin{algorithmic}
        \Function{Foo}{$I \in \mathbb{Z}, N \in \mathbb{Z}, c \in \mathbb{Z}$}
            \State $i \gets I$
            \While{$i~\text{cmp}~N$}
                \State \Call{DoSomething}{}
                \State $i \gets i + c$
            \EndWhile
        \EndFunction
    \end{algorithmic}
    \caption{A general form of loop starting at $I$ and counting in increments of $c$ up to $N$}
    \label{fig:impl:general-loop}
\end{figure}

\newpage

\section{Determining unrollability}\label{sec:impl:unrollability}

Given that the primary goal of any optimization is to conserve semantics, most optimizations are based upon assumptions.
These assumptions will be assured, by checking corresponding preconditions before the optimization is applied, to ensure that its transformed product will be semantically equivalent.

In the case of loop unrolling, in~\Cref{fig:impl:general-loop} we laid out the structured of the targeted loops.
This section formalizes these requirements and extends them, such that the further unrolling process conserves semantics.

Firstly, in view of the fact that we use the existing loop unrolling functionality as a sub-step (see~\Cref{sec:impl:unroll}), it needs to be ensured that the respective~\libFIRM-graph is in LCSSA form.
We accomplish this by using the existing mechanics~\cite{aebi18bachelorarbeit}.
While it is a preliminary step, assuring LCSSA form can never be a hindrance to unrolling, since it is possible to convert any given graph into LCSSA.
Due to the restrictions of the existing loop unrolling mechanism, a loop must also be the innermost loop, meaning it does not have any nested loops inside of it.
Nested loops inherently cause larger code sizes and hardly saves jumps, since most jumps will occur in the inner loops.
Therefore, the restriction will in practice most likely not harm performance.
\Cref{sec:impl:sel-factor} describes the mechanics for determining if and how a given loop should be unrolled based on size.

Moreover, in order for loops to be in the form described in \Cref{fig:impl:general-loop}, loops have to have a header, which itself controls the control flow by comparing a counter to a bound, using any of the four allowed comparison types.
The header is the only point in the loop from which the loop can exit; meaning there are no conditionals in the body that allow the control flow to leave the loop.
This primarily requires there not to be any \texttt{break}-like structures.

Seeing that there is an explicit entry point for the loop, the header, there are no preconditions for $I$, since it is therefore only evaluate once in a block dominating the header, but inherently not determining of the control flow after the initial evaluation.
On the contrary, $N$, the bound, has to be loop-invariant, which means that it may not change through the entire evaluation of the loop, because it is checked against $i$ in every iteration.
As an example, consider a loop, such as the one in \Cref{fig:impl:general-loop}, replacing \texttt{DoSomething} with \texttt{$N =$ randomNumber}.
If we now execute the body $f > 1$ times consecutively, we will effectively lose $f - 1$ checks.
Assume that initially $I = 0, c = 1, f = 2, N = 2$, and assume in the first execution in the loop body $N$ is set to $0$ by chance, whereas in the second iteration it is set to $7$.
Now given that when unrolling the condition is removed for the entering the second body, the loop body would at least four times, which does not conserve semantics, as it should only be executed once.
Concluding, only if $N$ is loop-invariant, the bound checks can be performed less often, while keeping the original semantics intact.

If $N$ is constant it is obviously loop-invariant, but what if it is the result of a function call or of a load from memory?
For the case that $N$ is function call, this function call must be pure (i.e., not have any side-effects), and only have loop-invariant arguments, seeing that the call is then by definition loop-invariant itself.

In case that $N$ is being loaded from memory, stricter conditions have to be met.
All stores within the loop must be sure not to alias the memory location of $N$.
Further, any calls must either invoke functions known at compile time and none of these may contain aliasing stores or have aliasing parameters.
Otherwise, the loop cannot be unrolled with a loaded bound, due to these called functions potentially modifying $N$.

Lastly, the unroll-factor -- meaning how often the loop body is copied inside the unrolled loop -- $f$ is selected (see \Cref{sec:impl:sel-factor} and hence known at compile time.
We can therefore restrict the increment $c$, such that $t_{min} \leq c \cdot (f+1) \leq t_{max}$, where $t_{min}$ is the minimum value of the integer type of $c$ and $t_{max}$ the respective maximum.
Hence, we prevent $c \cdot (f + 1)$ from overflowing, which will turn out to be important in \Cref{sec:impl:fixup:header-cond} and \Cref{sec:impl:fixup:duff}, and further discussed there.
In order to assure this property, we have to force $c$ to be a compile-time constant (which inherently is loop invariant).
Even though the restrictions on $c$ seem comparatively tight, in real-world code (gcc, \texttt{spec2006}) only approximately $1.2\%$ of loops that meet the previous conditions are not unrollable because of the restriction that  $t_{min} \leq c \cdot (f+1) \leq t_{max}$.

It is worth mentioning that the unrollability with the method above is only checked if the current loop unrolling mechanism~\cite{aebi18bachelorarbeit} determines that the current unrolling process cannot be applied.
We chose this design, for the reason that statically unrolling without any further fixup code inherently simplifies the control flow and hence should yield better, or (at least) equal, performance.

\section{Unrolling}\label{sec:impl:unroll}

To get started with unrolling loops that have unknown bounds, we unroll them by a given factor without considering whether the transformation is semantically invariant.
Semantic equivalence, which is broken due to the failure to consider how the factor relates to the original amount of iterations, will be restored in \Cref{sec:impl:fixup}.

\libFIRM{} already provides an unrolling mechanism for unrolling a loop with a given factor $f$~\cite{aebi18bachelorarbeit}.\footnote{N.B.: All following operations preserve the LCSSA property of the code.}
In order to avoid code duplication, we will use be utilizing this solution.

Further, figures~\ref{fig:impl:unroll:unroll-factor-2-before}~and~\ref{fig:impl:unroll:unroll-factor-2-after} show a firm graph of a loop that is to be unrolled or is unrolled using a factor of two, respectively.
Especially to be noted is that in \Cref{fig:impl:unroll:unroll-factor-2-after} we duplicate the loop header, and that hence the number of conditional jumps did not decrease through the loop unroll.
With the previous usage, this was not an issue, because~\libFIRM{} would automatically remove these excess headers~\cite{aebi18bachelorarbeit} using its constant bit analysis.
Unfortunately though in the use cases of an unknown bound, the constant bit analysis does not suffice.
This is due to the fact that the additional semantics, meaning that we are sure not to have to exit the unrolled loop from its body at any time, that are implicitly affixed to the transformed loop, cannot be recognized by~\libFIRM.
Therefore, the need to manually prune the graph to remove the excess headers arises.
\Cref{alg:impl:unroll:prune-headers} shows the algorithm used to accomplish this.
First, we rewire all $\Phi$-nodes in the excess header, such that all in-loop nodes depending on any given $\Phi$-node each get the in-loop predecessors of the $\Phi$-node as predecessors themselves, while the $\Phi$-node falls into desuetude.
We apply the same transformations to the descendants of the block itself.

\begin{algorithm}
    \begin{algorithmic}
        \Function{PruneExcessHeader}{$copiedHeader: \text{Block}$}
            \ForAll{$phi \in copiedHeade$}
                \State \Call{PrunePhi}{$phi, copiedHeader$}
            \EndFor
            \ForAll{$post \in h.descendants$}
                \State $post.predecessors \gets (post.predecessors \backslash \{copiedHeader\}) \cup \newline \{b \vert  b \in copiedHeader.predecessors, b.loop = copiedHeader.loop\}$
            \EndFor
        \EndFunction
        \Function{PrunePhi}{$phi: \text{Phi-Node}, copiedHeader: \text{Block}$}
            \ForAll{$out \in phi.descendants$}
                \Comment $out$ is ensured to be $\phi$ node by the LCSSA construction algorithm~\cite{aebi18bachelorarbeit}
                \If{$out.block \neq copiedHeader$}
                    \State $out.predecessors \gets (out.predecessors \backslash \{phi\}) \cup \newline \{n \vert  n \in phi.predecessors, n.loop = out.loop\}$
                \EndIf
            \EndFor
        \EndFunction
    \end{algorithmic}
    \caption{Pruning excess headers after unrolling}
    \label{alg:impl:unroll:prune-headers}
\end{algorithm}

\begin{algorithm}[h]
    \begin{algorithmic}
        \Function{UnrollExisting}{$factor: \mathbb{N}_{> 1}, loop: \text{Loop}$}
            \State \Call{AssureLCSSA}{$loop$}
            \ForAll{$block \in loop$}
                \For{$i \in \{1..(factor - 1)\}$}
                    \State \Call{DuplicateBlock}{$block$}
                \EndFor
            \EndFor
            \State \Call{RewireDuplicatedBlocks}{} \Comment{Attach blocks to form unrolled structure}
            \State \Comment{$loop$ is still in LCSSA form after unrolling}
        \EndFunction
    \end{algorithmic}
    \caption{Pseudo code for the existing unrolling mechanism~\cite{aebi18bachelorarbeit}}
    \label{fig:impl:unroll:existing-mechanism}
\end{algorithm}

\begin{figure}[H]
    \centering
    \begin{adjustbox}{max width=\textwidth}
        \centering
        % Scale factor 0.024395857307249712
\definecolor{color0}{RGB}{222,239,234}
\definecolor{color1}{RGB}{192,192,192}
\definecolor{color2}{RGB}{153,153,255}
\definecolor{color3}{RGB}{255,153,153}
\definecolor{color4}{RGB}{255,255,255}
\definecolor{color5}{RGB}{255,255,153}
\definecolor{color6}{RGB}{153,255,153}
\definecolor{color7}{RGB}{0,150,60}
\definecolor{color8}{RGB}{170,30,30}
\definecolor{color9}{RGB}{255,0,0}
\definecolor{color10}{RGB}{100,100,255}
\definecolor{color11}{RGB}{0,0,0}
\definecolor{color12}{RGB}{128,0,128}
% Bounding Box: 968.0, 869.0
\begin{tikzpicture}
	\node[fill=color0, draw, minimum width=17.40644418872267cm, minimum height=8.611737629459148cm] (n1) at (14.895584286649068cm ,-2.427387802071346cm) {};
	% 1 node layouts
	\node[scale=0.8871220838999895, transform shape] at (14.895584286649068cm ,1.599834579976985cm) {Block  227};
	\node[fill=color0, draw, minimum width=9.73825221688215cm, minimum height=8.855696202531645cm] (n2) at (5.601001827658567cm ,7.282163406214039cm) {};
	% 1 node layouts
	\node[scale=0.8871220838999895, transform shape] at (5.601001827658567cm ,11.431365074798618cm) {Block  217};
	\node[fill=color1, draw, minimum width=23.5988063810104cm, minimum height=21.2cm] (n3) at (12.165341050113947cm ,3.5008055235903335cm) {};
	% 1 node layouts
	\node[scale=0.8871220838999895, transform shape] at (12.165341050113947cm ,13.822159090909091cm) {\#LOOP-6};
	\node[fill=color2, draw, minimum width=2.342002301495972cm, minimum height=0.7318757192174914cm] (n4) at (2.312583767684289cm ,8.709321058688147cm) {};
	% 1 node layouts
	\node[scale=0.8871220838999895, transform shape] at (2.312583767684289cm ,8.709321058688147cm) {Phi[loop]  218};
	\node[fill=color3, draw, minimum width=2.634752589182969cm, minimum height=0.7318757192174914cm] (n5) at (5.761727475800447cm ,3.5861910241657076cm) {};
	% 1 node layouts
	\node[scale=0.8871220838999895, transform shape] at (5.761727475800447cm ,3.5861910241657076cm) {Proj X false 226};
	\node[fill=color3, draw, minimum width=2.53716915995397cm, minimum height=0.7318757192174914cm] (n6) at (8.83560549651391cm ,3.5861910241657076cm) {};
	% 1 node layouts
	\node[scale=0.8871220838999895, transform shape] at (8.83560549651391cm ,3.5861910241657076cm) {Proj X true 225};
	\node[fill=color3, draw, minimum width=1.854085155350978cm, minimum height=0.7318757192174914cm] (n7) at (7.264942801055981cm ,5.293901035673187cm) {};
	% 1 node layouts
	\node[scale=0.8871220838999895, transform shape] at (7.264942801055981cm ,5.293901035673187cm) {Cond  224};
	\node[fill=color4, draw, minimum width=3.5373993095512084cm, minimum height=0.7318757192174914cm] (n8) at (7.264942801055981cm ,7.001611047180667cm) {};
	% 1 node layouts
	\node[scale=0.8871220838999895, transform shape] at (7.264942801055981cm ,7.001611047180667cm) {Cmp b less\_equal 223};
	\node[fill=color5, draw, minimum width=2.9762945914844647cm, minimum height=0.7318757192174914cm] (n9) at (8.149292628443783cm ,8.709321058688147cm) {};
	% 1 node layouts
	\node[scale=0.8871220838999895, transform shape] at (8.149292628443783cm ,8.709321058688147cm) {Const 0x10 Is 222};
	\node[fill=color6, draw, minimum width=1.7565017261219793cm, minimum height=0.7318757192174914cm] (n10) at (5.20349285859338cm ,8.709321058688147cm) {};
	% 1 node layouts
	\node[scale=0.8871220838999895, transform shape] at (5.20349285859338cm ,8.709321058688147cm) {Phi Is 219};
	\node[fill=color5, draw, minimum width=2.781127733026467cm, minimum height=0.7318757192174914cm] (n11) at (6.2984363365599405cm ,10.417031070195627cm) {};
	% 1 node layouts
	\node[scale=0.8871220838999895, transform shape] at (6.2984363365599405cm ,10.417031070195627cm) {Const 0x0 Is 215};
	\node[fill=color7, draw, minimum width=0.48791714614499426cm, minimum height=0.48791714614499426cm] (n12) at (1.3417721518987342cm ,7.001611047180667cm) {};
	\node[fill=color7, draw, minimum width=0.48791714614499426cm, minimum height=0.48791714614499426cm] (n13) at (4.7643674270628855cm ,7.001611047180667cm) {};
	\node[fill=color7, draw, minimum width=0.48791714614499426cm, minimum height=0.48791714614499426cm] (n14) at (2.8005009138292833cm ,10.417031070195627cm) {};
	\node[fill=color7, draw, minimum width=0.48791714614499426cm, minimum height=0.48791714614499426cm] (n15) at (1.824666621539295cm ,10.417031070195627cm) {};
	\node[fill=color7, draw, minimum width=0.48791714614499426cm, minimum height=0.48791714614499426cm] (n16) at (4.175996750829215cm ,10.417031070195627cm) {};
	\node[fill=color7, draw, minimum width=0.48791714614499426cm, minimum height=0.48791714614499426cm] (n17) at (2.3176064441887227cm ,7.001611047180667cm) {};
	\node[fill=color7, draw, minimum width=0.48791714614499426cm, minimum height=0.48791714614499426cm] (n18) at (3.297745887768226cm ,7.001611047180667cm) {};
	\node[fill=color2, draw, minimum width=2.268814729574223cm, minimum height=0.7318757192174914cm] (n19) at (13.047598095624902cm ,-4.537629459148446cm) {};
	% 1 node layouts
	\node[scale=0.8871220838999895, transform shape] at (13.047598095624902cm ,-4.537629459148446cm) {Proj M M 235};
	\node[fill=color2, draw, minimum width=1.6345224395857307cm, minimum height=0.7318757192174914cm] (n20) at (13.047598095624902cm ,-2.8299194476409664cm) {};
	% 1 node layouts
	\node[scale=0.8871220838999895, transform shape] at (13.047598095624902cm ,-2.8299194476409664cm) {Call  234};
	\node[fill=color5, draw, minimum width=3.659378596087457cm, minimum height=0.7318757192174914cm] (n21) at (9.961522146257813cm ,-1.1222094361334867cm) {};
	% 1 node layouts
	\node[scale=0.8871220838999895, transform shape] at (9.961522146257813cm ,-1.1222094361334867cm) {Address \&\_printf P 229};
	\node[fill=color5, draw, minimum width=3.41542002301496cm, minimum height=0.7318757192174914cm] (n22) at (16.011694758455743cm ,-1.1222094361334867cm) {};
	% 1 node layouts
	\node[scale=0.8871220838999895, transform shape] at (16.011694758455743cm ,-1.1222094361334867cm) {Address \&str.0 P 233};
	\node[fill=color2, draw, minimum width=1.5369390103567317cm, minimum height=0.7318757192174914cm] (n23) at (13.047598095624902cm ,-1.1222094361334867cm) {};
	% 1 node layouts
	\node[scale=0.8871220838999895, transform shape] at (13.047598095624902cm ,-1.1222094361334867cm) {Phi  403};
	\node[fill=color3, draw, minimum width=1.6833141542002301cm, minimum height=0.7318757192174914cm] (n24) at (7.3999571289965935cm ,0.585500575373993cm) {};
	% 1 node layouts
	\node[scale=0.8871220838999895, transform shape] at (7.3999571289965935cm ,0.585500575373993cm) {Jmp  240};
	\node[fill=color4, draw, minimum width=1.9272727272727272cm, minimum height=0.7318757192174914cm] (n25) at (20.477007926300555cm ,-2.8299194476409664cm) {};
	% 1 node layouts
	\node[scale=0.8871220838999895, transform shape] at (20.477007926300555cm ,-2.8299194476409664cm) {Add Is 239};
	\node[fill=color5, draw, minimum width=2.781127733026467cm, minimum height=0.7318757192174914cm] (n26) at (21.842304654888423cm ,-1.1222094361334867cm) {};
	% 1 node layouts
	\node[scale=0.8871220838999895, transform shape] at (21.842304654888423cm ,-1.1222094361334867cm) {Const 0x1 Is 238};
	\node[fill=color6, draw, minimum width=1.7565017261219793cm, minimum height=0.7318757192174914cm] (n27) at (19.085572779169205cm ,-1.1222094361334867cm) {};
	% 1 node layouts
	\node[scale=0.8871220838999895, transform shape] at (19.085572779169205cm ,-1.1222094361334867cm) {Phi Is 402};
	\node[fill=color7, draw, minimum width=0.48791714614499426cm, minimum height=0.48791714614499426cm] (n28) at (13.047598095624902cm ,0.585500575373993cm) {};
	\node[fill=color7, draw, minimum width=0.48791714614499426cm, minimum height=0.48791714614499426cm] (n29) at (19.085572779169205cm ,0.585500575373993cm) {};
	\node[fill=color7, draw, minimum width=0.48791714614499426cm, minimum height=0.48791714614499426cm] (n30) at (13.047598095624902cm ,-6.123360184119678cm) {};
	\node[fill=color7, draw, minimum width=0.48791714614499426cm, minimum height=0.48791714614499426cm] (n31) at (20.477007926300555cm ,-4.537629459148446cm) {};
	\node[fill=color8, draw, minimum width=0.48791714614499426cm, minimum height=0.48791714614499426cm] (n32) at (5.1228908594508cm ,12.929804372842348cm) {};
	\node[fill=color8, draw, minimum width=0.48791714614499426cm, minimum height=0.48791714614499426cm] (n33) at (7.3999571289965935cm ,-1.1222094361334867cm) {};
	\draw[color=color9, -latex] (14.895584286649068cm ,1.8784810126582279cm) -- (14.895584286649068cm ,2.366398158803222cm) -- (8.83560549651391cm ,2.366398158803222cm) -- (8.83560549651391cm ,3.220253164556962cm);
	\node[] at (15.090751145107065cm ,2.2337456846950516cm) {
		\scalebox{0.8871220838999895}{0}
	};
	\draw[color=color10, -latex] (2.8980843430582826cm ,9.075258918296893cm) -- (2.8980843430582826cm ,9.563176064441887cm) -- (2.8005009138292833cm ,9.563176064441887cm) -- (2.8005009138292833cm ,10.17307249712313cm);
	\node[] at (3.09325120151628cm ,9.430523590333717cm) {
		\scalebox{0.8871220838999895}{0}
	};
	\draw[color=color10, -latex] (1.7270831923102963cm ,9.075258918296893cm) -- (1.7270831923102963cm ,9.563176064441887cm) -- (1.824666621539295cm ,9.563176064441887cm) -- (1.824666621539295cm ,10.17307249712313cm);
	\node[] at (1.9222500507682938cm ,9.430523590333717cm) {
		\scalebox{0.8871220838999895}{1}
	};
	\draw[color=color9, -latex] (5.761727475800447cm ,3.9521288837744533cm) -- (5.761727475800447cm ,4.440046029919448cm) -- (6.801421512218236cm ,4.440046029919448cm) -- (6.801421512218236cm ,4.927963176064441cm);
	\node[] at (5.956894334258444cm ,4.307393555811277cm) {
		\scalebox{0.8871220838999895}{0}
	};
	\draw[color=color9, -latex] (8.83560549651391cm ,3.9521288837744533cm) -- (8.83560549651391cm ,4.440046029919448cm) -- (7.728464089893725cm ,4.440046029919448cm) -- (7.728464089893725cm ,4.927963176064441cm);
	\node[] at (9.030772354971909cm ,4.307393555811277cm) {
		\scalebox{0.8871220838999895}{0}
	};
	\draw[color=color11, -latex] (7.264942801055981cm ,5.659838895281933cm) -- (7.264942801055981cm ,6.635673187571921cm);
	\node[] at (7.460109659513979cm ,6.128029703682393cm) {
		\scalebox{0.8871220838999895}{0}
	};
	\draw[color=color11, -latex] (6.380592973668179cm ,7.367548906789413cm) -- (6.380592973668179cm ,7.855466052934407cm) -- (5.642618290123875cm ,7.855466052934407cm) -- (5.642618290123875cm ,8.343383199079401cm);
	\node[] at (6.5757598321261765cm ,7.722813578826237cm) {
		\scalebox{0.8871220838999895}{0}
	};
	\draw[color=color11, -latex] (8.149292628443783cm ,7.367548906789413cm) -- (8.149292628443783cm ,8.343383199079401cm);
	\node[] at (8.34445948690178cm ,7.835739715189873cm) {
		\scalebox{0.8871220838999895}{1}
	};
	\draw[color=color11, -latex] (5.642618290123875cm ,9.075258918296893cm) -- (5.642618290123875cm ,9.563176064441887cm) -- (6.2984363365599405cm ,9.563176064441887cm) -- (6.2984363365599405cm ,10.051093210586881cm);
	\node[] at (5.837785148581872cm ,9.430523590333717cm) {
		\scalebox{0.8871220838999895}{0}
	};
	\draw[color=color11, -latex] (4.7643674270628855cm ,9.075258918296893cm) -- (4.7643674270628855cm ,9.563176064441887cm) -- (4.175996750829215cm ,9.563176064441887cm) -- (4.175996750829215cm ,10.17307249712313cm);
	\node[] at (4.959534285520883cm ,9.430523590333717cm) {
		\scalebox{0.8871220838999895}{1}
	};
	\draw[color=color10, -latex] (1.3417721518987342cm ,7.245569620253164cm) -- (1.3417721518987342cm ,7.855466052934407cm) -- (1.5319163338522985cm ,7.855466052934407cm) -- (1.5319163338522985cm ,8.343383199079401cm);
	\node[] at (1.5369390103567317cm ,7.661823935558113cm) {
		\scalebox{0.8871220838999895}{0}
	};
	\draw[color=color11, -latex] (4.7643674270628855cm ,7.245569620253164cm) -- (4.7643674270628855cm ,8.343383199079401cm);
	\node[] at (4.959534285520883cm ,7.713760428653624cm) {
		\scalebox{0.8871220838999895}{0}
	};
	\draw[color=color10, -latex] (2.3176064441887227cm ,7.245569620253164cm) -- (2.312583767684289cm ,8.343383199079401cm);
	\node[] at (2.5116432004332228cm ,7.713760428653624cm) {
		\scalebox{0.8871220838999895}{0}
	};
	\draw[color=color12, -latex] (3.297745887768226cm ,7.245569620253164cm) -- (3.297745887768226cm ,7.855466052934407cm) -- (3.09325120151628cm ,7.855466052934407cm) -- (3.09325120151628cm ,8.343383199079401cm);
	\node[] at (3.4929127462262235cm ,7.661823935558113cm) {
		\scalebox{0.8871220838999895}{1}
	};
	\draw[color=color10, -latex] (13.047598095624902cm ,-4.171691599539701cm) -- (13.047598095624902cm ,-3.195857307249712cm);
	\node[] at (13.242764954082899cm ,-3.7035007911392404cm) {
		\scalebox{0.8871220838999895}{0}
	};
	\draw[color=color10, -latex] (13.047598095624902cm ,-2.4639815880322207cm) -- (13.047598095624902cm ,-1.4881472957422324cm);
	\node[] at (13.242764954082899cm ,-1.9957907796317607cm) {
		\scalebox{0.8871220838999895}{0}
	};
	\draw[color=color11, -latex] (12.502757282429657cm ,-2.4639815880322207cm) -- (12.502757282429657cm ,-1.9760644418872266cm) -- (9.961522146257813cm ,-1.9760644418872266cm) -- (9.961522146257813cm ,-1.4881472957422324cm);
	\node[] at (12.697924140887656cm ,-2.108716915995397cm) {
		\scalebox{0.8871220838999895}{1}
	};
	\draw[color=color11, -latex] (13.592438908820146cm ,-2.4639815880322207cm) -- (13.592438908820144cm ,-1.9760644418872266cm) -- (16.011694758455743cm ,-1.9760644418872266cm) -- (16.011694758455743cm ,-1.4881472957422324cm);
	\node[] at (13.787605767278144cm ,-2.108716915995397cm) {
		\scalebox{0.8871220838999895}{2}
	};
	\draw[color=color10, -latex] (13.047598095624902cm ,-0.7562715765247411cm) -- (13.047598095624902cm ,0.34154200230149595cm);
	\node[] at (13.242764954082899cm ,-0.2880807681242808cm) {
		\scalebox{0.8871220838999895}{0}
	};
	\draw[color=color11, -latex] (19.99518974448237cm ,-2.4639815880322207cm) -- (19.99518974448237cm ,-1.9760644418872266cm) -- (19.085572779169205cm ,-1.9760644418872266cm) -- (19.085572779169205cm ,-1.4881472957422324cm);
	\node[] at (20.19035660294037cm ,-2.108716915995397cm) {
		\scalebox{0.8871220838999895}{0}
	};
	\draw[color=color11, -latex] (20.958826108118735cm ,-2.4639815880322207cm) -- (20.958826108118735cm ,-1.9760644418872266cm) -- (21.842304654888423cm ,-1.9760644418872266cm) -- (21.842304654888423cm ,-1.4881472957422324cm);
	\node[] at (21.153992966576734cm ,-2.108716915995397cm) {
		\scalebox{0.8871220838999895}{1}
	};
	\draw[color=color11, -latex] (19.085572779169205cm ,-0.7562715765247411cm) -- (19.085572779169205cm ,0.34154200230149595cm);
	\node[] at (19.280739637627203cm ,-0.2880807681242808cm) {
		\scalebox{0.8871220838999895}{0}
	};
	\draw[color=color10, -latex] (13.047598095624902cm ,-5.8794016110471805cm) -- (13.047598095624902cm ,-4.903567318757192cm);
	\node[] at (13.242764954082899cm ,-5.41121080264672cm) {
		\scalebox{0.8871220838999895}{1}
	};
	\draw[color=color11, -latex] (20.477007926300555cm ,-4.293670886075949cm) -- (20.477007926300555cm ,-3.195857307249712cm);
	\node[] at (20.67217478475855cm ,-3.825480077675489cm) {
		\scalebox{0.8871220838999895}{1}
	};
	\draw[color=color9, -latex] (8.035564881879104cm ,11.710011507479862cm) -- (8.035564881879104cm ,12.197928653624857cm) -- (5.1228908594508cm ,12.197928653624857cm) -- (5.1228908594508cm ,12.68584579976985cm);
	\node[] at (8.2307317403371cm ,12.065276179516685cm) {
		\scalebox{0.8871220838999895}{1}
	};
	\draw[color=color9, -latex] (7.3999571289965935cm ,-0.8782508630609897cm) -- (7.3999571289965935cm ,0.2195627157652474cm);
	\node[] at (7.595123987454591cm ,-0.41006005466052936cm) {
		\scalebox{0.8871220838999895}{1}
	};
\end{tikzpicture}

    \end{adjustbox}
    \caption{Firm graph of a loop with an unknown bound}
    \label{fig:impl:unroll:unroll-factor-2-before}
\end{figure}
\begin{figure}[h]
    \centering
    \begin{adjustbox}{max width=\textwidth}
        \centering
        % Scale factor 0.012809667673716012
\definecolor{color13}{RGB}{222,239,234}
\definecolor{color14}{RGB}{192,192,192}
\definecolor{color15}{RGB}{153,153,255}
\definecolor{color16}{RGB}{255,255,153}
\definecolor{color17}{RGB}{255,153,153}
\definecolor{color18}{RGB}{255,255,255}
\definecolor{color19}{RGB}{153,255,153}
\definecolor{color20}{RGB}{0,150,60}
\definecolor{color21}{RGB}{170,30,30}
\definecolor{color22}{RGB}{255,0,0}
\definecolor{color23}{RGB}{100,100,255}
\definecolor{color24}{RGB}{0,0,0}
\definecolor{color25}{RGB}{128,0,128}
% Bounding Box: 1325.0, 1655.0
\begin{tikzpicture}
	\node[fill=color13, draw, minimum width=9.139697885196375cm, minimum height=4.521812688821752cm] (n34) at (12.39926257798397cm ,-1.2745619335347431cm) {};
	% 1 node layouts
	\node[scale=0.4658060972260368, transform shape] at (12.39926257798397cm ,0.8400339879154078cm) {Block  413};
	\node[fill=color13, draw, minimum width=5.539051004087435cm, minimum height=4.649909365558912cm] (n35) at (3.153815532255198cm ,13.892084592145014cm) {};
	% 1 node layouts
	\node[scale=0.4658060972260368, transform shape] at (3.153815532255198cm ,16.070728851963747cm) {Block  217};
	\node[fill=color13, draw, minimum width=8.326283987915408cm, minimum height=4.521812688821752cm] (n36) at (8.09521423961539cm ,8.793836858006042cm) {};
	% 1 node layouts
	\node[scale=0.4658060972260368, transform shape] at (8.09521423961539cm ,10.908432779456193cm) {Block  227};
	\node[fill=color13, draw, minimum width=4.925317220543807cm, minimum height=4.521812688821752cm] (n37) at (6.897510312122942cm ,3.7596374622356494cm) {};
	% 1 node layouts
	\node[scale=0.4658060972260368, transform shape] at (6.897510312122942cm ,5.874233383685801cm) {Block  406};
	% 1 node layouts
	\node[fill=color15, draw, minimum width=1.1912990936555892cm, minimum height=0.38429003021148034cm] (n39) at (9.613159858950738cm ,7.685800604229607cm) {};
	% 1 node layouts
	\node[scale=0.4658060972260368, transform shape] at (9.613159858950738cm ,7.685800604229607cm) {Proj M M 235};
	\node[fill=color15, draw, minimum width=0.8582477341389728cm, minimum height=0.38429003021148034cm] (n40) at (9.613159858950738cm ,8.582477341389728cm) {};
	% 1 node layouts
	\node[scale=0.4658060972260368, transform shape] at (9.613159858950738cm ,8.582477341389728cm) {Call  234};
	\node[fill=color16, draw, minimum width=1.9214501510574018cm, minimum height=0.38429003021148034cm] (n41) at (7.992736898225662cm ,9.479154078549849cm) {};
	% 1 node layouts
	\node[scale=0.4658060972260368, transform shape] at (7.992736898225662cm ,9.479154078549849cm) {Address \&\_printf P 229};
	\node[fill=color16, draw, minimum width=1.7933534743202417cm, minimum height=0.38429003021148034cm] (n42) at (11.169534481307233cm ,9.479154078549849cm) {};
	% 1 node layouts
	\node[scale=0.4658060972260368, transform shape] at (11.169534481307233cm ,9.479154078549849cm) {Address \&str.0 P 233};
	\node[fill=color15, draw, minimum width=0.8070090634441087cm, minimum height=0.38429003021148034cm] (n43) at (9.613159858950738cm ,9.479154078549849cm) {};
	% 1 node layouts
	\node[scale=0.4658060972260368, transform shape] at (9.613159858950738cm ,9.479154078549849cm) {Phi  403};
	\node[fill=color17, draw, minimum width=0.8838670694864048cm, minimum height=0.38429003021148034cm] (n44) at (6.897510312122942cm ,10.37583081570997cm) {};
	% 1 node layouts
	\node[scale=0.4658060972260368, transform shape] at (6.897510312122942cm ,10.37583081570997cm) {Jmp  240};
	\node[fill=color18, draw, minimum width=1.0119637462235649cm, minimum height=0.38429003021148034cm] (n45) at (5.314448882112871cm ,8.582477341389728cm) {};
	% 1 node layouts
	\node[scale=0.4658060972260368, transform shape] at (5.314448882112871cm ,8.582477341389728cm) {Add Is 239};
	\node[fill=color16, draw, minimum width=1.4603021148036255cm, minimum height=0.38429003021148034cm] (n46) at (6.032857744147112cm ,9.479154078549849cm) {};
	% 1 node layouts
	\node[scale=0.4658060972260368, transform shape] at (6.032857744147112cm ,9.479154078549849cm) {Const 0x1 Is 238};
	\node[fill=color19, draw, minimum width=0.9222960725075529cm, minimum height=0.38429003021148034cm] (n47) at (4.585365297017201cm ,9.479154078549849cm) {};
	% 1 node layouts
	\node[scale=0.4658060972260368, transform shape] at (4.585365297017201cm ,9.479154078549849cm) {Phi Is 402};
	\node[fill=color20, draw, minimum width=0.25619335347432026cm, minimum height=0.25619335347432026cm] (n48) at (9.613159858950738cm ,10.37583081570997cm) {};
	\node[fill=color20, draw, minimum width=0.25619335347432026cm, minimum height=0.25619335347432026cm] (n49) at (4.585365297017202cm ,10.37583081570997cm) {};
	\node[fill=color20, draw, minimum width=0.25619335347432026cm, minimum height=0.25619335347432026cm] (n50) at (9.613159858950738cm ,6.853172205438066cm) {};
	\node[fill=color20, draw, minimum width=0.25619335347432026cm, minimum height=0.25619335347432026cm] (n51) at (5.314448882112871cm ,7.685800604229607cm) {};
	\node[fill=color15, draw, minimum width=1.1912990936555892cm, minimum height=0.38429003021148034cm] (n52) at (11.428930251699981cm ,-2.3825981873111783cm) {};
	% 1 node layouts
	\node[scale=0.4658060972260368, transform shape] at (11.428930251699981cm ,-2.3825981873111783cm) {Proj M M 417};
	\node[fill=color15, draw, minimum width=0.8582477341389728cm, minimum height=0.38429003021148034cm] (n53) at (11.428930251699981cm ,-1.4859214501510574cm) {};
	% 1 node layouts
	\node[scale=0.4658060972260368, transform shape] at (11.428930251699981cm ,-1.4859214501510574cm) {Call  418};
	\node[fill=color16, draw, minimum width=1.9214501510574018cm, minimum height=0.38429003021148034cm] (n54) at (9.808507290974907cm ,-0.5892447129909365cm) {};
	% 1 node layouts
	\node[scale=0.4658060972260368, transform shape] at (9.808507290974907cm ,-0.5892447129909365cm) {Address \&\_printf P 229};
	\node[fill=color16, draw, minimum width=1.7933534743202417cm, minimum height=0.38429003021148034cm] (n55) at (12.985304874056476cm ,-0.5892447129909365cm) {};
	% 1 node layouts
	\node[scale=0.4658060972260368, transform shape] at (12.985304874056476cm ,-0.5892447129909365cm) {Address \&str.0 P 233};
	\node[fill=color15, draw, minimum width=0.8070090634441087cm, minimum height=0.38429003021148034cm] (n56) at (11.428930251699981cm ,-0.5892447129909365cm) {};
	% 1 node layouts
	\node[scale=0.4658060972260368, transform shape] at (11.428930251699981cm ,-0.5892447129909365cm) {Phi  419};
	\node[fill=color17, draw, minimum width=0.8838670694864048cm, minimum height=0.38429003021148034cm] (n57) at (8.463492185234724cm ,0.3074320241691843cm) {};
	% 1 node layouts
	\node[scale=0.4658060972260368, transform shape] at (8.463492185234724cm ,0.3074320241691843cm) {Jmp  416};
	\node[fill=color18, draw, minimum width=1.0119637462235649cm, minimum height=0.38429003021148034cm] (n58) at (15.329931546477708cm ,-1.4859214501510574cm) {};
	% 1 node layouts
	\node[scale=0.4658060972260368, transform shape] at (15.329931546477708cm ,-1.4859214501510574cm) {Add Is 414};
	\node[fill=color16, draw, minimum width=1.4603021148036255cm, minimum height=0.38429003021148034cm] (n59) at (16.0468154480746cm ,-0.5892447129909365cm) {};
	% 1 node layouts
	\node[scale=0.4658060972260368, transform shape] at (16.0468154480746cm ,-0.5892447129909365cm) {Const 0x1 Is 238};
	\node[fill=color19, draw, minimum width=0.9222960725075529cm, minimum height=0.38429003021148034cm] (n60) at (14.599323000944695cm ,-0.5892447129909365cm) {};
	% 1 node layouts
	\node[scale=0.4658060972260368, transform shape] at (14.599323000944695cm ,-0.5892447129909365cm) {Phi Is 415};
	\node[fill=color20, draw, minimum width=0.25619335347432026cm, minimum height=0.25619335347432026cm] (n61) at (11.428930251699981cm ,0.3074320241691843cm) {};
	\node[fill=color20, draw, minimum width=0.25619335347432026cm, minimum height=0.25619335347432026cm] (n62) at (14.599323000944695cm ,0.3074320241691843cm) {};
	\node[fill=color20, draw, minimum width=0.25619335347432026cm, minimum height=0.25619335347432026cm] (n63) at (11.428930251699981cm ,-3.215226586102719cm) {};
	\node[fill=color20, draw, minimum width=0.25619335347432026cm, minimum height=0.25619335347432026cm] (n64) at (15.329931546477708cm ,-2.3825981873111783cm) {};
	\node[fill=color15, draw, minimum width=1.2297280966767372cm, minimum height=0.38429003021148034cm] (n65) at (5.27345794555698cm ,4.573051359516616cm) {};
	% 1 node layouts
	\node[scale=0.4658060972260368, transform shape] at (5.27345794555698cm ,4.573051359516616cm) {Phi[loop]  412};
	\node[fill=color17, draw, minimum width=1.3834441087613294cm, minimum height=0.38429003021148034cm] (n66) at (6.8879030613676555cm ,1.8830211480362538cm) {};
	% 1 node layouts
	\node[scale=0.4658060972260368, transform shape] at (6.8879030613676555cm ,1.8830211480362538cm) {Proj X false 411};
	\node[fill=color17, draw, minimum width=1.3322054380664652cm, minimum height=0.38429003021148034cm] (n67) at (8.501921188255873cm ,1.8830211480362538cm) {};
	% 1 node layouts
	\node[scale=0.4658060972260368, transform shape] at (8.501921188255873cm ,1.8830211480362538cm) {Proj X true 407};
	\node[fill=color17, draw, minimum width=0.973534743202417cm, minimum height=0.38429003021148034cm] (n68) at (7.688080302052448cm ,2.7796978851963745cm) {};
	% 1 node layouts
	\node[scale=0.4658060972260368, transform shape] at (7.688080302052448cm ,2.7796978851963745cm) {Cond  408};
	\node[fill=color18, draw, minimum width=1.8574018126888217cm, minimum height=0.38429003021148034cm] (n69) at (7.688080302052448cm ,3.6763746223564953cm) {};
	% 1 node layouts
	\node[scale=0.4658060972260368, transform shape] at (7.688080302052448cm ,3.6763746223564953cm) {Cmp b less\_equal 409};
	\node[fill=color16, draw, minimum width=1.5627794561933535cm, minimum height=0.38429003021148034cm] (n70) at (8.152430755224653cm ,4.573051359516616cm) {};
	% 1 node layouts
	\node[scale=0.4658060972260368, transform shape] at (8.152430755224653cm ,4.573051359516616cm) {Const 0x10 Is 222};
	\node[fill=color19, draw, minimum width=0.9222960725075529cm, minimum height=0.38429003021148034cm] (n71) at (6.605663383623445cm ,4.573051359516616cm) {};
	% 1 node layouts
	\node[scale=0.4658060972260368, transform shape] at (6.605663383623445cm ,4.573051359516616cm) {Phi Is 410};
	\node[fill=color20, draw, minimum width=0.25619335347432026cm, minimum height=0.25619335347432026cm] (n72) at (4.755093393693938cm ,3.6763746223564953cm) {};
	\node[fill=color20, draw, minimum width=0.25619335347432026cm, minimum height=0.25619335347432026cm] (n73) at (6.375089365496557cm ,3.6763746223564953cm) {};
	\node[fill=color20, draw, minimum width=0.25619335347432026cm, minimum height=0.25619335347432026cm] (n74) at (5.27345794555698cm ,5.405679758308157cm) {};
	\node[fill=color20, draw, minimum width=0.25619335347432026cm, minimum height=0.25619335347432026cm] (n75) at (6.605663383623445cm ,5.405679758308157cm) {};
	\node[fill=color20, draw, minimum width=0.25619335347432026cm, minimum height=0.25619335347432026cm] (n76) at (5.267480100642579cm ,3.6763746223564953cm) {};
	\node[fill=color20, draw, minimum width=0.25619335347432026cm, minimum height=0.25619335347432026cm] (n77) at (5.77986680759122cm ,3.6763746223564953cm) {};
	\node[fill=color15, draw, minimum width=1.2297280966767372cm, minimum height=0.38429003021148034cm] (n78) at (1.247812333392571cm ,14.641450151057402cm) {};
	% 1 node layouts
	\node[scale=0.4658060972260368, transform shape] at (1.247812333392571cm ,14.641450151057402cm) {Phi[loop]  218};
	\node[fill=color17, draw, minimum width=1.3834441087613294cm, minimum height=0.38429003021148034cm] (n79) at (3.4510751732717253cm ,11.95141993957704cm) {};
	% 1 node layouts
	\node[scale=0.4658060972260368, transform shape] at (3.4510751732717253cm ,11.95141993957704cm) {Proj X false 226};
	\node[fill=color17, draw, minimum width=1.3322054380664652cm, minimum height=0.38429003021148034cm] (n80) at (5.065093300159942cm ,11.95141993957704cm) {};
	% 1 node layouts
	\node[scale=0.4658060972260368, transform shape] at (5.065093300159942cm ,11.95141993957704cm) {Proj X true 225};
	\node[fill=color17, draw, minimum width=0.973534743202417cm, minimum height=0.38429003021148034cm] (n81) at (4.227190332326284cm ,12.84809667673716cm) {};
	% 1 node layouts
	\node[scale=0.4658060972260368, transform shape] at (4.227190332326284cm ,12.84809667673716cm) {Cond  224};
	\node[fill=color18, draw, minimum width=1.8574018126888217cm, minimum height=0.38429003021148034cm] (n82) at (4.227190332326284cm ,13.744773413897281cm) {};
	% 1 node layouts
	\node[scale=0.4658060972260368, transform shape] at (4.227190332326284cm ,13.744773413897281cm) {Cmp b less\_equal 223};
	\node[fill=color16, draw, minimum width=1.5627794561933535cm, minimum height=0.38429003021148034cm] (n83) at (4.691540785498489cm ,14.641450151057402cm) {};
	% 1 node layouts
	\node[scale=0.4658060972260368, transform shape] at (4.691540785498489cm ,14.641450151057402cm) {Const 0x10 Is 222};
	\node[fill=color19, draw, minimum width=0.9222960725075529cm, minimum height=0.38429003021148034cm] (n84) at (3.144773413897281cm ,14.641450151057402cm) {};
	% 1 node layouts
	\node[scale=0.4658060972260368, transform shape] at (3.144773413897281cm ,14.641450151057402cm) {Phi Is 219};
	\node[fill=color16, draw, minimum width=1.4603021148036255cm, minimum height=0.38429003021148034cm] (n85) at (2.6184467744801845cm ,15.538126888217523cm) {};
	% 1 node layouts
	\node[scale=0.4658060972260368, transform shape] at (2.6184467744801845cm ,15.538126888217523cm) {Const 0x0 Is 215};
	\node[fill=color20, draw, minimum width=0.25619335347432026cm, minimum height=0.25619335347432026cm] (n86) at (1.5040056868668914cm ,15.538126888217523cm) {};
	\node[fill=color20, draw, minimum width=0.25619335347432026cm, minimum height=0.25619335347432026cm] (n87) at (0.9916189799182509cm ,15.538126888217523cm) {};
	\node[fill=color20, draw, minimum width=0.25619335347432026cm, minimum height=0.25619335347432026cm] (n88) at (3.7328878620934773cm ,15.538126888217523cm) {};
	\node[fill=color20, draw, minimum width=0.25619335347432026cm, minimum height=0.25619335347432026cm] (n89) at (0.7045317220543806cm ,13.744773413897281cm) {};
	\node[fill=color20, draw, minimum width=0.25619335347432026cm, minimum height=0.25619335347432026cm] (n90) at (2.914199395770393cm ,13.744773413897281cm) {};
	\node[fill=color20, draw, minimum width=0.25619335347432026cm, minimum height=0.25619335347432026cm] (n91) at (1.216918429003021cm ,13.744773413897281cm) {};
	\node[fill=color20, draw, minimum width=0.25619335347432026cm, minimum height=0.25619335347432026cm] (n92) at (1.8818908832415138cm ,13.744773413897281cm) {};
	\node[fill=color21, draw, minimum width=0.25619335347432026cm, minimum height=0.25619335347432026cm] (n93) at (3.087199311589797cm ,16.85752265861027cm) {};
	\node[fill=color21, draw, minimum width=0.25619335347432026cm, minimum height=0.25619335347432026cm] (n94) at (8.463492185234724cm ,-0.5892447129909365cm) {};
	\draw[color=color22, -latex] (12.39926257798397cm ,0.9863444108761329cm) -- (12.39926257798397cm ,1.2425377643504532cm) -- (8.501921188255873cm ,1.2425377643504532cm) -- (8.501921188255873cm ,1.6908761329305135cm);
	\node[] at (12.501739919373698cm ,1.1728851963746223cm) {
		\scalebox{0.4658060972260368}{0}
	};
	\draw[color=color22, -latex] (8.09521423961539cm ,11.054743202416919cm) -- (8.095214239615387cm ,11.310936555891239cm) -- (5.065093300159942cm ,11.310936555891239cm) -- (5.065093300159942cm ,11.759274924471299cm);
	\node[] at (8.197691581005117cm ,11.241283987915407cm) {
		\scalebox{0.4658060972260368}{0}
	};
	\draw[color=color22, -latex] (6.897510312122942cm ,6.0205438066465256cm) -- (6.897510312122942cm ,10.183685800604229cm);
	\node[] at (6.999987653512671cm ,6.2663793429003025cm) {
		\scalebox{0.4658060972260368}{0}
	};
	\draw[color=color23, -latex] (9.613159858950738cm ,7.877945619335347cm) -- (9.613159858950738cm ,8.390332326283987cm);
	\node[] at (9.715637200340465cm ,8.123781155589123cm) {
		\scalebox{0.4658060972260368}{0}
	};
	\draw[color=color23, -latex] (9.613159858950738cm ,8.774622356495469cm) -- (9.613159858950738cm ,9.287009063444108cm);
	\node[] at (9.715637200340465cm ,9.020457892749246cm) {
		\scalebox{0.4658060972260368}{0}
	};
	\draw[color=color24, -latex] (9.327077280904412cm ,8.774622356495469cm) -- (9.327077280904412cm ,9.030815709969788cm) -- (7.992736898225662cm ,9.030815709969788cm) -- (7.992736898225662cm ,9.287009063444108cm);
	\node[] at (9.42955462229414cm ,8.961163141993957cm) {
		\scalebox{0.4658060972260368}{1}
	};
	\draw[color=color24, -latex] (9.899242436997062cm ,8.774622356495469cm) -- (9.89924243699706cm ,9.030815709969788cm) -- (11.169534481307233cm ,9.030815709969788cm) -- (11.169534481307233cm ,9.287009063444108cm);
	\node[] at (10.00171977838679cm ,8.961163141993957cm) {
		\scalebox{0.4658060972260368}{2}
	};
	\draw[color=color23, -latex] (9.613159858950738cm ,9.67129909365559cm) -- (9.613159858950738cm ,10.24773413897281cm);
	\node[] at (9.715637200340465cm ,9.917134629909366cm) {
		\scalebox{0.4658060972260368}{0}
	};
	\draw[color=color24, -latex] (5.06145794555698cm ,8.774622356495469cm) -- (5.06145794555698cm ,9.030815709969788cm) -- (4.585365297017201cm ,9.030815709969788cm) -- (4.585365297017201cm ,9.287009063444108cm);
	\node[] at (5.163935286946709cm ,8.961163141993957cm) {
		\scalebox{0.4658060972260368}{0}
	};
	\draw[color=color24, -latex] (5.567439818668762cm ,8.774622356495469cm) -- (5.567439818668762cm ,9.030815709969788cm) -- (6.032857744147112cm ,9.030815709969788cm) -- (6.032857744147112cm ,9.287009063444108cm);
	\node[] at (5.669917160058491cm ,8.961163141993957cm) {
		\scalebox{0.4658060972260368}{1}
	};
	\draw[color=color24, -latex] (4.585365297017201cm ,9.67129909365559cm) -- (4.585365297017202cm ,10.24773413897281cm);
	\node[] at (4.68784263840693cm ,9.917134629909366cm) {
		\scalebox{0.4658060972260368}{0}
	};
	\draw[color=color23, -latex] (9.613159858950738cm ,6.981268882175226cm) -- (9.613159858950738cm ,7.493655589123867cm);
	\node[] at (9.715637200340465cm ,7.227104418429003cm) {
		\scalebox{0.4658060972260368}{0}
	};
	\draw[color=color24, -latex] (5.314448882112871cm ,7.813897280966767cm) -- (5.314448882112871cm ,8.390332326283987cm);
	\node[] at (5.4169262235026cm ,8.059732817220544cm) {
		\scalebox{0.4658060972260368}{0}
	};
	\draw[color=color23, -latex] (11.428930251699981cm ,-2.190453172205438cm) -- (11.428930251699981cm ,-1.6780664652567976cm);
	\node[] at (11.53140759308971cm ,-1.9446176359516616cm) {
		\scalebox{0.4658060972260368}{0}
	};
	\draw[color=color23, -latex] (11.428930251699981cm ,-1.2937764350453171cm) -- (11.428930251699981cm ,-0.7813897280966767cm);
	\node[] at (11.53140759308971cm ,-1.0479408987915408cm) {
		\scalebox{0.4658060972260368}{0}
	};
	\draw[color=color24, -latex] (11.142847673653657cm ,-1.2937764350453171cm) -- (11.142847673653657cm ,-1.037583081570997cm) -- (9.808507290974907cm ,-1.037583081570997cm) -- (9.808507290974907cm ,-0.7813897280966767cm);
	\node[] at (11.245325015043385cm ,-1.1072356495468278cm) {
		\scalebox{0.4658060972260368}{1}
	};
	\draw[color=color24, -latex] (11.715012829746307cm ,-1.2937764350453171cm) -- (11.715012829746305cm ,-1.037583081570997cm) -- (12.985304874056476cm ,-1.037583081570997cm) -- (12.985304874056476cm ,-0.7813897280966767cm);
	\node[] at (11.817490171136035cm ,-1.1072356495468278cm) {
		\scalebox{0.4658060972260368}{2}
	};
	\draw[color=color23, -latex] (11.428930251699981cm ,-0.3970996978851964cm) -- (11.428930251699981cm ,0.17933534743202417cm);
	\node[] at (11.53140759308971cm ,-0.15126416163141995cm) {
		\scalebox{0.4658060972260368}{0}
	};
	\draw[color=color24, -latex] (15.076940609921817cm ,-1.2937764350453171cm) -- (15.07694060992182cm ,-1.037583081570997cm) -- (14.599323000944695cm ,-1.037583081570997cm) -- (14.599323000944695cm ,-0.7813897280966767cm);
	\node[] at (15.179417951311548cm ,-1.1072356495468278cm) {
		\scalebox{0.4658060972260368}{0}
	};
	\draw[color=color24, -latex] (15.582922483033599cm ,-1.2937764350453171cm) -- (15.582922483033599cm ,-1.037583081570997cm) -- (16.0468154480746cm ,-1.037583081570997cm) -- (16.0468154480746cm ,-0.7813897280966767cm);
	\node[] at (15.685399824423326cm ,-1.1072356495468278cm) {
		\scalebox{0.4658060972260368}{1}
	};
	\draw[color=color24, -latex] (14.599323000944695cm ,-0.3970996978851964cm) -- (14.599323000944695cm ,0.17933534743202417cm);
	\node[] at (14.701800342334423cm ,-0.15126416163141995cm) {
		\scalebox{0.4658060972260368}{0}
	};
	\draw[color=color23, -latex] (11.428930251699981cm ,-3.087129909365559cm) -- (11.428930251699981cm ,-2.5747432024169186cm);
	\node[] at (11.53140759308971cm ,-2.8412943731117823cm) {
		\scalebox{0.4658060972260368}{1}
	};
	\draw[color=color24, -latex] (15.329931546477708cm ,-2.254501510574018cm) -- (15.329931546477708cm ,-1.6780664652567976cm);
	\node[] at (15.432408887867435cm ,-2.0086659743202415cm) {
		\scalebox{0.4658060972260368}{1}
	};
	\draw[color=color23, -latex] (5.27345794555698cm ,4.765196374622357cm) -- (5.27345794555698cm ,5.277583081570997cm);
	\node[] at (5.375935286946707cm ,5.011031910876133cm) {
		\scalebox{0.4658060972260368}{0}
	};
	\draw[color=color22, -latex] (6.8879030613676555cm ,2.075166163141994cm) -- (6.8879030613676555cm ,2.331359516616314cm) -- (7.444696616251844cm ,2.331359516616314cm) -- (7.444696616251844cm ,2.5875528700906343cm);
	\node[] at (6.990380402757383cm ,2.2617069486404833cm) {
		\scalebox{0.4658060972260368}{0}
	};
	\draw[color=color22, -latex] (8.501921188255873cm ,2.075166163141994cm) -- (8.501921188255873cm ,2.331359516616314cm) -- (7.931463987853052cm ,2.331359516616314cm) -- (7.931463987853052cm ,2.5875528700906343cm);
	\node[] at (8.6043985296456cm ,2.2617069486404833cm) {
		\scalebox{0.4658060972260368}{0}
	};
	\draw[color=color24, -latex] (7.688080302052448cm ,2.9718429003021147cm) -- (7.688080302052448cm ,3.4842296072507555cm);
	\node[] at (7.790557643442176cm ,3.2176784365558913cm) {
		\scalebox{0.4658060972260368}{0}
	};
	\draw[color=color24, -latex] (7.223729848880242cm ,3.8685196374622355cm) -- (7.223729848880242cm ,4.124712990936556cm) -- (6.836237401750333cm ,4.124712990936556cm) -- (6.836237401750333cm ,4.380906344410876cm);
	\node[] at (7.326207190269971cm ,4.055060422960725cm) {
		\scalebox{0.4658060972260368}{0}
	};
	\draw[color=color24, -latex] (8.152430755224653cm ,3.8685196374622355cm) -- (8.152430755224653cm ,4.380906344410876cm);
	\node[] at (8.25490809661438cm ,4.114355173716012cm) {
		\scalebox{0.4658060972260368}{1}
	};
	\draw[color=color24, -latex] (6.605663383623445cm ,4.765196374622357cm) -- (6.605663383623445cm ,5.277583081570997cm);
	\node[] at (6.708140725013173cm ,5.011031910876133cm) {
		\scalebox{0.4658060972260368}{0}
	};
	\draw[color=color23, -latex] (4.755093393693938cm ,3.8044712990936556cm) -- (4.755093393693938cm ,4.124712990936556cm) -- (4.863548579998067cm ,4.124712990936556cm) -- (4.863548579998067cm ,4.380906344410876cm);
	\node[] at (4.857570735083667cm ,4.023036253776435cm) {
		\scalebox{0.4658060972260368}{0}
	};
	\draw[color=color24, -latex] (6.375089365496557cm ,3.8044712990936556cm) -- (6.375089365496557cm ,4.380906344410876cm);
	\node[] at (6.477566706886285cm ,4.050306835347432cm) {
		\scalebox{0.4658060972260368}{0}
	};
	\draw[color=color23, -latex] (5.267480100642579cm ,3.8044712990936556cm) -- (5.27345794555698cm ,4.380906344410876cm);
	\node[] at (5.373711238048868cm ,4.050306835347432cm) {
		\scalebox{0.4658060972260368}{1}
	};
	\draw[color=color25, -latex] (5.77986680759122cm ,3.8044712990936556cm) -- (5.77986680759122cm ,4.124712990936556cm) -- (5.683367311115892cm ,4.124712990936556cm) -- (5.683367311115892cm ,4.380906344410876cm);
	\node[] at (5.882344148980947cm ,4.023036253776435cm) {
		\scalebox{0.4658060972260368}{3}
	};
	\draw[color=color23, -latex] (1.5552443575617554cm ,14.833595166163143cm) -- (1.5552443575617554cm ,15.089788519637462cm) -- (1.5040056868668914cm ,15.089788519637462cm) -- (1.5040056868668914cm ,15.410030211480363cm);
	\node[] at (1.6577216989514836cm ,15.020135951661631cm) {
		\scalebox{0.4658060972260368}{0}
	};
	\draw[color=color23, -latex] (0.9403803092233868cm ,14.833595166163143cm) -- (0.9403803092233868cm ,15.089788519637462cm) -- (0.9916189799182509cm ,15.089788519637462cm) -- (0.9916189799182509cm ,15.410030211480363cm);
	\node[] at (1.0428576506131149cm ,15.020135951661631cm) {
		\scalebox{0.4658060972260368}{1}
	};
	\draw[color=color22, -latex] (3.4510751732717253cm ,12.143564954682779cm) -- (3.4510751732717253cm ,12.3997583081571cm) -- (3.9838066465256796cm ,12.3997583081571cm) -- (3.9838066465256796cm ,12.65595166163142cm);
	\node[] at (3.5535525146614533cm ,12.330105740181269cm) {
		\scalebox{0.4658060972260368}{0}
	};
	\draw[color=color22, -latex] (5.065093300159942cm ,12.143564954682779cm) -- (5.065093300159942cm ,12.3997583081571cm) -- (4.4705740181268885cm ,12.3997583081571cm) -- (4.4705740181268885cm ,12.65595166163142cm);
	\node[] at (5.167570641549671cm ,12.330105740181269cm) {
		\scalebox{0.4658060972260368}{0}
	};
	\draw[color=color24, -latex] (4.227190332326284cm ,13.040241691842901cm) -- (4.227190332326284cm ,13.55262839879154cm);
	\node[] at (4.3296676737160125cm ,13.286077228096676cm) {
		\scalebox{0.4658060972260368}{0}
	};
	\draw[color=color24, -latex] (3.7628398791540785cm ,13.936918429003022cm) -- (3.7628398791540785cm ,14.193111782477342cm) -- (3.375347432024169cm ,14.193111782477342cm) -- (3.375347432024169cm ,14.449305135951661cm);
	\node[] at (3.8653172205438064cm ,14.12345921450151cm) {
		\scalebox{0.4658060972260368}{0}
	};
	\draw[color=color24, -latex] (4.691540785498489cm ,13.936918429003022cm) -- (4.691540785498489cm ,14.449305135951661cm);
	\node[] at (4.794018126888218cm ,14.182753965256797cm) {
		\scalebox{0.4658060972260368}{1}
	};
	\draw[color=color24, -latex] (2.914199395770393cm ,14.833595166163143cm) -- (2.914199395770393cm ,15.089788519637462cm) -- (2.6184467744801845cm ,15.089788519637462cm) -- (2.6184467744801845cm ,15.345981873111782cm);
	\node[] at (3.016676737160121cm ,15.020135951661631cm) {
		\scalebox{0.4658060972260368}{0}
	};
	\draw[color=color24, -latex] (3.375347432024169cm ,14.833595166163143cm) -- (3.375347432024169cm ,15.089788519637462cm) -- (3.7328878620934773cm ,15.089788519637462cm) -- (3.7328878620934773cm ,15.410030211480363cm);
	\node[] at (3.4778247734138974cm ,15.020135951661631cm) {
		\scalebox{0.4658060972260368}{1}
	};
	\draw[color=color23, -latex] (0.7045317220543806cm ,13.872870090634441cm) -- (0.7045317220543806cm ,14.193111782477342cm) -- (0.8379029678336587cm ,14.193111782477342cm) -- (0.8379029678336587cm ,14.449305135951661cm);
	\node[] at (0.8070090634441087cm ,14.09143504531722cm) {
		\scalebox{0.4658060972260368}{0}
	};
	\draw[color=color24, -latex] (2.914199395770393cm ,13.872870090634441cm) -- (2.914199395770393cm ,14.449305135951661cm);
	\node[] at (3.016676737160121cm ,14.118705626888218cm) {
		\scalebox{0.4658060972260368}{0}
	};
	\draw[color=color23, -latex] (1.216918429003021cm ,13.872870090634441cm) -- (1.216918429003021cm ,14.193111782477342cm) -- (1.247812333392571cm ,14.193111782477342cm) -- (1.247812333392571cm ,14.449305135951661cm);
	\node[] at (1.3193957703927492cm ,14.09143504531722cm) {
		\scalebox{0.4658060972260368}{0}
	};
	\draw[color=color25, -latex] (1.8818908832415138cm ,13.872870090634441cm) -- (1.8818908832415138cm ,14.193111782477342cm) -- (1.6577216989514836cm ,14.193111782477342cm) -- (1.6577216989514836cm ,14.449305135951661cm);
	\node[] at (1.9843682246312417cm ,14.09143504531722cm) {
		\scalebox{0.4658060972260368}{1}
	};
	\draw[color=color22, -latex] (4.538578283277056cm ,16.21703927492447cm) -- (4.538578283277056cm ,16.47323262839879cm) -- (3.087199311589797cm ,16.47323262839879cm) -- (3.087199311589797cm ,16.729425981873113cm);
	\node[] at (4.641055624666785cm ,16.40358006042296cm) {
		\scalebox{0.4658060972260368}{1}
	};
	\draw[color=color22, -latex] (8.463492185234724cm ,-0.4611480362537764cm) -- (8.463492185234724cm ,0.1152870090634441cm);
	\node[] at (8.565969526624453cm ,-0.2153125cm) {
		\scalebox{0.4658060972260368}{1}
	};
\end{tikzpicture}

    \end{adjustbox}
    \caption{Firm graph of the loop shown in \cref{fig:impl:unroll:unroll-factor-2-before} unrolled with a factor of two}
    \label{fig:impl:unroll:unroll-factor-2-after}
\end{figure}

\newpage

\section{Fixup strategies}\label{sec:impl:fixup}

In \Cref{sec:impl:unroll} we discussed the unrolling process.
There, we did not consider the fixup code needed, but instead plainly focused on unrolling the loop.
Firstly, we will now focus on making the loop run less than or equal times compared to the original loop in section~\Cref{sec:impl:fixup:header-cond}.
Less-than or equal is not good enough though, we want our transformed loop to run exactly as often as the original loop.
Therefore, we will create fixup code, as discussed in this section and its subsections.
\Cref{sec:impl:fixup:duff} uses a generalized version of Duff's device to create the required fixup code, whereas in \Cref{sec:impl:fixup:loop} a copy of the original loop will be used.
After that, in \Cref{sec:eval}, we evaluate which approach yields faster binary run-times.

To see the reason why we need fixup code and to understand what is required of it, we formally lay out conditions that need to be met in the equations~(\ref{eqn:impl:fixup:duff:conserve-semantics-identity}) through~(\ref{eqn:impl:fixup:duff:fixup-i-mult}).

Let $M \in \mathbb{N}_0$ be the number of times a loop runs before the transformation; $\Mloop \in \mathbb{N}_0$, $\Mfixup \in \mathbb{N}_0$ the number of times the unrolled body will run in the unrolled loop, or the fixup code respectively, after the transformation.
Further, the unroll-factor will be again denoted by $f \in \mathbb{N}, f > 1$.
Henceforth, we will assume all arithmetic operations to be integer operations for integers in the interval $\lbrack t_{min}, t_{max} \rbrack$.
Please note that unmarked integer division will be assumed to round towards zero: For example $\frac{5}{3} \overset{\text{integer divison}}{=} 1$.
Another convention we will introduce is that any interval will be integral, meaning it will only contain integers.
Additionally $x \mp y$ is henceforth defined as
$\begin{cases}
     x + y &, x < 0\\
     x - y &, x > 0
\end{cases}$

We will now lay out properties that form the basis of further arithmetic considerations.
The primary identity that is to be conserved, to retain the original semantics, is shown below in \Cref{eqn:impl:fixup:duff:conserve-semantics-identity}.
Since we know our original loop ran $M$ times, we know that our transformed loop and the fixup code must in total also run $M$ times.

\begin{equation}\label{eqn:impl:fixup:duff:conserve-semantics-identity}
\begin{aligned}
    M = \Mloop + \Mfixup
\end{aligned}
\end{equation}

In order to use Duff's device, we need to restrict the amount of times the fixup code needs to run.
With the requirements to preserve the semantics in mind, we will maximize $\Mloop$ and minimize $\Mfixup$.

\begin{equation}\label{eqn:impl:fixup:duff:loop-iterations}
\begin{aligned}
    \Mloop &\overset{\text{integer divison}}{=} \frac{M}{f} \cdot f\\
    &\overset{\text{integer divison}}{\in} \medspace \rbrack M-f,M \rbrack
\end{aligned}
\end{equation}
By construction of the unrolled loop, \Cref{eqn:impl:fixup:duff:loop-iterations} is always true, as the unrolled loop tries to run as often as possible, while running less than or equal times to the original loop.
\begin{proof}\label{proof:impl:fixup:duff:loop-iterations}
To prove the conjecture of \Cref{eqn:impl:fixup:duff:loop-iterations}, assume for contradiction
\[\Mloop = M - f - b,  b \in \left \lbrack 0, f \right\lbrack \]
and hence
\[\Mloop \leq M - f \Rightarrow \Mloop \notin \medspace \rbrack M-f,f \rbrack,\]
then by rerunning the unrolled again the body would be executed $f$ times causing $\Mloop = M - b \in \medspace \rbrack M-f,f \rbrack$, which would be a contradiction of the assumption.
We then induct this pattern for $\Mloop = M - nf - b, n \in \mathbb{N}_{+}, b \in \lbrack 0, f \lbrack $.
In these cases, the loop must merely be iterated multiple times.
\end{proof}

As the loop runs as often as possible, the fixup code will always run less times than the unroll factor.

\begin{equation}\label{eqn:impl:fixup:duff:fixup-interval}
\begin{aligned}
    \Mfixup \in \lbrack0, f\lbrack
\end{aligned}
\end{equation}

\begin{proof}\label{proof:impl:fixup:duff:fixup-interval}
Conjecture: $\Mfixup \in \lbrack0, f\lbrack$.\\

Assume for contradiction $\Mfixup = f' > (f - 1)$
\begin{align*}
    \Mloop + \Mfixup &\overset{\ref{eqn:impl:fixup:duff:loop-iterations}}{\geq} M - (f - 1) + f' \\
    &> M - (f - 1) + (f - 1) \\
    &= M \medspace \overset{\ref{eqn:impl:fixup:duff:conserve-semantics-identity}}{\mLightning}
\end{align*}
\end{proof}

For the following mathematical considerations, we need to round up in integer division.
The following lemma describes how this can be accomplished.
\begin{lem}\label{lem:impl:fixup:duff:ceil-mp}
    Given $Y \neq 0: \medspace \ceil{\frac{X}{Y}} = \frac{X + (Y \mp 1)}{Y}$
\end{lem}

\begin{proof}
    To prove \Cref{lem:impl:fixup:duff:ceil-mp}, we will consider cases $X \mod Y = 0$ and $X \mod Y \neq 0$.
    Further, we will assume $Y > 0$, since the proof for $Y < 0$ can be performed analogously.
    Consider the case that $X \mod Y = 0$.
    In this case $\exists n \in \mathbb{N}: n \cdot Y = X$ and $\ceil{\frac{X}{Y}} = \frac{X}{Y} = n~(\star)$.
    \begin{align*}
        \Rightarrow \frac{X + (Y - 1)}{Y} &= \frac{n \cdot Y + (Y - 1)}{Y}\\
        &=\underbrace{\frac{(n + 1) \cdot Y - 1}{Y}}_{< \frac{(n + 1) \cdot Y}{Y}}\\
        &\overset{\text{integer divison}}{=} \frac{n \cdot Y}{Y}\\
        &= n\\
        &\overset{\star}{=} \ceil{\frac{X}{Y}}
    \end{align*}
    Now consider $X \mod Y \neq 0$.
    In this case $\exists n \in \mathbb{N}: n \cdot Y < X < (n + 1) \cdot Y$ and $\ceil{\frac{X}{Y}} \overset{\text{integer division}}{=} n + 1$.
    \begin{alignat*}{2}
        \Rightarrow n \cdot Y + (Y - 1) &< X + (Y - 1) &&< (n + 1) \cdot Y + (Y - 1)\\
        \Rightarrow (n + 1) \cdot Y - 1&< X + (Y - 1) &&< (n + 2) \cdot Y - 1\\
        \overset{\text{integers}}{\Rightarrow}  (n + 1) \cdot Y &\leq X + (Y - 1) &&< (n + 2) \cdot Y\\
        \overset{Y > 0}{\Rightarrow} \frac{(n + 1) \cdot Y}{Y} &\leq \frac{X + (Y - 1)}{Y} &&< \frac{(n + 2) \cdot Y}{Y}\\
        &\Rightarrow \frac{X + (Y - 1)}{Y} &&\overset{\text{integer divison}}= n + 1\\
        & &&= \ceil{\frac{X}{Y}}
    \end{alignat*}
\end{proof}

Using our now derived lemma, we use the loop parameters $I$, $N$ and $c$ to calculate the total number of loop iterations.

\begin{equation}\label{eqn:impl:fixup:duff:total-iteration-based-on-bounds}
\begin{aligned}
    M &= \ceil{\frac{N - I}{c}} \\
    & \overset{\text{integer divison}}{=} \frac{N - I + (c \mp 1)}{c}
\end{aligned}
\end{equation}

With this result we can then calculate $M$ accurately from the structure of the loop, since $M$ is not directly known.
Since $I$ is not the initial value for the fixup's counter, we will need to calculate it based on known parameters.

\begin{equation}\label{eqn:impl:fixup:duff:i-after-loop}
\begin{aligned}
    \ipl = c \cdot \Mloop + I
\end{aligned}
\end{equation}

We can then use the last two equations to calculate $\Mfixup$ based on only quantities that are known at either compile-time or at run-time.

\begin{equation}\label{eqn:impl:fixup:duff:fixup-i}
\begin{aligned}
    \Mfixup & \overset{\ref{eqn:impl:fixup:duff:conserve-semantics-identity}}{=} M - \Mloop \\
    & \overset{\ref{eqn:impl:fixup:duff:total-iteration-based-on-bounds}}{=}
    \frac{N - I + (c \mp 1)}{c} - \Mloop \\
    & \overset{\ref{eqn:impl:fixup:duff:i-after-loop}}{=}
    \frac{N - I + (c \mp 1)}{c} - \frac{\ipl + I}{c} \\
    &= \frac{N - I + I - \ipl + (c \mp 1)}{c}\\
    &= \frac{N - \ipl + (c \mp 1)}{c}
\end{aligned}
\end{equation}

As the above equation has a costly divison operation in it, we will rearrange it, such that it never needs to be computed at runtime.

\begin{equation}\label{eqn:impl:fixup:duff:fixup-i-mult}
\begin{aligned}
    (\ref{eqn:impl:fixup:duff:fixup-i}) &\overset{\text{integer division}}{\Longleftrightarrow} \Mfixup \cdot c = N - \ipl + (c \mp 1) \overset{\ref{eqn:impl:fixup:duff:fixup-interval}}{\in} \cinterval
\end{aligned}
\end{equation}

\Cref{eqn:impl:fixup:duff:fixup-i-mult} is especially significant in the construction of the generalization of Duff's device, as seen in~\Cref{sec:impl:fixup:duff}.

\subsection{Updating the loop condition}\label{sec:impl:fixup:header-cond}

In the following Sections~\ref{sec:impl:fixup:duff}, and~\ref{sec:impl:fixup:loop} we will use that $\Mloop = \frac{M}{f} \cdot f$.
Though, when unrolling (as described in~\Cref{sec:impl:unroll}), the original bound ($N$) is kept.
Unfortunately, this does not guarantee $\Mloop$ to be correct, as made clear by an example where a loop with $I = 0, N = 3, c = 1, f = 2, cmp = <$ is unrolled.
In this example, this would yield the following: $\Mloop = 4 > M = 3 \mLightning$, due to the fact that after the first iteration of the unrolled loop $i = 2 < 3 = N$.
To combat this, we set the bound of the unrolled loop to $\hat{N} = N - c \cdot (f - 1)$.\footnote{N.B.: $c \cdot (f - 1)$ does not overflow, as per preconditions}
Now we will prove the conjecture that using the bound $\hat{N}$, the unrolled loop runs $\Mloop$ times, given the operation to calculate $\hat{N}$ will not over- or underflow.
\begin{proof}
    Let $\Mloop'$ be the number of times then unrolled loop with bound $\hat{N}$ runs.
    The proof is complete, iff $\Mloop' = \Mloop$.
    \begin{align*}
        \Mloop' &\overset{\text{loop construction}}{=} \ceil{\frac{\hat{N} - I}{c \cdot f}} \cdot f\\
        &\overset{\text{integer division}}{=} \frac{\hat{N} - I + (c \cdot f \mp 1)}{c \cdot f} \cdot f\\
        &= \frac{N - c \cdot (f - 1) - I + (c \cdot f \mp 1)}{c \cdot f} \cdot f\\
        &= \frac{N - c \cdot f + c - I + c \cdot f \mp 1}{c \cdot f} \cdot f\\
        &= \frac{N - I + c \mp 1}{c \cdot f} \cdot f\\
        &= \frac{\frac{N - I + c \mp 1}{c}}{f} \cdot f\\
        &\overset{\text{integer division}}{=} \frac{\ceil{\frac{N - I}{c}}}{f} \cdot f\\
        &\overset{\ref{eqn:impl:fixup:duff:total-iteration-based-on-bounds}}{=} \frac{M}{f} \cdot f\\
        &\overset{\ref{eqn:impl:fixup:duff:loop-iterations}}{=} \Mloop
    \end{align*}
\end{proof}

Therefore we change the header condition of the unrolled loop to $i~`cmp`~\hat{N}$.
Note that even though, we are calculating the rounding to a multiple of $f$ without the need for a slow divison operation.

\Cref{fig:impl:fixup:header-cond:firm} shows the comparison of the original condition to the changed header condition, for the loop shown in \Cref{fig:impl:fixup:fixup-firm-loop}.
It is to be noted that the graph in graph with bound $\hat{N}$ can be constant folded to the same size, as the original header.

\begin{figure}[H]
    \centering
    % Scale factor 0.010830324909747292
\definecolor{color8}{RGB}{192,192,192}
\definecolor{color9}{RGB}{255,255,255}
\definecolor{color10}{RGB}{153,255,153}
\definecolor{color11}{RGB}{255,255,153}
\definecolor{color12}{RGB}{153,153,255}
\definecolor{color13}{RGB}{0,150,60}
\definecolor{color14}{RGB}{0,0,0}
\definecolor{color15}{RGB}{100,100,255}
% Bounding Box: 1385.0, 2288.0
\begin{tikzpicture}
	% 1 node layouts
	\node[scale=0.39382999671808333, transform shape] at (18.93490027571706cm ,15.87725631768953cm) {Cmp b less 226};
	\node[fill=color10, draw, minimum width=0.779783393501805cm, minimum height=0.3249097472924188cm] (n20) at (20.304936376800093cm ,16.63537906137184cm) {};
	% 1 node layouts
	\node[scale=0.39382999671808333, transform shape] at (20.304936376800093cm ,16.63537906137184cm) {Phi Is 219};
	\node[fill=color11, draw, minimum width=1.2346570397111913cm, minimum height=0.3249097472924188cm] (n21) at (19.802367645892012cm ,17.610108303249095cm) {};
	% 1 node layouts
	\node[scale=0.39382999671808333, transform shape] at (19.802367645892012cm ,17.610108303249095cm) {Const 0x0 Is 215};
	\node[fill=color9, draw, minimum width=0.855595667870036cm, minimum height=0.3249097472924188cm] (n22) at (18.64518908438132cm ,16.63537906137184cm) {};
	% 1 node layouts
	\node[scale=0.39382999671808333, transform shape] at (18.64518908438132cm ,16.63537906137184cm) {Sub Is 556};
	\node[fill=color11, draw, minimum width=1.6895306859205776cm, minimum height=0.3249097472924188cm] (n23) at (18.12366728488118cm ,17.610108303249095cm) {};
	% 1 node layouts
	\node[scale=0.39382999671808333, transform shape] at (18.12366728488118cm ,17.610108303249095cm) {Const 0x9 Is 555};
	\node[fill=color9, draw, minimum width=0.9747292418772563cm, minimum height=0.3249097472924188cm] (n24) at (21.12366728488118cm ,17.610108303249095cm) {};
	% 1 node layouts
	\node[scale=0.39382999671808333, transform shape] at (21.12366728488118cm ,17.610108303249095cm) {Proj Is 0 225};
	\node[fill=color9, draw, minimum width=1.3754512635379061cm, minimum height=0.3249097472924188cm] (n25) at (21.12366728488118cm ,18.368231046931406cm) {};
	% 1 node layouts
	\node[scale=0.39382999671808333, transform shape] at (21.12366728488118cm ,18.368231046931406cm) {Proj T T\_result 224};
	\node[fill=color12, draw, minimum width=0.7256317689530686cm, minimum height=0.3249097472924188cm] (n26) at (21.12366728488118cm ,19.126353790613717cm) {};
	% 1 node layouts
	\node[scale=0.39382999671808333, transform shape] at (21.12366728488118cm ,19.126353790613717cm) {Call  222};
	\node[fill=color11, draw, minimum width=1.3646209386281587cm, minimum height=0.3249097472924188cm] (n27) at (19.925875004958936cm ,19.9927797833935cm) {};
	% 1 node layouts
	\node[scale=0.39382999671808333, transform shape] at (19.925875004958936cm ,19.9927797833935cm) {Address \&\_n P 220};
	\node[fill=color11, draw, minimum width=1.2129963898916967cm, minimum height=0.3249097472924188cm] (n28) at (21.43129016741381cm ,19.9927797833935cm) {};
	% 1 node layouts
	\node[scale=0.39382999671808333, transform shape] at (21.43129016741381cm ,19.9927797833935cm) {Const 0xF Is 221};
	\node[fill=color13, draw, minimum width=0.21660649819494585cm, minimum height=0.21660649819494585cm] (n29) at (20.651506773912008cm ,15.87725631768953cm) {};
	\node[fill=color13, draw, minimum width=0.21660649819494585cm, minimum height=0.21660649819494585cm] (n30) at (21.935941653112227cm ,17.610108303249095cm) {};
	\node[fill=color13, draw, minimum width=0.21660649819494585cm, minimum height=0.21660649819494585cm] (n31) at (22.36269810965208cm ,19.9927797833935cm) {};
	\node[fill=color13, draw, minimum width=0.21660649819494585cm, minimum height=0.21660649819494585cm] (n32) at (21.674972477883124cm ,16.63537906137184cm) {};
	\node[fill=color13, draw, minimum width=0.21660649819494585cm, minimum height=0.21660649819494585cm] (n33) at (20.218293777522113cm ,15.87725631768953cm) {};
	\draw[color=color14, -latex] (19.2246114670528cm ,16.03971119133574cm) -- (19.2246114670528cm ,16.256317689530686cm) -- (20.045008578966158cm ,16.256317689530686cm) -- (20.045008578966158cm ,16.47292418772563cm);
	\node[] at (19.31125406633078cm ,16.197427797833935cm) {
		\scalebox{0.39382999671808333}{0}
	};
	\draw[color=color14, -latex] (18.64518908438132cm ,16.03971119133574cm) -- (18.64518908438132cm ,16.47292418772563cm);
	\node[] at (18.731831683659298cm ,16.247560356498195cm) {
		\scalebox{0.39382999671808333}{1}
	};
	\draw[color=color14, -latex] (20.109990528424643cm ,16.79783393501805cm) -- (20.109990528424643cm ,17.231046931407942cm) -- (19.802367645892012cm ,17.231046931407942cm) -- (19.802367645892012cm ,17.447653429602887cm);
	\node[] at (20.19663312770262cm ,17.005683100180505cm) {
		\scalebox{0.39382999671808333}{0}
	};
	\draw[color=color14, -latex] (20.499882225175543cm ,16.79783393501805cm) -- (20.499882225175543cm ,17.014440433212997cm) -- (21.935941653112227cm ,17.014440433212997cm) -- (21.935941653112227cm ,17.501805054151625cm);
	\node[] at (20.586524824453523cm ,16.955550541516246cm) {
		\scalebox{0.39382999671808333}{1}
	};
	\draw[color=color14, -latex] (18.85908800134883cm ,16.79783393501805cm) -- (18.85908800134883cm ,17.122743682310468cm) -- (20.879984974411865cm ,17.122743682310468cm) -- (20.879984974411865cm ,17.447653429602887cm);
	\node[] at (18.945730600626806cm ,17.005683100180505cm) {
		\scalebox{0.39382999671808333}{0}
	};
	\draw[color=color14, -latex] (18.43129016741381cm ,16.79783393501805cm) -- (18.43129016741381cm ,17.231046931407942cm) -- (18.12366728488118cm ,17.231046931407942cm) -- (18.12366728488118cm ,17.447653429602887cm);
	\node[] at (18.51793276669179cm ,17.005683100180505cm) {
		\scalebox{0.39382999671808333}{1}
	};
	\draw[color=color14, -latex] (21.12366728488118cm ,17.772563176895307cm) -- (21.12366728488118cm ,18.205776173285198cm);
	\node[] at (21.21030988415916cm ,17.98041234205776cm) {
		\scalebox{0.39382999671808333}{0}
	};
	\draw[color=color14, -latex] (21.12366728488118cm ,18.530685920577618cm) -- (21.12366728488118cm ,18.96389891696751cm);
	\node[] at (21.21030988415916cm ,18.73853508574007cm) {
		\scalebox{0.39382999671808333}{0}
	};
	\draw[color=color14, -latex] (20.881790028563493cm ,19.28880866425993cm) -- (20.88179002856349cm ,19.613718411552345cm) -- (19.925875004958936cm ,19.613718411552345cm) -- (19.925875004958936cm ,19.83032490974729cm);
	\node[] at (20.96843262784147cm ,19.496657829422382cm) {
		\scalebox{0.39382999671808333}{1}
	};
	\draw[color=color14, -latex] (21.12366728488118cm ,19.28880866425993cm) -- (21.12366728488118cm ,19.613718411552345cm) -- (21.43129016741381cm ,19.613718411552345cm) -- (21.43129016741381cm ,19.83032490974729cm);
	\node[] at (21.21030988415916cm ,19.496657829422382cm) {
		\scalebox{0.39382999671808333}{2}
	};
	\draw[color=color15, -latex] (21.36554454119887cm ,19.28880866425993cm) -- (21.36554454119887cm ,19.505415162454874cm) -- (22.36269810965208cm ,19.505415162454874cm) -- (22.36269810965208cm ,19.884476534296027cm);
	\node[] at (21.45218714047685cm ,19.446525270758123cm) {
		\scalebox{0.39382999671808333}{0}
	};
	\draw[color=color14, -latex] (20.651506773912008cm ,15.985559566787003cm) -- (20.651506773912008cm ,16.256317689530686cm) -- (20.564864174634028cm ,16.256317689530686cm) -- (20.564864174634028cm ,16.47292418772563cm);
	\node[] at (20.738149373189984cm ,16.170351985559567cm) {
		\scalebox{0.39382999671808333}{0}
	};
	\draw[color=color14, -latex] (21.674972477883124cm ,16.743682310469314cm) -- (21.674972477883124cm ,17.231046931407942cm) -- (21.367349595350493cm ,17.231046931407942cm) -- (21.367349595350493cm ,17.447653429602887cm);
	\node[] at (21.761615077161103cm ,16.951531475631768cm) {
		\scalebox{0.39382999671808333}{0}
	};
	\draw[color=color14, -latex] (20.218293777522113cm ,15.985559566787003cm) -- (20.218293777522113cm ,16.256317689530686cm) -- (20.304936376800093cm ,16.256317689530686cm) -- (20.304936376800093cm ,16.47292418772563cm);
	\node[] at (20.304936376800093cm ,16.170351985559567cm) {
		\scalebox{0.39382999671808333}{0}
	};
\end{tikzpicture}

    \caption{The changed header condition for \cref{fig:impl:fixup:duff:fixup-firm-loop} after constant folding}
    \label{fig:impl:fixup:header-cond:firm}
\end{figure}

\begin{figure}[H]
	\begin{algorithmic}
		\State $i \gets 0$
		\While{i < 29}
			\State \Call{Print}{$\text{HelloWorld}$}
			\State $i \gets i + 3$
		\EndWhile
	\end{algorithmic}
	\caption{An example loop, for which the unrolling process will be explained}\label{fig:impl:fixup:fixup-firm-loop}
\end{figure}

We know that $c \cdot (f + 1)$ cannot over- or underflow, as per the preconditions laid out in \Cref{sec:impl:unrollability}.
Since $f > 0$, $c \cdot (f - 1)$ will therefore also not overflow.
Though, $N - c \cdot (f - 1)$ can still over- or underflow, due to the subtraction of $c \cdot (f - 1)$ from $N$, and hence there is nevertheless a possibility to construct an example where this change does not conserve semantics.

Suppose the datatype of a loop with parameters $N = 2, I = 0, c = 1, cmp = <$ is a 32-bit unsigned integer, and we unroll this loop by a factor of four.
In this case $\hat{N} = 2 - 1 \cdot (4 - 1) = -1 \overset{\text{unsigned integer}}{=} t_{max}$.
Thus, the loop would run $t_{max} > 2$ times.

To circumvent this problem, we use \Cref{alg:basics:overflow:detect} from \Cref{sec:basics:overflow} as a check for over- or underflows of the operation.
We implement this check by placing a block between the header and its predecessors.
If an under- or overflow is detected, the control flow will jump directly the fixup code.
Otherwise, it will route the control flow to the header and let the loop progress as normal, given that $\hat{N}$ now restores semantics in the header.
\Cref{alg:impl:fixup:header-cond:preheader} shows the creation of this structure in~\libFIRM.

\begin{algorithm}[H]
    \begin{algorithmic}
        \Function{CreatePreHeader}{$header, firstFixupBlock: \text{block}, loop: \text{loop}$}
            \State $pre \gets $ \Call{NewEmptyBlock}{}
            \State $pre.predecessors \gets \{node \vert node \in header.predecessors, node \notin loop\}$
            \For{$phi \in header.phis$}
                \State $phi' \gets $ \Call{NewPhiInBlock}{$pre$}
                \State $phi'.predecessors \gets \{node \vert node \in phi.predecessors, node.block \notin loop\}$
                \State $phi.predecessors \gets \{phi'\} \cup \{node \vert node \in phi.predecessors, node.block \in loop\}$
                \For{$succ \in phi.successors$}
                    \If{$succ.block$~dominated by~$firstFixupBlock$}
                        \State $succ.predecessors$.prepend($pre$)
                    \EndIf
                \EndFor
            \EndFor
            \State $(trueExit, falseExit) \gets $\Call{CreateOverflowCondition}{}
            \State $firstFixupBlock.predecessors.prepend(falseExit)$
            \State $header.predecessors \gets \{trueExit\} \cup \{node \vert node \in header.predecessors, node \in loop\}$
        \EndFunction
    \end{algorithmic}
    \caption{Algorithm that creates the check to ensure $\hat{N}$ does not over- or underflow}
    \label{alg:impl:fixup:header-cond:preheader}
\end{algorithm}

\newpage

\subsection{Generalized Duff's device}\label{sec:impl:fixup:duff}

\Cref{sec:basics:duffs} describes the original version of Duff's device.
The problem with this initial approach is that it assumes $c = 1$, even though \hyperref[sec:impl::def-c]{$c$ is defined} as any non-zero integer in the considered loops.
Therefore, a need for generalization arises.

Using equations~\ref{eqn:impl:fixup:duff:conserve-semantics-identity} through~\ref{eqn:impl:fixup:duff:fixup-i-mult} and the general idea of Duff's device (see \Cref{sec:basics:duffs}), we will create fixup code in form of a generalized Duff's device.
The structure of this fixup code can be seen in \Cref{fig:impl:fixup:duff:fixup-M_fixup}, which we then practically implement, using \Cref{eqn:impl:fixup:duff:fixup-i-mult}, as shown in \Cref{fig:impl:fixup:duff:fixup-bound}.

\begin{figure}[H]
    \begin{algorithmic}
        \Switch{$\Mfixup$}
            \Case{$f$}
                \State \Call{Body}{}
            \EndCase
            \Case{$f - 1$}
                \State \Call{Body}{}
            \EndCase
            \State \ldots
            \Case{$1$}
                \State \Call{Body}{}
            \EndCase
        \EndSwitch
    \end{algorithmic}
    \caption{Generalized Duff's device fixup code based on $\Mfixup$}
    \label{fig:impl:fixup:duff:fixup-M_fixup}
\end{figure}

\begin{figure}[H]
    \begin{algorithmic}
        \Switch{$N - \ipl + (c - 1)$}
            \Case{$c \cdot f$}
                \State \Call{Body}{}
            \EndCase
            \Case{$c \cdot (f - 1)$}
                \State \Call{Body}{}
            \EndCase
            \State \ldots
            \Case{$c \cdot 1$}
                \State \Call{Body}{}
            \EndCase
        \EndSwitch
    \end{algorithmic}
    \caption{Generalized duff's device fixup code based on variables present}
    \label{fig:impl:fixup:duff:fixup-bound}
\end{figure}

For the fixup code to work correctly, it is to be ensured that $(f + 1) \cdot c$ does not overflow, as otherwise, the interval \cinterval could potentially be invalid, iff an integer over- or underflow occurs, meaning $
\begin{cases}
    c \cdot (f + 1) < 0 &, \medspace c > 0\\
    c \cdot (f + 1) > 0 &, \medspace c < 0
\end{cases}$.
To avoid these problems altogether, c is restricted to being a compile-time constant, such that for integers defined from $t_{min}$ to $t_{max}$, $c \in \lbrack \frac{t_{min}}{f + 1}, \frac{t_{max}}{f + 1} \rbrack$.
Using this restriction, it can be asserted that $c \cdot (f + 1) \in \lbrack t_{min}, t_{max} \rbrack$ and therefore does not overflow.
\Cref{alg:impl:fixup:duff:create-fixup} details how the mechanics described above are translated into \libFIRM.
At first, we duplicate the loop body $f - 1$ times and add keepalive edges to all duplicated nodes, to make sure they do not disappear through implicit premature optimizations.
Then we will create the \textit{fixup header}, meaning a block, with the calculation of $n \coloneqq N - i + (c \mp 1)$.
$f - 1$ newly created condition blocks will then use the calculated value by the header.
In the $i^{\text{th}}$ (counting starts at 0) condition block, it will be checked whether $n$ is in the interval spanned by $c \cdot (f - i)$ and  $c \cdot (f - i + 1)$.\footnote{Please note: $c$ being positive or not determines, which limit is the upper and which is the lower bound.}
After this, we will wire all duplicated blocks such that they are reachable by the conditions.
Further, upon false evaluation of a condition, the following condition is evaluated, except if it is the last condition, in which case the false target is the post loop block.
Additionally, except in the case of the first duplicated header, they are attached to the previous blocks as fallthrough.
Lastly, the last block of the fixup code now precedes the post loop block, and the false exit fo the last condition.
An example of the result for creating fixup code for a loop and for $f = 2$, as seen in \Cref{fig:impl:fixup:fixup-firm-loop}, can be seen in \Cref{fig:impl:fixup:duff:fixup-firm}.
Further, \Cref{fig:impl:fixup:duff:fixup-firm-comp} shows the completed unroll process with the added generalized Duff's device, given $f = 2$.
Once this process is completed~\Cref{fig:impl:fixup:duff:general-loop} shows the resulting structure in pseudo-code notation.

\begin{figure}[H]
    \centering
    \begin{algorithmic}
        \Function{Foo}{$I \in \mathbb{Z}, N \in \mathbb{Z}, c \in \mathbb{Z} \backslash \zeroset, cmp \in \{<, >, \leq, \geq\}$}
        \State $i \gets I$
        \If{$\neg$\Call{SubtractionWillLeaveBounds}{$N - c \cdot (f - 1)$}}
            \While{$i~`cmp`~ (N - c \cdot (f - 1))$}
                \State \Call{DoSomething}{} \Comment{$f$ times}
                \State $i \gets i + c$ \Comment{$f$ times}
            \EndWhile
        \EndIf
        \Switch{$N - i + (c \mp 1)$}
            \Case{$\lbrack c \cdot (f - 1), c \cdot f \lbrack$} \Comment flip bounds for $c < 0$
                \State \Call{DoSomething}{}
                \State $i \gets i + c$ \Comment{Fall-through}
            \EndCase
            \Case{$\left \lbrack c \cdot (f - 2), c \cdot (f - 1) \right \lbrack$} \Comment flip bounds for $c < 0$
                \State \Call{DoSomething}{}
                \State $i \gets i + c$ \Comment{Fall-through}
            \EndCase
            \State
            \ldots
            \Case{$\lbrack c \cdot 1, c \cdot 2\lbrack$} \Comment flip bounds for $c < 0$
                \State \Call{DoSomething}{}
                \State $i \gets i + c$
            \EndCase
        \EndSwitch
        \EndFunction
    \end{algorithmic}
    \caption{The general form of a loop starting at $I$ counting in increments of $c$ up to $N$ transformed by loop unrolling with generalized Duff's device fixup}
    \label{fig:impl:fixup:duff:general-loop}
\end{figure}

\begin{algorithm}
    \caption{Algorithm to build switch fixup for a given loop in \libFIRM}
    \label{alg:impl:fixup:duff:create-fixup}
    \begin{algorithmic}
        \Function{CreateFixupSwitch}{$loop: \text{Loop}, factor: \mathbb{N}_{>1}$}
            \For{$i \in \{0, .., (factor - 1)\}$}
                \State \Call{DuplicateBody}{$loop$}
            \EndFor
            \For{$node \in allCopiedNodes$}
                \If{$\neg$\Call{HasKeepalive}{$node.link$}}
                    \State \Call{AddKeepAlive}{$node$} \Comment{Prevent premature disappearance}
                \EndIf
            \EndFor
            \State $relation \gets header.cmp.relation$
            \State $inverseRelation \gets $ \Call{GetInverseRelation}{$relation$}
            \State $duffHeader \gets NewEmptyBlock$
            \State $duffHeader.predecessors \gets \{loop.header\}$
            \State $val \gets duffHeader.addNode(N - i + (c
                \left\{\!\begin{aligned}
                   - &, \medspace c > 0\\
                   + &, \medspace c < 0
                \end{aligned}\right\} 1)
            )$
            \State $i \gets 0$
            \State $prevLast: \text{Block}$
            \State $prevCond: \text{Block}$
            \For{$body \in duplicatedLoopBodies$}
                \State $firstBlock \gets$ \Call{GetFirstBlockInBody}{$body$}
                \State $condBlock \gets$ \Call{NewEmptyBlock}{}
                \State $cond \gets val \medspace `relation` \medspace (factor - i) \medspace \wedge \medspace val \medspace `inverseRelation` \medspace (factor - i - 1)$
                \State $condBlock.addNode(cond)$
                \State $condBlock.predecessors \gets \begin{cases}\{duffHeader\} &, \medspace i = 0\\ \{prevCond.falseExit, \medspace prevLast\} &, \medspace i \neq 0\end{cases}$
                \State $firstBlock.predecessors \gets cond.trueExit$
                \State $i \gets i + 1$
                \State $prevLast \gets$ \Call{GetLastBlockInBody}{$body$}
                \State $prevCond \gets cond$
                \For{$phi \in body.phis$}
                    \State $phi.predecessors \gets \begin{cases}\{phi.link.predecessors\} &, \medspace i = 0\\ \{phi.link.predecessors, \medspace prevLast.exitFor(phi)\} &, \medspace i \neq 0\end{cases}$
                \EndFor
            \EndFor
            \For{$node \in allCopiedNodes$}
                \If{$\neg$\Call{HasKeepalive}{$node.link$}}
                    \State \Call{RemoveKeepAlive}{$node$}
                \EndIf
            \EndFor
            \State $postLoopBlock.predecessors \gets \{prevCond.falseExit, \medspace prevLast\}$
            \State \Call{RewirePhis}{} \Comment{Wire just like for duplicated loop body phi's}
        \EndFunction
    \end{algorithmic}
\end{algorithm}

\begin{figure}[H]
	\begin{algorithmic}
		\State $i \gets 0$
		\If{$\neg$\Call{SubtractionWillLeaveBounds}{$29 - 6$}}
			\While{i < 23}
				\State \Call{Print}{$\text{HelloWorld}$}
				\State $i \gets i + 3$
				\State \Call{Print}{$\text{HelloWorld}$}
				\State $i \gets i + 3$
			\EndWhile
		\EndIf
		\Switch{31 - i}
			\Case{$\lbrack 3, 6 \lbrack$}
				\State \Call{Print}{$\text{HelloWorld}$}
				\State $i \gets i + 3$
			\EndCase
		\EndSwitch
	\end{algorithmic}
	\caption{An example loop, as seen in \Cref{fig:impl:fixup:fixup-firm-loop}, unrolled by a factor of two, and with generalized Duff's device fixup}\label{fig:impl:fixup:duff:fixup-firm-comp}
\end{figure}

\begin{figure}
	\begin{adjustbox}{max height=\textheight}
		% Scale factor 0.02717391304347826
\definecolor{color0}{RGB}{222,239,234}
\definecolor{color1}{RGB}{153,153,255}
\definecolor{color2}{RGB}{255,255,255}
\definecolor{color3}{RGB}{255,255,153}
\definecolor{color4}{RGB}{153,255,153}
\definecolor{color5}{RGB}{255,153,153}
\definecolor{color6}{RGB}{0,150,60}
\definecolor{color7}{RGB}{255,0,0}
\definecolor{color8}{RGB}{100,100,255}
\definecolor{color9}{RGB}{0,0,0}
% Bounding Box: 552.0, 1332.0
\begin{tikzpicture}
	\node[fill=color0, draw, minimum width=13.301630434782608cm, minimum height=9.592391304347826cm] (n1) at (8.75679347826087cm ,-29.36141304347826cm) {};
	% 1 node layouts
	\node[scale=0.9881422924901185, transform shape] at (8.75679347826087cm ,-24.875594429347824cm) {Duff header};
	\node[fill=color0, draw, minimum width=10.516304347826086cm, minimum height=9.864130434782608cm] (n2) at (5.665760869565217cm ,-40.17663043478261cm) {};
	% 1 node layouts
	\node[scale=0.9881422924901185, transform shape] at (5.665760869565217cm ,-35.55494225543478cm) {Condition block};
	\node[fill=color0, draw, minimum width=12.744565217391305cm, minimum height=7.690217391304348cm] (n3) at (8.994565217391305cm ,-50.04076086956522cm) {};
	% 1 node layouts
	\node[scale=0.9881422924901185, transform shape] at (8.994565217391305cm ,-46.50602921195652cm) {Body};
	\node[fill=color0, draw, minimum width=6.168478260869565cm, minimum height=5.788043478260869cm] (n4) at (8.179347826086955cm ,-57.86684782608695cm) {};
	% 1 node layouts
	\node[scale=0.9881422924901185, transform shape] at (8.179347826086955cm ,-55.283203125cm) {Post loop block};
	\node[fill=color1, draw, minimum width=1.7119565217391304cm, minimum height=0.8152173913043478cm] (n5) at (3.369565217391304cm ,-27.907608695652172cm) {};
	% 1 node layouts
	\node[scale=0.9881422924901185, transform shape] at (3.369565217391304cm ,-27.907608695652172cm) {Phi  420};
	\node[fill=color2, draw, minimum width=2.1467391304347827cm, minimum height=0.8152173913043478cm] (n6) at (11.888586956521738cm ,-31.71195652173913cm) {};
	% 1 node layouts
	\node[scale=0.9881422924901185, transform shape] at (11.888586956521738cm ,-31.71195652173913cm) {Add Is 425};
	\node[fill=color3, draw, minimum width=3.0978260869565215cm, minimum height=0.8152173913043478cm] (n7) at (13.451086956521738cm ,-29.809782608695652cm) {};
	% 1 node layouts
	\node[scale=0.9881422924901185, transform shape] at (13.451086956521738cm ,-29.809782608695652cm) {Const 0x2 Is 514};
	\node[fill=color2, draw, minimum width=2.0652173913043477cm, minimum height=0.8152173913043478cm] (n8) at (10.326086956521738cm ,-29.809782608695652cm) {};
	% 1 node layouts
	\node[scale=0.9881422924901185, transform shape] at (10.326086956521738cm ,-29.809782608695652cm) {Sub Is 423};
	\node[fill=color4, draw, minimum width=1.9565217391304348cm, minimum height=0.8152173913043478cm] (n9) at (7.96195652173913cm ,-27.907608695652172cm) {};
	% 1 node layouts
	\node[scale=0.9881422924901185, transform shape] at (7.96195652173913cm ,-27.907608695652172cm) {Phi Is 440};
	\node[fill=color4, draw, minimum width=1.9565217391304348cm, minimum height=0.8152173913043478cm] (n10) at (12.051630434782608cm ,-27.907608695652172cm) {};
	% 1 node layouts
	\node[scale=0.9881422924901185, transform shape] at (12.051630434782608cm ,-27.907608695652172cm) {Phi Is 441};
	\node[fill=color3, draw, minimum width=3.0978260869565215cm, minimum height=0.8152173913043478cm] (n11) at (10.869565217391305cm ,-26.005434782608695cm) {};
	% 1 node layouts
	\node[scale=0.9881422924901185, transform shape] at (10.869565217391305cm ,-26.005434782608695cm) {Const 0x0 Is 215};
	\node[fill=color5, draw, minimum width=1.875cm, minimum height=0.8152173913043478cm] (n12) at (5.665760869565217cm ,-26.005434782608695cm) {};
	% 1 node layouts
	\node[scale=0.9881422924901185, transform shape] at (5.665760869565217cm ,-26.005434782608695cm) {Jmp  426};
	\node[fill=color6, draw, minimum width=0.5434782608695652cm, minimum height=0.5434782608695652cm] (n13) at (2.8260869565217392cm ,-26.005434782608695cm) {};
	\node[fill=color6, draw, minimum width=0.5434782608695652cm, minimum height=0.5434782608695652cm] (n14) at (3.9130434782608696cm ,-26.005434782608695cm) {};
	\node[fill=color6, draw, minimum width=0.5434782608695652cm, minimum height=0.5434782608695652cm] (n15) at (7.418478260869565cm ,-26.005434782608695cm) {};
	\node[fill=color6, draw, minimum width=0.5434782608695652cm, minimum height=0.5434782608695652cm] (n16) at (8.505434782608695cm ,-26.005434782608695cm) {};
	\node[fill=color6, draw, minimum width=0.5434782608695652cm, minimum height=0.5434782608695652cm] (n17) at (13.233695652173912cm ,-26.005434782608695cm) {};
	\node[fill=color6, draw, minimum width=0.5434782608695652cm, minimum height=0.5434782608695652cm] (n18) at (11.345108695652174cm ,-33.47826086956522cm) {};
	\node[fill=color6, draw, minimum width=0.5434782608695652cm, minimum height=0.5434782608695652cm] (n19) at (12.432065217391305cm ,-33.47826086956522cm) {};
	\node[fill=color6, draw, minimum width=0.5434782608695652cm, minimum height=0.5434782608695652cm] (n20) at (2.8260869565217392cm ,-29.809782608695652cm) {};
	\node[fill=color6, draw, minimum width=0.5434782608695652cm, minimum height=0.5434782608695652cm] (n21) at (3.9130434782608696cm ,-29.809782608695652cm) {};
	\node[fill=color5, draw, minimum width=2.934782608695652cm, minimum height=0.8152173913043478cm] (n22) at (3.953804347826087cm ,-44.29347826086956cm) {};
	% 1 node layouts
	\node[scale=0.9881422924901185, transform shape] at (3.953804347826087cm ,-44.29347826086956cm) {Proj X false 437};
	\node[fill=color5, draw, minimum width=2.8260869565217392cm, minimum height=0.8152173913043478cm] (n23) at (7.377717391304348cm ,-44.29347826086956cm) {};
	% 1 node layouts
	\node[scale=0.9881422924901185, transform shape] at (7.377717391304348cm ,-44.29347826086956cm) {Proj X true 436};
	\node[fill=color5, draw, minimum width=2.0652173913043477cm, minimum height=0.8152173913043478cm] (n24) at (5.665760869565217cm ,-42.391304347826086cm) {};
	% 1 node layouts
	\node[scale=0.9881422924901185, transform shape] at (5.665760869565217cm ,-42.391304347826086cm) {Cond  435};
	\node[fill=color2, draw, minimum width=2.092391304347826cm, minimum height=0.8152173913043478cm] (n25) at (5.665760869565217cm ,-40.48913043478261cm) {};
	% 1 node layouts
	\node[scale=0.9881422924901185, transform shape] at (5.665760869565217cm ,-40.48913043478261cm) {And b 434};
	\node[fill=color2, draw, minimum width=3.4510869565217392cm, minimum height=0.8152173913043478cm] (n26) at (7.7853260869565215cm ,-38.58695652173913cm) {};
	% 1 node layouts
	\node[scale=0.9881422924901185, transform shape] at (7.7853260869565215cm ,-38.58695652173913cm) {Cmp b greater 516};
	\node[fill=color3, draw, minimum width=3.0978260869565215cm, minimum height=0.8152173913043478cm] (n27) at (8.967391304347826cm ,-36.68478260869565cm) {};
	% 1 node layouts
	\node[scale=0.9881422924901185, transform shape] at (8.967391304347826cm ,-36.68478260869565cm) {Const 0x2 Is 514};
	\node[fill=color2, draw, minimum width=3.9402173913043477cm, minimum height=0.8152173913043478cm] (n28) at (3.546195652173913cm ,-38.58695652173913cm) {};
	% 1 node layouts
	\node[scale=0.9881422924901185, transform shape] at (3.546195652173913cm ,-38.58695652173913cm) {Cmp b less\_equal 519};
	\node[fill=color3, draw, minimum width=3.0978260869565215cm, minimum height=0.8152173913043478cm] (n29) at (2.3641304347826084cm ,-36.68478260869565cm) {};
	% 1 node layouts
	\node[scale=0.9881422924901185, transform shape] at (2.3641304347826084cm ,-36.68478260869565cm) {Const 0x5 Is 518};
	\node[fill=color6, draw, minimum width=0.5434782608695652cm, minimum height=0.5434782608695652cm] (n30) at (6.603260869565217cm ,-36.68478260869565cm) {};
	\node[fill=color6, draw, minimum width=0.5434782608695652cm, minimum height=0.5434782608695652cm] (n31) at (4.728260869565217cm ,-36.68478260869565cm) {};
	\node[fill=color1, draw, minimum width=2.527173913043478cm, minimum height=0.8152173913043478cm] (n32) at (6.154891304347826cm ,-51.440217391304344cm) {};
	% 1 node layouts
	\node[scale=0.9881422924901185, transform shape] at (6.154891304347826cm ,-51.440217391304344cm) {Proj M M 413};
	\node[fill=color1, draw, minimum width=1.8206521739130435cm, minimum height=0.8152173913043478cm] (n33) at (6.154891304347826cm ,-49.53804347826087cm) {};
	% 1 node layouts
	\node[scale=0.9881422924901185, transform shape] at (6.154891304347826cm ,-49.53804347826087cm) {Call  414};
	\node[fill=color3, draw, minimum width=4.076086956521739cm, minimum height=0.8152173913043478cm] (n34) at (6.154891304347826cm ,-47.63586956521739cm) {};
	% 1 node layouts
	\node[scale=0.9881422924901185, transform shape] at (6.154891304347826cm ,-47.63586956521739cm) {Address \&\_printf P 232};
	\node[fill=color3, draw, minimum width=3.8043478260869565cm, minimum height=0.8152173913043478cm] (n35) at (10.638586956521738cm ,-47.63586956521739cm) {};
	% 1 node layouts
	\node[scale=0.9881422924901185, transform shape] at (10.638586956521738cm ,-47.63586956521739cm) {Address \&str.0 P 236};
	\node[fill=color5, draw, minimum width=1.875cm, minimum height=0.8152173913043478cm] (n36) at (14.021739130434781cm ,-47.63586956521739cm) {};
	% 1 node layouts
	\node[scale=0.9881422924901185, transform shape] at (14.021739130434781cm ,-47.63586956521739cm) {Jmp  418};
	\node[fill=color6, draw, minimum width=0.5434782608695652cm, minimum height=0.5434782608695652cm] (n37) at (3.3016304347826084cm ,-47.63586956521739cm) {};
	\node[fill=color6, draw, minimum width=0.5434782608695652cm, minimum height=0.5434782608695652cm] (n38) at (6.154891304347826cm ,-53.20652173913043cm) {};
	\node[fill=color5, draw, minimum width=2.282608695652174cm, minimum height=0.8152173913043478cm] (n39) at (7.832880434782607cm ,-59.94565217391304cm) {};
	% 1 node layouts
	\node[scale=0.9881422924901185, transform shape] at (7.832880434782607cm ,-59.94565217391304cm) {Return  245};
	\node[fill=color3, draw, minimum width=3.0978260869565215cm, minimum height=0.8152173913043478cm] (n40) at (9.307065217391303cm ,-58.04347826086956cm) {};
	% 1 node layouts
	\node[scale=0.9881422924901185, transform shape] at (9.307065217391303cm ,-58.04347826086956cm) {Const 0x0 Is 215};
	\node[fill=color1, draw, minimum width=1.7119565217391304cm, minimum height=0.8152173913043478cm] (n41) at (6.358695652173911cm ,-58.04347826086956cm) {};
	% 1 node layouts
	\node[scale=0.9881422924901185, transform shape] at (6.358695652173911cm ,-58.04347826086956cm) {Phi  404};
	\node[fill=color6, draw, minimum width=0.5434782608695652cm, minimum height=0.5434782608695652cm] (n42) at (6.902173913043477cm ,-56.27717391304348cm) {};
	\node[fill=color6, draw, minimum width=0.5434782608695652cm, minimum height=0.5434782608695652cm] (n43) at (5.815217391304346cm ,-56.27717391304348cm) {};
	\draw[color=color7, -latex] (5.665760869565217cm ,-35.244565217391305cm) -- (5.665760869565217cm ,-26.413043478260867cm);
	\node[] at (5.883152173913043cm ,-34.723059612771735cm) {
		\scalebox{0.9881422924901185}{0}
	};
	\draw[color=color7, -latex] (8.994565217391305cm ,-46.19565217391304cm) -- (8.994565217391305cm ,-45.65217391304348cm) -- (7.377717391304348cm ,-45.65217391304348cm) -- (7.377717391304348cm ,-44.701086956521735cm);
	\node[] at (9.21195652173913cm ,-45.79993206521739cm) {
		\scalebox{0.9881422924901185}{0}
	};
	\draw[color=color7, -latex] (6.637228260869564cm ,-54.97282608695652cm) -- (6.637228260869564cm ,-54.42934782608695cm) -- (2.3369565217391304cm ,-54.42934782608695cm) -- (2.3369565217391304cm ,-45.65217391304348cm) -- (3.953804347826087cm ,-45.65217391304348cm) -- (3.953804347826087cm ,-44.701086956521735cm);
	\node[] at (6.85461956521739cm ,-54.57710597826087cm) {
		\scalebox{0.9881422924901185}{0}
	};
	\draw[color=color7, -latex] (9.721467391304346cm ,-54.97282608695652cm) -- (9.721467391304346cm ,-54.42934782608695cm) -- (14.021739130434781cm ,-54.42934782608695cm) -- (14.021739130434781cm ,-48.04347826086956cm);
	\node[] at (9.938858695652172cm ,-54.57710597826087cm) {
		\scalebox{0.9881422924901185}{1}
	};
	\draw[color=color8, -latex] (2.9415760869565215cm ,-27.5cm) -- (2.9415760869565215cm ,-26.956521739130434cm) -- (2.8260869565217392cm ,-26.956521739130434cm) -- (2.8260869565217392cm ,-26.277173913043477cm);
	\node[] at (3.1589673913043477cm ,-27.104279891304348cm) {
		\scalebox{0.9881422924901185}{0}
	};
	\draw[color=color8, -latex] (3.797554347826087cm ,-27.5cm) -- (3.797554347826087cm ,-26.956521739130434cm) -- (3.9130434782608696cm ,-26.956521739130434cm) -- (3.9130434782608696cm ,-26.277173913043477cm);
	\node[] at (4.014945652173913cm ,-27.104279891304348cm) {
		\scalebox{0.9881422924901185}{1}
	};
	\draw[color=color9, -latex] (11.351902173913043cm ,-31.304347826086957cm) -- (11.351902173913043cm ,-30.76086956521739cm) -- (10.326086956521738cm ,-30.76086956521739cm) -- (10.326086956521738cm ,-30.217391304347824cm);
	\node[] at (11.56929347826087cm ,-30.908627717391305cm) {
		\scalebox{0.9881422924901185}{0}
	};
	\draw[color=color9, -latex] (12.425271739130434cm ,-31.304347826086957cm) -- (12.425271739130434cm ,-30.76086956521739cm) -- (13.451086956521738cm ,-30.76086956521739cm) -- (13.451086956521738cm ,-30.217391304347824cm);
	\node[] at (12.64266304347826cm ,-30.908627717391305cm) {
		\scalebox{0.9881422924901185}{1}
	};
	\draw[color=color9, -latex] (9.809782608695652cm ,-29.402173913043477cm) -- (9.809782608695652cm ,-28.858695652173914cm) -- (7.96195652173913cm ,-28.858695652173914cm) -- (7.96195652173913cm ,-28.315217391304348cm);
	\node[] at (10.027173913043478cm ,-29.006453804347824cm) {
		\scalebox{0.9881422924901185}{0}
	};
	\draw[color=color9, -latex] (10.842391304347826cm ,-29.402173913043477cm) -- (10.842391304347826cm ,-28.858695652173914cm) -- (12.051630434782608cm ,-28.858695652173914cm) -- (12.051630434782608cm ,-28.315217391304348cm);
	\node[] at (11.059782608695652cm ,-29.006453804347824cm) {
		\scalebox{0.9881422924901185}{1}
	};
	\draw[color=color9, -latex] (7.4728260869565215cm ,-27.5cm) -- (7.4728260869565215cm ,-26.956521739130434cm) -- (7.418478260869565cm ,-26.956521739130434cm) -- (7.418478260869565cm ,-26.277173913043477cm);
	\node[] at (7.690217391304348cm ,-27.104279891304348cm) {
		\scalebox{0.9881422924901185}{0}
	};
	\draw[color=color9, -latex] (8.451086956521738cm ,-27.5cm) -- (8.451086956521738cm ,-26.956521739130434cm) -- (8.505434782608695cm ,-26.956521739130434cm) -- (8.505434782608695cm ,-26.277173913043477cm);
	\node[] at (8.668478260869565cm ,-27.104279891304348cm) {
		\scalebox{0.9881422924901185}{1}
	};
	\draw[color=color9, -latex] (11.5625cm ,-27.5cm) -- (11.5625cm ,-26.956521739130434cm) -- (10.869565217391305cm ,-26.956521739130434cm) -- (10.869565217391305cm ,-26.413043478260867cm);
	\node[] at (11.779891304347826cm ,-27.104279891304348cm) {
		\scalebox{0.9881422924901185}{0}
	};
	\draw[color=color9, -latex] (12.540760869565217cm ,-27.5cm) -- (12.540760869565217cm ,-26.956521739130434cm) -- (13.233695652173912cm ,-26.956521739130434cm) -- (13.233695652173912cm ,-26.277173913043477cm);
	\node[] at (12.758152173913043cm ,-27.104279891304348cm) {
		\scalebox{0.9881422924901185}{1}
	};
	\draw[color=color9, -latex] (11.345108695652174cm ,-33.20652173913044cm) -- (11.351902173913043cm ,-32.119565217391305cm);
	\node[] at (11.567299220872963cm ,-32.68501613451087cm) {
		\scalebox{0.9881422924901185}{0}
	};
	\draw[color=color9, -latex] (12.432065217391305cm ,-33.20652173913044cm) -- (12.425271739130434cm ,-32.119565217391305cm);
	\node[] at (12.647736922554346cm ,-32.68501613451087cm) {
		\scalebox{0.9881422924901185}{0}
	};
	\draw[color=color8, -latex] (2.8260869565217392cm ,-29.538043478260867cm) -- (2.8260869565217392cm ,-28.858695652173914cm) -- (2.9415760869565215cm ,-28.858695652173914cm) -- (2.9415760869565215cm ,-28.315217391304348cm);
	\node[] at (3.043478260869565cm ,-29.07438858695652cm) {
		\scalebox{0.9881422924901185}{0}
	};
	\draw[color=color8, -latex] (3.9130434782608696cm ,-29.538043478260867cm) -- (3.9130434782608696cm ,-28.858695652173914cm) -- (3.797554347826087cm ,-28.858695652173914cm) -- (3.797554347826087cm ,-28.315217391304348cm);
	\node[] at (4.130434782608695cm ,-29.07438858695652cm) {
		\scalebox{0.9881422924901185}{0}
	};
	\draw[color=color7, -latex] (3.953804347826087cm ,-43.88586956521739cm) -- (3.953804347826087cm ,-43.34239130434783cm) -- (5.14945652173913cm ,-43.34239130434783cm) -- (5.14945652173913cm ,-42.79891304347826cm);
	\node[] at (4.171195652173913cm ,-43.490149456521735cm) {
		\scalebox{0.9881422924901185}{0}
	};
	\draw[color=color7, -latex] (7.377717391304348cm ,-43.88586956521739cm) -- (7.377717391304348cm ,-43.34239130434783cm) -- (6.182065217391304cm ,-43.34239130434783cm) -- (6.182065217391304cm ,-42.79891304347826cm);
	\node[] at (7.595108695652174cm ,-43.490149456521735cm) {
		\scalebox{0.9881422924901185}{0}
	};
	\draw[color=color9, -latex] (5.665760869565217cm ,-41.983695652173914cm) -- (5.665760869565217cm ,-40.89673913043478cm);
	\node[] at (5.883152173913043cm ,-41.462190047554344cm) {
		\scalebox{0.9881422924901185}{0}
	};
	\draw[color=color9, -latex] (6.188858695652174cm ,-40.08152173913043cm) -- (6.188858695652174cm ,-39.53804347826087cm) -- (7.7853260869565215cm ,-39.53804347826087cm) -- (7.7853260869565215cm ,-38.994565217391305cm);
	\node[] at (6.40625cm ,-39.68580163043478cm) {
		\scalebox{0.9881422924901185}{0}
	};
	\draw[color=color9, -latex] (5.142663043478261cm ,-40.08152173913043cm) -- (5.142663043478261cm ,-39.53804347826087cm) -- (3.546195652173913cm ,-39.53804347826087cm) -- (3.546195652173913cm ,-38.994565217391305cm);
	\node[] at (5.360054347826087cm ,-39.68580163043478cm) {
		\scalebox{0.9881422924901185}{1}
	};
	\draw[color=color9, -latex] (8.648097826086957cm ,-38.17934782608695cm) -- (8.648097826086957cm ,-37.63586956521739cm) -- (8.967391304347826cm ,-37.63586956521739cm) -- (8.967391304347826cm ,-37.09239130434783cm);
	\node[] at (8.865489130434783cm ,-37.783627717391305cm) {
		\scalebox{0.9881422924901185}{1}
	};
	\draw[color=color9, -latex] (6.922554347826087cm ,-38.17934782608695cm) -- (6.922554347826087cm ,-37.63586956521739cm) -- (6.603260869565217cm ,-37.63586956521739cm) -- (6.603260869565217cm ,-36.95652173913044cm);
	\node[] at (7.139945652173913cm ,-37.783627717391305cm) {
		\scalebox{0.9881422924901185}{0}
	};
	\draw[color=color9, -latex] (2.561141304347826cm ,-38.17934782608695cm) -- (2.561141304347826cm ,-37.63586956521739cm) -- (2.3641304347826084cm ,-37.63586956521739cm) -- (2.3641304347826084cm ,-37.09239130434783cm);
	\node[] at (2.778532608695652cm ,-37.783627717391305cm) {
		\scalebox{0.9881422924901185}{1}
	};
	\draw[color=color9, -latex] (4.53125cm ,-38.17934782608695cm) -- (4.53125cm ,-37.63586956521739cm) -- (4.728260869565217cm ,-37.63586956521739cm) -- (4.728260869565217cm ,-36.95652173913044cm);
	\node[] at (4.748641304347826cm ,-37.783627717391305cm) {
		\scalebox{0.9881422924901185}{0}
	};
	\draw[color=color8, -latex] (6.154891304347826cm ,-51.03260869565217cm) -- (6.154891304347826cm ,-49.94565217391304cm);
	\node[] at (6.372282608695652cm ,-50.51110309103261cm) {
		\scalebox{0.9881422924901185}{0}
	};
	\draw[color=color9, -latex] (6.154891304347826cm ,-49.130434782608695cm) -- (6.154891304347826cm ,-48.04347826086956cm);
	\node[] at (6.372282608695652cm ,-48.60892917798913cm) {
		\scalebox{0.9881422924901185}{1}
	};
	\draw[color=color9, -latex] (6.76177536231884cm ,-49.130434782608695cm) -- (6.76177536231884cm ,-48.58695652173913cm) -- (10.638586956521738cm ,-48.58695652173913cm) -- (10.638586956521738cm ,-48.04347826086956cm);
	\node[] at (6.979166666666666cm ,-48.73471467391304cm) {
		\scalebox{0.9881422924901185}{2}
	};
	\draw[color=color8, -latex] (5.5480072463768115cm ,-49.130434782608695cm) -- (5.548007246376811cm ,-48.58695652173913cm) -- (3.3016304347826084cm ,-48.58695652173913cm) -- (3.3016304347826084cm ,-47.90760869565217cm);
	\node[] at (5.765398550724638cm ,-48.73471467391304cm) {
		\scalebox{0.9881422924901185}{0}
	};
	\draw[color=color8, -latex] (6.154891304347826cm ,-52.93478260869565cm) -- (6.154891304347826cm ,-51.84782608695652cm);
	\node[] at (6.372282608695652cm ,-52.413277004076086cm) {
		\scalebox{0.9881422924901185}{1}
	};
	\draw[color=color8, -latex] (7.262228260869564cm ,-59.53804347826087cm) -- (7.262228260869564cm ,-58.994565217391305cm) -- (6.358695652173911cm ,-58.994565217391305cm) -- (6.358695652173911cm ,-58.451086956521735cm);
	\node[] at (7.47961956521739cm ,-59.14232336956522cm) {
		\scalebox{0.9881422924901185}{0}
	};
	\draw[color=color9, -latex] (8.40353260869565cm ,-59.53804347826087cm) -- (8.40353260869565cm ,-58.994565217391305cm) -- (9.307065217391303cm ,-58.994565217391305cm) -- (9.307065217391303cm ,-58.451086956521735cm);
	\node[] at (8.620923913043477cm ,-59.14232336956522cm) {
		\scalebox{0.9881422924901185}{1}
	};
	\draw[color=color8, -latex] (6.786684782608694cm ,-57.63586956521739cm) -- (6.786684782608694cm ,-57.09239130434783cm) -- (6.902173913043477cm ,-57.09239130434783cm) -- (6.902173913043477cm ,-56.54891304347826cm);
	\node[] at (7.00407608695652cm ,-57.240149456521735cm) {
		\scalebox{0.9881422924901185}{0}
	};
	\draw[color=color8, -latex] (5.930706521739129cm ,-57.63586956521739cm) -- (5.930706521739129cm ,-57.09239130434783cm) -- (5.815217391304346cm ,-57.09239130434783cm) -- (5.815217391304346cm ,-56.54891304347826cm);
	\node[] at (6.148097826086955cm ,-57.240149456521735cm) {
		\scalebox{0.9881422924901185}{1}
	};
\end{tikzpicture}

	\end{adjustbox}
	\caption{Fixup code for the loop from \Cref{fig:impl:fixup:fixup-firm-loop}, given $f = 2$ in \libFIRM}\label{fig:impl:fixup:duff:fixup-firm}
\end{figure}

\subsection{Loop duplication}\label{sec:impl:fixup:loop}

Another, perhaps simpler, way of creating fixup code is to duplicate the original loop, such that it will run $\Mloop$ times after the unrolled loop.
Just like when using the generalized form of Duff's device, we unroll the loop using the existing mechanics by a factor of $f$.
Therefore, Equations~\ref{eqn:impl:fixup:duff:conserve-semantics-identity} through~\ref{eqn:impl:fixup:duff:i-after-loop} still hold true.

The approach now taken is to copy the original loop, change its initial value to $\ipl$ and use it as fixup code, as seen in \Cref{fig:impl:fixup:loop:fixup-loop}.

\begin{figure}[H]
    \begin{algorithmic}
        \State $i \gets \ipl$
        \While{$i~`cmp`~N$}
            \State \Call{Body}{}
            \State $i \gets i + c$
        \EndWhile
    \end{algorithmic}
    \caption{The loop to run the body the remaining $\Mfixup$ times}
    \label{fig:impl:fixup:loop:fixup-loop}
\end{figure}

\begin{proof}
    To prove that this fixup code preserves semantics, first note that $\Mloop \overset{\ref{eqn:impl:fixup:duff:loop-iterations}}{\leq} M$.
    Then consider two cases:
    \begin{enumerate}
        \item $\Mloop = M$
        \item $\Mloop < M$
    \end{enumerate}
    In the first case, $\ipl~`cmp`~N$ must be false, as otherwise the unrolled loop would have broken semantics, and hence the new loop is never run.
    Therefore: $\Mfixup = 0 \Rightarrow \Mloop + \Mfixup = M$

    In the second case, the new loop runs until the condition is met.
    As the unrolled loop kept the increment semantics intact, the result is hence $\Mfixup = M - \Mloop$, which conserves the semantics, as per \Cref{eqn:impl:fixup:duff:conserve-semantics-identity}.
\end{proof}

\Cref{alg:impl:fixup:loop:fixup-loop} shows how we create this structure in~\libFIRM.
Firstly, we copy the loop, after which we rewire it, such that the fixup loop points to it, and its old successors point to the fixup loop.
Once this is completed, we can unroll the original loop.
\Cref{fig:impl:fixup:loop:general-loop} shows the resulting structure of the entire process in pseudo-code notation.

\Cref{fig:impl:fixup:loop:fixup-firm-comp} shows the result for unroll the loop from \Cref{fig:impl:fixup:fixup-firm-loop} using loop duplication fixup code and a factor of two.

\begin{figure}[H]
	\begin{algorithmic}
		\State $i \gets 0$
		\While{i < 21}
			\State \Call{Print}{$\text{HelloWorld}$}
			\State $i \gets i + 3$
			\State \Call{Print}{$\text{HelloWorld}$}
			\State $i \gets i + 3$
		\EndWhile
		\While{i < 29}
			\State \Call{Print}{$\text{HelloWorld}$}
			\State $i \gets i + 3$
		\EndWhile
	\end{algorithmic}
	\caption{An example loop, as seen in \Cref{fig:impl:fixup:fixup-firm-loop}, unrolled by a factor of two, and with loop duplication fixup}\label{fig:impl:fixup:loop:fixup-firm-comp}
\end{figure}

\begin{figure}[H]
    \centering
    \begin{algorithmic}
        \Function{Foo}{$I \in \mathbb{Z}, N \in \mathbb{Z}, c \in \mathbb{Z} \backslash \zeroset, cmp \in \{<, >, \leq, \geq\}$}
        \State $i \gets I$
        \If{$\neg$\Call{SubtractionWillLeaveBounds}{$N - c \cdot (f - 1)$}}
            \While{$i~`cmp`~ (N - c \cdot (f - 1))$}
                \State \Call{DoSomething}{} \Comment{$f$ times}
                \State $i \gets i + c$ \Comment{$f$ times}
            \EndWhile
        \EndIf
        \While{$i~`cmp`~N$}
            \State \Call{DoSomething}{}
            \State $i \gets i + c$
        \EndWhile
        \EndFunction
    \end{algorithmic}
    \caption{The general form of loop starting at $I$ and counting in increments of $c$ up to $N$ transformed by the created loop unrolling with loop fixup}
    \label{fig:impl:fixup:loop:general-loop}
\end{figure}


\begin{algorithm}[H]
    \begin{algorithmic}
        \Function{CreateFixupLoop}{$loop: \text{Loop}$}
            \State $loop' \gets $ \Call{ExactCopy}{$loop$}
            \For{$succ \in loop.header.sucessors$}
                \If{$succ.loop \notin loop$}
                    \State $succ.predecessors = succ.predecessors \backslash \{loop.header\} \cup \{loop'.header\}$
                \EndIf
            \EndFor
            \For{$node \in loop.header$}
                \For{$succ \in loop.header.sucessors$}
                    \If{$succ.loop \notin loop$}
                        \State $succ.predecessors = succ.predecessors \backslash \{node\} \cup \{node.link\}$
                    \EndIf
                \EndFor
            \EndFor
            \For{$phi \in loop.header.phis$}
                \For{$pred \in phi.predecessors$}
                    \If{$pred \notin loop$}
                        \State $phi.link.predecessors =phi.link.predecessors \backslash \{pred\} \cup \{phi\}$
                    \EndIf
                \EndFor
            \EndFor
            \State $loop'.header.predecessors \gets \{loop.header.falseExit\}$
            \Comment{Now unroll $loop$}
        \EndFunction
    \end{algorithmic}
    \caption{The algorithm to create a fixup loop in~\libFIRM}
    \label{alg:impl:fixup:loop:fixup-loop}
\end{algorithm}

\newpage

\section{Selecting an unroll-factor}\label{sec:impl:sel-factor}

Previously the factor $f$ seemed like it was chosen somewhat arbitrarily.
Further, \Cref{sec:basics:unrolling} describes that there are multiple factors influencing performance of unrolled loops.
Using this, we devise a selection process.

As a convention, we will henceforth let size be the number of~\libFIRM-nodes in a given loop.
The -- admittedly straightforward -- algorithm tries to find a factor $f = 2^n, n \in \mathbb{N}_{>0}$, that minimizes the absolute difference between the unrolled size ($= f \cdot \text{original size}$), and a pre-determined maximum size.
\Cref{alg:impl:sel-factor:sel-factor} shows the procedure used to find these values.
It is to be noted that the algorithm can also return $0$ and $1$, which does not fit the definition of the $f$ described.
In the case that the algorithm returns one of these two values, we will interpret it as ``do not unroll''.

\begin{algorithm}[H]
    \begin{algorithmic}
        \Function{CalculateFactor}{$loop: \text{loop}, maxSize: \mathbb{N}_{>0}$}
            \State $loopSize \gets\text{ }$ \Call{CountNodes}{$loop$}
            \State $factorPlain \gets maxSize \div loopSize$
            \State $factor_h \gets\text{ }$ \Call{RoundToNextHigherPowerOfTwo}{$factorPlain$}
            \State $factor_l \gets factor_h \div 2$
            \State $size_{h,l} \gets loopSize \cdot factor_{h,l}$
            \If{$\norm{size_h - maxsize} < \norm{size_l - maxsize}$}
                \State \Return $factor_h$
            \Else
                \State \Return $factor_l$
            \EndIf
        \EndFunction
    \end{algorithmic}
    \caption{Algorithm to determine the optimal unroll factor}
    \label{alg:impl:sel-factor:sel-factor}
\end{algorithm}
\chapter{Evaluation}\label{sec:eval}

\section{Unrollability}\label{sec:eval:unrollability}

One of the primary goals of this thesis was to increase the number of loops that are unrollable with~\libFIRM.
To evaluate to what extent this goal was achieved, we ran the benchmark suite \texttt{spec2006}, and logged how many loops we encountered, how many of them were innermost loops, how many could be unrolled using the old method, and how many that were previously not unrollable can now be unrolled\footnote{N.B.: The test was conducted with a max loop size of $\infty$}.
Considering it is expected for many loops to have non-constant bounds, such as the length of a container data-structure (e.g., a list or an unbounded array), we predict the new optimization to cause a significant increase in unrollable loops.
\Cref{fig:eval:unrollability:cmp-unrollability} shows a table with the results.
We can see, as mentioned in \Cref{sec:basics:unrolling}, that prior to the new optimization, 5.87\% of the innermost loops could be unrolled.
Now we can unroll an additional 7.37\% of loops.
Contrasted to the baseline of the constant bound unrolled loops this is a 125.65\% increase.
This means we more than doubled the number of unrollable loops using our approach.
Furthermore, we note that more than 70\% of loops are innermost loops.
Thus, even if unrolling nested loops were advantageous -- which is highly doubted -- we would not miss out on many loops.


\begin{figure}[h]
    \begin{center}
        \begin{tabular}{lcccc}
            \toprule
            Type & \makecell{Total \\ count} & \makecell{Relative to \\ loops} & \makecell{Relative to \\ innermost} & \makecell{Relative to \\ constant bound \\ unrollable} \\
            \midrule
            Loops & 23948 & \textbf{100\%} & --- & --- \\
            \makecell[l]{Innermost} & 17014 & 71.05\% & \textbf{100\%} & --- \\
            \makecell[l]{Constant \\ bound unrollable~\cite{aebi18bachelorarbeit}} & 8819 & 36.83\% & 51.83\% & \textbf{100\%} \\
            \makecell[l]{Non-constant \\ bound unrollable} & 1428 & 5.96\% & 8.39\% & 16.19\% \\
            \bottomrule
        \end{tabular}
    \end{center}
    \caption{Comparing total loops, innermost loops and the old unrolling process to the newly implemented proceess in terms of loops unrolled.
    The considered loops were all the ones present within the \texttt{spec2006} benchmark suite.}
    \label{fig:eval:unrollability:cmp-unrollability}
\end{figure}

\section{Performance}\label{sec:eval:perf}

Even though a high unrollability is a noble goal, most compiler optimizations aim to improve the runtime of the binaries they produce.
In order to evaluate the optimization in this regard, \texttt{spec2006} is used as a benchmark suite and run on a machine with an Intel Core i7 6700 clocked at 3.4GHz.
We run the tests on the Ubuntu 16.04 operating system, with cparser~\cite{cparser} as the frontend for~\libFIRM, and the native \texttt{x86} backend of~\libFIRM{} in use.
We use the same setup as used in the referenced work~\cite{aebi18bachelorarbeit}, such that we can get as comparable results as possible.

To evaluate the performance gain, we run the new optimization in conjunction with the old unrolling (see \Cref{sec:impl:unrollability}), given that it is intended as an extension.
As a result of there being two approaches for the fixup, as seen in \Cref{sec:impl:fixup}, we will try both of these, to see if one or the other yields better binary runtimes.
Further, as described in~\Cref{sec:impl:sel-factor}, the maximum unrolled size determines  the scope of the optimization.
Therefore, all sizes~$l \in \{2^n, n \in \lbrack 5, 10 \rbrack \}$~are each tried for both the fixup code strategies.
The reason that we chose 32 as a lower bound, is that very small loops are already more than eight nodes in size and hence wouldn't be unrolled with a maximum size that is a smaller power of two.
In order to compensate for measurement uncertainties, all benchmarks run ten times, and the average ($\mu$), as well as the standard deviation ($\sigma$) will be recorded and discussed.
We will compare all results to the reference benchmark run, which itself is a run of \texttt{spec2006} without any loop unrolling turned on.
These reference results can be seen in \Cref{fig:eval:perf:ref}.

In order to evaluate our findings in terms of performance, we should compare them to unrollability broken down by benchmark.
\Cref{fig:eval:unrollability:cmp-unrollability-bench} shows the number of unrollable loops\footnote{Both constant and non-constant bound unrollable loops are considered together} compared to the total number of loops.
Like in \Cref{fig:eval:unrollability:cmp-unrollability}, we assume a maximum size of infinity to collect this data.
Seeing this data, we would suspect \texttt{bzip2}, \texttt{mcf} and to a lesser extent (even though it has the most unrollable loops in absolute terms) \texttt{h264ref} to have the most considerable speedup.

\begin{figure}[H]
    \begin{center}
        \begin{tabular}{lrrr}
            \toprule
            Benchmark & Loops & Unrollable loops & Compared to total loops\\
            \midrule
            perlbench & 3201 & 104 & 3.25\%\\
            bzip2 & 347 & 231 & 66.58\%\\
            gcc & 10207  & 495 & 4.85\%\\
            mcf & 74 & 42 & 56.76\%\\
            gobmk & 3966 & 387 & 9.76\%\\
            hmmer & 1412 & 239 & 16.93\%\\
            sjeng & 410 & 43 & 10.49\% \\
            libquantum & 213 & 27 & 12.68\%\\
            h264ref & 2459 & 684 & 27.82\%\\
            \bottomrule
        \end{tabular}
    \end{center}
    \caption{The unrollability broken down by \texttt{spec2006}'s benchmarks}
    \label{fig:eval:unrollability:cmp-unrollability-bench}
\end{figure}

\begin{figure}[h]
    \begin{center}
        \begin{tabular}{lrr}
            \toprule
            Benchmark & $\mu$ & $\sigma$\\
            \midrule
            perlbench & 245.18s & 0.62s\\
            bzip2 & 342.68s & 0.49s\\
            gcc & 181.74s & 0.45s\\
            mcf & 129.16s & 0.18s\\
            gobmk & 356.15s & 0.29s\\
            hmmer & 603.86s & 0.07s\\
            sjeng & 393.72s & 0.40s\\
            libquantum & 297.28s & 0.47s\\
            h264ref & 405.12s & 0.27s\\
            \bottomrule
        \end{tabular}
    \end{center}
    \caption{Results of spec2006 after running it using libfirm without any loop unrolling}
    \label{fig:eval:perf:ref}
\end{figure}

\subsection{Duff's device fixup}\label{sec:eval:perf:duff}

Figures~\ref{fig:eval:perf:duff:32} through~\ref{fig:eval:perf:duff:1024} show the results we obtained.
While for most benchmarks the results hover around the 100\% mark, with no significant benefit or drawback, \texttt{h264ref} seems to profit from unrolling with maximum sizes 32 and 64, by being close to 4.5\% faster.
Though on account of the ratios of all the other benchmarks only diverting by three percent or less from the reference runtimes, unrolling does not seem to have a significant effect on performance.

The standard deviations are less than $1\%$ across all tests, due to the highly controlled test environment.
Though they do not entirely account for the percentage deltas, which are small, yet measurable.
Further, we can expect a small percentage of systemic errors in our measurements due to system process scheduling and similar factors.
Runtimes are, independent of the maximum loop size, within $\lbrack 99\%, 101\% \rbrack$, so we can still consider them to be within the margin of error.

\subsection{Loop fixup}\label{sec:eval:perf:loop}

Figures~\ref{fig:eval:perf:loop:32} through~\ref{fig:eval:perf:loop:1024} show the results obtained for the unrolling run with the loop fixup code.
As was the case in~\Cref{sec:eval:perf:duff}, there does not seem to be any noticeable performance gain or loss in any benchmark, except for \texttt{h264ref}, which again sped up through unrolling by up to 5\%.
The other benchmarks were, compared to the reference, within the interval $\lbrack 99\%, 103\% \rbrack$.
It further becomes evident that there is no correlation between unrollability and performance gain, since, while \texttt{h264ref} has one of the highest unrollabilities and gains performance, \texttt{bzip2} and \texttt{mcf} have higher unrollabilities, yet see no improvement.

\begin{figure}[h]
    \begin{center}
        \begin{tabular}{lrrrrr}
            \toprule
            & \multicolumn{2}{c}{Result} & \multicolumn{2}{c}{Reference}\\
            Benchmark & $\mu$ & $\sigma$ & $\mu$ & $\sigma$ & Ratio to reference\\
            \midrule
            perlbench & 245.60s & 0.15s & 245.18s & 0.62s & 100.17\%\\
            bzip2 & 349.34s & 0.50s & 342.68s & 0.49s & 101.94\%\\
            gcc & 180.70s & 0.31s & 181.74s & 0.45s & 99.43\%\\
            mcf & 129.34s & 0.40s & 129.16s & 0.18s & 100.14\%\\
            gobmk & 354.00s & 0.42s & 356.15s & 0.29s & 99.40\%\\
            hmmer & 603.70s & 0.15s & 603.86s & 0.07s & 99.97\%\\
            sjeng & 390.68s & 0.23s & 393.72s & 0.40s & 99.23\%\\
            libquantum & 297.33s & 0.61s & 297.28s & 0.47s & 100.02\%\\
            h264ref & 383.62s & 0.73s & 405.12s & 0.27s & 94.69\%\\
            \midrule
            Average & & & & & 99.44\%\\
            \bottomrule
        \end{tabular}
    \end{center}
    \caption{Results of spec2006 after unrolling with a factor of 32 using the modified duff's device fixup strategy}
    \label{fig:eval:perf:duff:32}
\end{figure}
\begin{figure}[h]
    \begin{center}
        \begin{tabular}{lrrrrr}
            \toprule
            & \multicolumn{2}{c}{Result} & \multicolumn{2}{c}{Reference}\\
            Benchmark & $\mu$ & $\sigma$ & $\mu$ & $\sigma$ & Ratio to reference\\
            \midrule
            perlbench & 243.65s & 0.55s & 245.18s & 0.62s & 99.37\%\\
            bzip2 & 349.74s & 1.07s & 342.68s & 0.49s & 102.06\%\\
            gcc & 181.47s & 0.30s & 181.74s & 0.45s & 99.86\%\\
            mcf & 129.68s & 0.60s & 129.16s & 0.18s & 100.40\%\\
            gobmk & 354.12s & 0.22s & 356.15s & 0.29s & 99.43\%\\
            hmmer & 603.80s & 0.19s & 603.86s & 0.07s & 99.99\%\\
            sjeng & 390.77s & 0.43s & 393.72s & 0.40s & 99.25\%\\
            libquantum & 297.82s & 2.06s & 297.28s & 0.47s & 100.18\%\\
            h264ref & 383.16s & 0.49s & 405.12s & 0.27s & 94.58\%\\
            \midrule
            Average & & & & & 99.46\%\\
            \bottomrule
        \end{tabular}
    \end{center}
    \caption{Results of spec2006 after unrolling with a factor of 64 using the modified duff's device fixup strategy}
    \label{fig:eval:perf:duff:64}
\end{figure}
\begin{figure}[h]
    \begin{center}
        \begin{tabular}{lrrrrr}
            \toprule
            & \multicolumn{2}{c}{Result} & \multicolumn{2}{c}{Reference}\\
            Benchmark & $\mu$ & $\sigma$ & $\mu$ & $\sigma$ & Ratio to reference\\
            \midrule
            perlbench & 246.36s & 0.26s & 245.18s & 0.62s & 100.48\%\\
            bzip2 & 342.24s & 0.32s & 342.68s & 0.49s & 99.87\%\\
            gcc & 181.36s & 0.16s & 181.74s & 0.45s & 99.79\%\\
            mcf & 129.57s & 0.47s & 129.16s & 0.18s & 100.32\%\\
            gobmk & 356.85s & 0.31s & 356.15s & 0.29s & 100.19\%\\
            hmmer & 603.68s & 0.22s & 603.86s & 0.07s & 99.97\%\\
            sjeng & 394.15s & 0.12s & 393.72s & 0.40s & 100.11\%\\
            libquantum & 297.32s & 0.40s & 297.28s & 0.47s & 100.01\%\\
            h264ref & 401.97s & 0.33s & 405.12s & 0.27s & 99.22\%\\
            \midrule
            Average & & & & & 100.00\%\\
            \bottomrule
        \end{tabular}
    \end{center}
    \caption{Results of spec2006 after unrolling with a factor of 128 using the modified duff's device fixup strategy}
    \label{fig:eval:perf:duff:128}
\end{figure}
\begin{figure}[h]
    \begin{center}
        \begin{tabular}{lrrrrr}
            \toprule
            & \multicolumn{2}{c}{Result} & \multicolumn{2}{c}{Reference}\\
            Benchmark & $\mu$ & $\sigma$ & $\mu$ & $\sigma$ & Ratio to reference\\
            \midrule
            perlbench & 248.95s & 0.69s & 245.18s & 0.62s & 101.54\%\\
            bzip2 & 348.61s & 0.48s & 342.68s & 0.49s & 101.73\%\\
            gcc & 181.01s & 0.22s & 181.74s & 0.45s & 99.60\%\\
            mcf & 129.62s & 0.43s & 129.16s & 0.18s & 100.36\%\\
            gobmk & 355.51s & 0.22s & 356.15s & 0.29s & 99.82\%\\
            hmmer & 603.34s & 0.17s & 603.86s & 0.07s & 99.91\%\\
            sjeng & 396.56s & 0.37s & 393.72s & 0.40s & 100.72\%\\
            libquantum & 297.14s & 0.34s & 297.28s & 0.47s & 99.95\%\\
            h264ref & 402.13s & 0.47s & 405.12s & 0.27s & 99.26\%\\
            \midrule
            Average & & & & & 100.32\%\\
            \bottomrule
        \end{tabular}
    \end{center}
    \caption{Results of spec2006 after unrolling with a factor of 256 using the modified duff's device fixup strategy}
    \label{fig:eval:perf:duff:256}
\end{figure}
\begin{figure}[h]
    \begin{center}
        \begin{tabular}{lrrrrr}
            \toprule
            & \multicolumn{2}{c}{Result} & \multicolumn{2}{c}{Reference}\\
            Benchmark & $\mu$ & $\sigma$ & $\mu$ & $\sigma$ & Ratio to reference\\
            \midrule
            perlbench & 252.26s & 0.29s & 245.18s & 0.62s & 102.89\%\\
            bzip2 & 349.73s & 0.34s & 342.68s & 0.49s & 102.06\%\\
            gcc & 180.59s & 0.30s & 181.74s & 0.45s & 99.37\%\\
            mcf & 129.43s & 0.45s & 129.16s & 0.18s & 100.21\%\\
            gobmk & 353.90s & 0.25s & 356.15s & 0.29s & 99.37\%\\
            hmmer & 603.68s & 0.19s & 603.86s & 0.07s & 99.97\%\\
            sjeng & 390.69s & 0.22s & 393.72s & 0.40s & 99.23\%\\
            libquantum & 297.20s & 0.47s & 297.28s & 0.47s & 99.97\%\\
            h264ref & 392.48s & 0.71s & 405.12s & 0.27s & 96.88\%\\
            \midrule
            Average & & & & & 99.99\%\\
            \bottomrule
        \end{tabular}
    \end{center}
    \caption{Results of spec2006 after unrolling with a factor of 512 using the modified duff's device fixup strategy}
    \label{fig:eval:perf:duff:512}
\end{figure}
\begin{figure}[h]
    \begin{center}
        \begin{tabular}{lrrrrr}
            \toprule
            & \multicolumn{2}{c}{Result} & \multicolumn{2}{c}{Reference}\\
            Benchmark & $\mu$ & $\sigma$ & $\mu$ & $\sigma$ & Ratio to reference\\
            \midrule
            perlbench & 252.07s & 0.23s & 245.18s & 0.62s & 102.81\%\\
            bzip2 & 349.04s & 0.46s & 342.68s & 0.49s & 101.86\%\\
            gcc & 181.40s & 0.45s & 181.74s & 0.45s & 99.82\%\\
            mcf & 129.58s & 0.49s & 129.16s & 0.18s & 100.33\%\\
            gobmk & 356.92s & 0.23s & 356.15s & 0.29s & 100.22\%\\
            hmmer & 603.31s & 0.14s & 603.86s & 0.07s & 99.91\%\\
            sjeng & 390.46s & 0.29s & 393.72s & 0.40s & 99.17\%\\
            libquantum & 297.96s & 2.23s & 297.28s & 0.47s & 100.23\%\\
            h264ref & 396.44s & 0.26s & 405.12s & 0.27s & 97.86\%\\
            \midrule
            Average & & & & & 100.24\%\\
            \bottomrule
        \end{tabular}
    \end{center}
    \caption{Results of spec2006 after unrolling with a factor of 1024 using the modified duff's device fixup strategy}
    \label{fig:eval:perf:duff:1024}
\end{figure}

\begin{figure}[h]
    \begin{center}
        \begin{tabular}{lrrrrr}
            \toprule
            & \multicolumn{2}{c}{Result} & \multicolumn{2}{c}{Reference}\\
            Benchmark & $\mu$ & $\sigma$ & $\mu$ & $\sigma$ & Ratio to reference\\
            \midrule
            perlbench & 243.77s & 0.62s & 245.04s & 0.31s & 99.48\%\\
            bzip2 & 339.12s & 0.54s & 342.69s & 0.38s & 98.96\%\\
            gcc & 181.65s & 0.18s & 181.86s & 0.26s & 99.88\%\\
            mcf & 128.88s & 0.38s & 129.60s & 0.49s & 99.44\%\\
            gobmk & 357.52s & 0.47s & 355.54s & 0.09s & 100.56\%\\
            hmmer & 603.34s & 0.09s & 603.98s & 0.13s & 99.89\%\\
            sjeng & 393.71s & 0.31s & 393.89s & 0.32s & 99.95\%\\
            libquantum & 297.27s & 0.47s & 297.30s & 0.56s & 99.99\%\\
            h264ref & 402.55s & 0.54s & 402.67s & 0.44s & 99.97\%\\
            \midrule
            Average & & & & & 99.79\%\\
            \bottomrule
        \end{tabular}
    \end{center}
    \caption{Results of spec2006 after unrolling with maximum size 32 using the loop fixup strategy}
    \label{fig:eval:perf:loop:32}
\end{figure}
\begin{figure}[h]
    \begin{center}
        \begin{tabular}{lrrrrr}
            \toprule
            & \multicolumn{2}{c}{Result} & \multicolumn{2}{c}{Reference}\\
            Benchmark & $\mu$ & $\sigma$ & $\mu$ & $\sigma$ & Ratio to reference\\
            \midrule
            perlbench & 252.32s & 0.63s & 245.04s & 0.31s & 102.97\%\\
            bzip2 & 349.91s & 1.04s & 342.69s & 0.38s & 102.11\%\\
            gcc & 180.74s & 0.32s & 181.86s & 0.26s & 99.38\%\\
            mcf & 129.65s & 0.62s & 129.60s & 0.49s & 100.04\%\\
            gobmk & 353.93s & 0.16s & 355.54s & 0.09s & 99.55\%\\
            hmmer & 603.71s & 0.24s & 603.98s & 0.13s & 99.96\%\\
            sjeng & 390.49s & 0.28s & 393.89s & 0.32s & 99.14\%\\
            libquantum & 298.51s & 2.76s & 297.30s & 0.56s & 100.41\%\\
            h264ref & 392.60s & 0.29s & 402.67s & 0.44s & 97.50\%\\
            \midrule
            Average & & & & & 100.12\%\\
            \bottomrule
        \end{tabular}
    \end{center}
    \caption{Results of spec2006 after unrolling with maximum size 64 using the loop fixup strategy}
    \label{fig:eval:perf:loop:64}
\end{figure}
\begin{figure}[h]
    \begin{center}
        \begin{tabular}{lrrrrr}
            \toprule
            & \multicolumn{2}{c}{Result} & \multicolumn{2}{c}{Reference}\\
            Benchmark & $\mu$ & $\sigma$ & $\mu$ & $\sigma$ & Ratio to reference\\
            \midrule
            perlbench & 251.88s & 0.21s & 245.04s & 0.31s & 102.79\%\\
            bzip2 & 349.04s & 0.45s & 342.69s & 0.38s & 101.85\%\\
            gcc & 181.25s & 0.22s & 181.86s & 0.26s & 99.67\%\\
            mcf & 129.86s & 0.52s & 129.60s & 0.49s & 100.20\%\\
            gobmk & 356.96s & 0.06s & 355.54s & 0.09s & 100.40\%\\
            hmmer & 603.31s & 0.14s & 603.98s & 0.13s & 99.89\%\\
            sjeng & 390.44s & 0.21s & 393.89s & 0.32s & 99.12\%\\
            libquantum & 297.42s & 0.41s & 297.30s & 0.56s & 100.04\%\\
            h264ref & 396.57s & 0.30s & 402.67s & 0.44s & 98.49\%\\
            \midrule
            Average & & & & & 100.27\%\\
            \bottomrule
        \end{tabular}
    \end{center}
    \caption{Results of spec2006 after unrolling with maximum size 128 using the loop fixup strategy}
    \label{fig:eval:perf:loop:128}
\end{figure}
\begin{figure}[h]
    \begin{center}
        \begin{tabular}{lrrrrr}
            \toprule
            & \multicolumn{2}{c}{Result} & \multicolumn{2}{c}{Reference}\\
            Benchmark & $\mu$ & $\sigma$ & $\mu$ & $\sigma$ & Ratio to reference\\
            \midrule
            perlbench & 246.02s & 0.70s & 245.04s & 0.31s & 100.40\%\\
            bzip2 & 349.61s & 0.38s & 342.69s & 0.38s & 102.02\%\\
            gcc & 180.83s & 0.25s & 181.86s & 0.26s & 99.43\%\\
            mcf & 129.77s & 0.30s & 129.60s & 0.49s & 100.13\%\\
            gobmk & 353.86s & 0.22s & 355.54s & 0.09s & 99.53\%\\
            hmmer & 603.65s & 0.16s & 603.98s & 0.13s & 99.95\%\\
            sjeng & 390.45s & 0.32s & 393.89s & 0.32s & 99.13\%\\
            libquantum & 297.06s & 0.25s & 297.30s & 0.56s & 99.92\%\\
            h264ref & 383.45s & 0.18s & 402.67s & 0.44s & 95.23\%\\
            \midrule
            Average & & & & & 99.53\%\\
            \bottomrule
        \end{tabular}
    \end{center}
    \caption{Results of spec2006 after unrolling with maximum size 256 using the loop fixup strategy}
    \label{fig:eval:perf:loop:256}
\end{figure}
\begin{figure}[h]
    \begin{center}
        \begin{tabular}{lrrrrr}
            \toprule
            & \multicolumn{2}{c}{Result} & \multicolumn{2}{c}{Reference}\\
            Benchmark & $\mu$ & $\sigma$ & $\mu$ & $\sigma$ & Ratio to reference\\
            \midrule
            perlbench & 243.47s & 0.36s & 245.04s & 0.31s & 99.36\%\\
            bzip2 & 349.52s & 0.59s & 342.69s & 0.38s & 101.99\%\\
            gcc & 181.78s & 0.53s & 181.86s & 0.26s & 99.96\%\\
            mcf & 129.75s & 0.46s & 129.60s & 0.49s & 100.11\%\\
            gobmk & 354.08s & 0.29s & 355.54s & 0.09s & 99.59\%\\
            hmmer & 603.63s & 0.06s & 603.98s & 0.13s & 99.94\%\\
            sjeng & 390.44s & 0.25s & 393.89s & 0.32s & 99.12\%\\
            libquantum & 297.29s & 0.36s & 297.30s & 0.56s & 100.00\%\\
            h264ref & 383.05s & 0.20s & 402.67s & 0.44s & 95.13\%\\
            \midrule
            Average & & & & & 99.47\%\\
            \bottomrule
        \end{tabular}
    \end{center}
    \caption{Results of spec2006 after unrolling with maximum size 512 using the loop fixup strategy}
    \label{fig:eval:perf:loop:512}
\end{figure}
\begin{figure}[h]
    \begin{center}
        \begin{tabular}{lrrrrr}
            \toprule
            & \multicolumn{2}{c}{Result} & \multicolumn{2}{c}{Reference}\\
            Benchmark & $\mu$ & $\sigma$ & $\mu$ & $\sigma$ & Ratio to reference\\
            \midrule
            perlbench & 248.73s & 0.53s & 245.04s & 0.31s & 101.51\%\\
            bzip2 & 348.71s & 0.46s & 342.69s & 0.38s & 101.76\%\\
            gcc & 181.01s & 0.27s & 181.86s & 0.26s & 99.53\%\\
            mcf & 129.49s & 0.51s & 129.60s & 0.49s & 99.92\%\\
            gobmk & 355.54s & 0.13s & 355.54s & 0.09s & 100.00\%\\
            hmmer & 603.39s & 0.13s & 603.98s & 0.13s & 99.90\%\\
            sjeng & 396.64s & 0.32s & 393.89s & 0.32s & 100.70\%\\
            libquantum & 297.20s & 0.36s & 297.30s & 0.56s & 99.97\%\\
            h264ref & 402.27s & 0.58s & 402.67s & 0.44s & 99.90\%\\
            \midrule
            Average & & & & & 100.35\%\\
            \bottomrule
        \end{tabular}
    \end{center}
    \caption{Results of spec2006 after unrolling with maximum size 1024 using the loop fixup strategy}
    \label{fig:eval:perf:loop:1024}
\end{figure}

\chapter{Conclusion}\label{sec:conclusion}

The results, discussed in \Cref{sec:eval:perf}, fall in line with the results from the ones for static bound unrolling~\cite{aebi18bachelorarbeit}.
They, therefore, suggest, that there is now empirical evidence that independent of the unrolling method, and the factor chosen, loop unrolling does not yield a significant performance benefit in the current state of~\libFIRM.
Even though more loops were able to be unrolled through the added loop optimization, the increase in unrollability only led to about one in ten loops being unrolled, which certainly is a contributing factor to the underwhelming improvements.
Probably some restrictions, such as disallowing \texttt{break}-like structures, are too limited and could be dealt with through further development.
Other restrictions, such as the conservative alias or call manipulation checks for the bound are unavoidable if the semantics are to be kept and forthright inherent to the task at hand.
Inconsiderate of these reasons, even the benchmarks with high unrollability of their loops, did not seem to benefit (with \texttt{h264ref} being an exception).
Further, it can be concluded that the choice of the fixup code strategy seems to have a negligible impact on performance.
Due to the very low standard deviations across all benchmarks, the results also lead to a firm belief the obtained results are trustable and hence provide a solid foundation for empirical conclusions.

Thus, the eminent challenge seems to be the lack of performance gain through unrolling loops.
Therefore, it would be a natural starting point to use the unrolled loops and optimize their bodies further.
An optimization could be created that takes advantage of the implicitly added semantics as for having a specific modulus, respective to $f$, for each unrolled block.
Before this potential is used, it likely would be a more lucrative endeavor, to stick to less fancy optimizations that can take advantage of the unrolled loop structures, such as automatically parallelizing non-conflicting operations.

Another factor that might have influenced the results was the method used to determine the unroll-factor.
In the future, it could be evaluated, whether the performance would improve through a more sophisticated unroll-factor selection, with a multi-parameter cost function.

Once these changes are in-place, the feasibility loop unrolling in~\libFIRM{} should be reevaluated.

Currently, the efforts of increasing unrollable loops seem to exceed the benefits.
Though, if the desire for more unroallability should pick up again, it would seem a good point to look at other loop structures, such as loops with breaks, or a non-counting loop, unlike the ones examined in this thesis.


\bibliographystyle{ieeetr}
\bibliography{bib}

\begin{otherlanguage}{ngerman}
\chapter*{Erklärung}
\pagestyle{empty}

  \vspace{20mm}
  Hiermit erkläre ich, \theauthor, dass ich die vorliegende Bachelorarbeit selbst\-ständig
verfasst habe und keine anderen als die angegebenen Quellen und Hilfsmittel
benutzt habe, die wörtlich oder inhaltlich übernommenen Stellen als solche kenntlich gemacht und
die Satzung des KIT zur Sicherung guter wissenschaftlicher Praxis beachtet habe.
  \vspace{20mm}
  \begin{tabbing}
  \rule{4cm}{.4pt}\hspace{1cm} \= \rule{7cm}{.4pt} \\
 Ort, Datum \> Unterschrift
  \end{tabbing}
\end{otherlanguage}


\pagestyle{fancy}
\appendix

\chapter{Sonstiges}

\section{Anmeldung}

Üblicherweise melden wir eine Arbeit erst an,
wenn der Student mit dem Schreiben begonnen hat,
also nach der Implementierung.
Das verringert die Bürokratie und den Stress,
der mit verpassten Deadlines kommt.

Außerdem ist ein Abbrechen nach der Anmeldung ein offizieller Akt
für den es wiederum Fristen gibt:

\begin{center}
\begin{tabular}{lrrr}
\toprule
 & Abbruchfrist nach Anmeldung \\
\midrule
Bachelor      & 4 Wochen \\
Master        & 2 Monate \\
Diplom        & 3 Monate \\
\bottomrule
\end{tabular}
\end{center}

Nach dieser Frist muss die abgebrochene Arbeit mit 5,0 bewertet werden.

Das ISS empfiehlt, dass Studenten sich zusätzlich selbst im Studienportal anmelden.
Das könnte die Eintragung der Note beschleunigen.

\section{Antrittsvortrag}

Bei internen Arbeiten jeglicher Art ist ein Antrittsvortrag optional.
Bei externen Arbeiten ist ein Antrittsvortrag Pflicht.

Dauer: 15 Minuten + 5 Minuten Fragezeit.

Ein Antrittsvortrag sollte nach der Einarbeitungsphase stattfinden,
wenn man einen Überblick hat und weiß was man vorhat.
Im Antrittsvortrag kann man abtasten was Prof.~Snelting von dem Thema hält
und wo man Schwerpunkte setzen oder erweitern sollte.

\section{Abgabe}

\begin{center}
\begin{tabular}{lrrr}
\toprule
 & Dauer & Umfang \\
\midrule
Bachelor      & 4 Monate & 30+ Seiten \\
Master        & 6 Monate & 50+ Seiten \\
Studienarbeit & 3 Monate & 30+ Seiten \\
Diplom        & 6 Monate & 50+ Seiten \\
\bottomrule
\end{tabular}
\end{center}

Man kann eine "4.0 Bescheinigung" bekommen,
bspw.\ für die Masteranmeldungen.

Abzugeben sind jeweils 4 gedruckte Examplare der Arbeit,
das Dokument als pdf Datei
und entstandener Code und andere Artefakte.
Außerdem könnten spätere Studenten dankbar sein für \TeX-Sourcen.

Zum Drucken empfehlen wir
Katz Copy\footnote{\url{http://www.katz-copy.com/}} am Kronenplatz,
weil wir in Sachen Qualität dort die besten Erfahrungen gemacht haben.
Bitte keine Spiralbindung,
da sich das schlecht Stapeln lässt.
Farbdruck ist nicht verpflichtend,
solange in Schwarzweiß noch alle Grafiken lesbar sind.

\section{Abschlussvortrag}

Die Abschlusspräsentation dauert für Bachelorarbeiten 15 Minuten
zuzüglich mind. 10 Minuten für Fragen.
Bei Masterarbeiten sind 20--25 Minuten für den Vortrag vorgesehen.

Der Vortrag soll innerhalb von vier Wochen nach Abgabe erfolgen,
entsprechend Prüfungsordnung.
Die Arbeit muss mindestens einen Tag vor dem Abschlussvortrag abgegeben sein,
damit sich Prof.~Snelting vorbereiten kann.

Am besten direkt im Anschluß den Vortrag ausarbeiten und ein oder zwei Wochen nach Abgabe halten.
Der Präsentationstermin muss ein bis zwei Monate im Voraus geplant werden,
denn Prof.~Snelting hat üblicherweise einen vollen Terminkalender.

\section{Gutachten}

Der Prüfer erstellt ein Gutachten zur Arbeit.
Um das Gutachten einzusehen muss ein Antrag beim Prüfungsamt gestellt werden.
Der Betreuer bzw. Prüfer darf das Gutachten nur mit genehmigtem Gutachten zeigen.
Mündliche Auskunft zur Note ist allerdings möglich.

\section{Bewertung}

\begin{itemize}
  \item Diplom- und Masterarbeiten \emph{müssen} eine wissenschaftliche Komponente enthalten.
    Bachelorarbeit \emph{sollten}, aber zum Bestehen ist es nicht notwendig.
    Wissenschaftlich ist was über reine Implementierungs- bzw. Softwareentwicklungsaufgaben hinausgeht.
    Üblicherweise findet man theoretische Betrachtungen zu Korrektheit und Effizienz.
    Willkürliche Daumenregel: Ohne Formel, keine Wissenschaft.
  \item Diplom- und Masterarbeiten benötigen eigentlich immer Wissen aus dem Diplom- bzw. Masterstudium.
    Falls das Wissen aus Vordiplom bzw. Bachelor ausreicht,
    sollte man nochmal darüber nachdenken.
  \item Positiv mit der Note korrelieren
    selbstständiges Arbeiten,
    regelmäßige Abstimmung mit dem Betreuer,
    mehrere Feedbackrunden mit verschiedenen Leuten,
    mehrmaliges Üben des Abschlussvortrags,
    Einbringen eigener Ideen,
    gutes Zuhören
    und sorgfältiges Debugging.
  \item Negativ mit der Note korrelieren
    wochenlanges Pausieren,
    Ignorieren von Feedback,
    Deadlines überziehen
    und Arbeiten im stillen Kämmerchen.
\end{itemize}

Disclaimer:
Nein, es gibt keinen konkreten Notenschlüssel.
Die obigen Punkte sind nur grobe Richtlinien und für niemanden in irgendeiner Weise bindend.

\section{\LaTeX\ Features}

\subsection{Schriftformatierungen}

\begin{tabular}{lccc}
\toprule
 & serif & sans-serif & fixed-width \\
\midrule
normal        & \textrm{\textup{Medium}} \textrm{\textup{\textbf{Bold}}} & \textsf{\textup{Medium}} \textsf{\textup{\textbf{Bold}}} & \texttt{\textup{Medium}} \texttt{\textup{\textbf{Bold}}} \\
italic        & \textrm{\textit{Medium}} \textrm{\textit{\textbf{Bold}}} & \textsf{\textit{Medium}} \textsf{\textit{\textbf{Bold}}} & \texttt{\textit{Medium}} \texttt{\textit{\textbf{Bold}}}\\
slanted       & \textrm{\textsl{Medium}} \textrm{\textsl{Bold}} & \textsf{\textsl{Medium}} \textsf{\textsl{\textbf{Bold}}} & \texttt{\textsl{Medium}} \texttt{\textsl{\textbf{Bold}}} \\
small-capital & \textrm{\textsc{Medium}} \textrm{\textsc{\textbf{Bold}}} & \textsf{\textsc{Medium}} \textsf{\textsc{\textbf{Bold}}} & \texttt{\textsc{Medium}} \texttt{\textsc{\textbf{Bold}}} \\
\bottomrule
\end{tabular}

Math fonts:
$\mathnormal{absXYZ}$,
$\mathrm{absXYZ}$,
$\mathbf{absXYZ}$,
$\mathsf{absXYZ}$,
$\mathit{absXYZ}$,
$\mathtt{absXYZ}$, and
$\mathcal{XYZ}$.

\subsection{Rand und Platz}

Viele Benutzer von \LaTeX\ wollen Ränder und Seitengröße anpassen.
Dazu empfehlen wir erstmal die KOMA Script Dokumentation (\texttt{koma-script.pdf}) zu lesen,
insbesondere Kapitel 2.2.
Bevor man mit \texttt{\textbackslash enlargethispage}
oder ähnlichen Tricks anfängt,
sollte man \texttt{\textbackslash typearea} anpassen.

Falls die Arbeit auf Englisch verfasst wird,
sollte man wissen, dass Absätze im Englischen üblicherweise anders formatiert werden.
Im Deutschen macht man eine Leerzeile zwischen Absätzen.
Im Englischen wird stattdessen die erste Zeile eines Absatzes eingerückt.


\end{document}
